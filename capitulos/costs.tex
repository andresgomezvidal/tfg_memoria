\section{Cost management}
In this section an estimation of the costs of the project is made based on the relevant data available at the end of the project.
\linej
The project is set in Spain, therefore the currency used is the Euro (\euro{}). This is used up to a precision of cents.
\linej
Even though Tarlogic is considered the client there is no payment to the student because this is the final project of a degree.
\linej
The costs were divided into \textit{direct costs} and \textit{indirect costs}.

\subsection{Direct costs}
All the software used in the project was free, including the Windows images (they were free trials).
\linej
\linej
Other materials are the CD to burn the software of the project and the version of this document in paper.
Their combined cost is estimated in 20\euro{}.
\linej
\linej
The rest of the direct costs are the human and hardware resources.

\subsubsection{Human resources}
%this needs the xfp package, for some reason the method with \the\numexpr operation_here \relax that works in planning does not here
\newcommand{\hourBias}{\the\numexpr 20*14*8 \relax}
\newcommand{\ProjectManagerAnnual}{45448}
\newcommand{\SeniorEngineerAnnual}{29754}
\newcommand{\SysAdminAnnual}{24234}
\newcommand{\JuniorDeveloperAnnual}{16000}
\newcommand{\ProjectManagerSC}{   \fpeval{trunc(\ProjectManagerAnnual * 0.32 ,2)}}
\newcommand{\ProjectManagerTotal}{\fpeval{trunc(\ProjectManagerAnnual * 1.32 ,2)}}
\newcommand{\ProjectManagerHour}{ \fpeval{trunc(\ProjectManagerTotal  / \hourBias ,2)}}
\newcommand{\SeniorEngineerSC}{\fpeval{trunc(   \SeniorEngineerAnnual * 0.32 ,2)}}
\newcommand{\SeniorEngineerTotal}{\fpeval{trunc(\SeniorEngineerAnnual * 1.32 ,2)}}
\newcommand{\SeniorEngineerHour}{\fpeval{trunc( \SeniorEngineerTotal  / \hourBias ,2)}}
\newcommand{\SysAdminSC}{\fpeval{trunc(   \SysAdminAnnual * 0.32 ,2)}}
\newcommand{\SysAdminTotal}{\fpeval{trunc(\SysAdminAnnual * 1.32 ,2)}}
\newcommand{\SysAdminHour}{\fpeval{trunc( \SysAdminTotal  / \hourBias ,2)}}
\newcommand{\JuniorDeveloperSC}{\fpeval{trunc(   \JuniorDeveloperAnnual * 0.32 ,2)}}
\newcommand{\JuniorDeveloperTotal}{\fpeval{trunc(\JuniorDeveloperAnnual * 1.32 ,2)}}
\newcommand{\JuniorDeveloperHour}{\fpeval{trunc( \JuniorDeveloperTotal  / \hourBias ,2)}}
\newcommand{\TutorHours}{11.25}
\newcommand{\ProjectManagerRoleHours}{\fpeval{trunc( \TutorHours * 2 ,2)}}
\newcommand{\ProjectManagerHourTotal}{\fpeval{trunc( \ProjectManagerHour * \TutorHours * 2 ,2)}}
\newcommand{\SeniorEngineerHourTotal}{\fpeval{trunc( \SeniorEngineerHour * \TutorHours ,2)}}
\newcommand{\SysAdminHourTotal}{\fpeval{trunc( \SysAdminHour * \TutorHours ,2)}}
\newcommand{\JuniorDeveloperHourTotal}{\fpeval{trunc( \JuniorDeveloperHour * \projecthours ,2)}}
\newcommand{\HRTotal}{\fpeval{trunc( \ProjectManagerHourTotal + \SeniorEngineerHourTotal + \SysAdminHourTotal + \JuniorDeveloperHourTotal,2)}}
The human resources are the student, the director from the University, the director from Tarlogic and two members of Tarlogic that helped the student with issues at several points in the project.
\linej
\linej
The next salary parameters are assumed:
\begin{itemize}
	\item There are 14 payments per year.
	\item Social Security costs 32\% of the brute salary.
	\item The job journey is 8 hours of work per day and there are 20 working days per month.
\end{itemize}
\linej
The estimation of the hours that the student worked on the project comes from:
\begin{itemize}
	\item The worked time while the project was on hold is zero.
	\item In the first weeks of the project the the estimation of hours worked per week is 21. The project started the day 1 of November and was put on hold 35 days later.
	\item After returning to work on the project on the 25 of February the estimation of hours worked per week is 11, and lasted for 68 days until the project was on hold for two weeks.
	\item In the last stage of the project the estimation of hours worked per week is 21 again. This lasted 65 days, from the 20 of May until the end of the project.
\end{itemize}
\linej
Therefore the estimation is: $(35+65) \cdot 21/7 + 68 \cdot 11/7 + \TutorHours$ = 418.10 hours.
\linej
\linej
The estimation of the hours that the directors and the involved members of Tarlogic is \TutorHours hours.
This value is the usual estimation of hours for tutoring and evaluation.
\linej
\linej
The annual brute income for the job titles are taken from the website \textit{Indeed}\cite{indeed}, which calculates the average of hundreds of job offers in the last months.
In the case of the student the salary was an optimistic value.
\linej
The job title assumed for the student is Junior Developer.
The job title assumed for the directors is Project Manager.
The Project Manager has \ProjectManagerRoleHours hours because this project has 2 directors.
\linej
\begin{table}[H]
	\begin{tabularx}{\textwidth}{|X|l|l|l|l|}
		\hline
		\rowcolor{gray!30}
		Role                             & Brute annual                 & Social Security          &    Total annual             & Cost/Hour\\ \hline
		Project Manager                  & \ProjectManagerAnnual\euro{} & \ProjectManagerSC\euro{} & \ProjectManagerTotal\euro{} & \ProjectManagerHour\euro{}\\ \hline
		Senior Engineer in Cybersecurity & \SeniorEngineerAnnual\euro{} & \SeniorEngineerSC\euro{} & \SeniorEngineerTotal\euro{} & \SeniorEngineerHour\euro{}\\ \hline
		System Administrator             & \SysAdminAnnual\euro{}       & \SysAdminSC\euro{}       & \SysAdminTotal\euro{}       & \SysAdminHour\euro{}\\ \hline
		Junior Developer                 & \JuniorDeveloperAnnual\euro{}& \JuniorDeveloperSC\euro{}& \JuniorDeveloperTotal\euro{}& \JuniorDeveloperHour\euro{}\\ \hline
	\end{tabularx}
	\caption{Annual costs of the human resources}
\end{table}
\begin{table}[H]
	\begin{tabularx}{\textwidth}{|X|l|l|l|}
		\hline
		\rowcolor{gray!30}
		Role                             & Hours                  & Cost/Hour                  & Total cost\\ \hline
		Project Manager                  & \fpeval{\TutorHours*2} & \ProjectManagerHour\euro{} & \ProjectManagerHourTotal\euro{}\\ \hline
		Senior engineer in cybersecurity & \TutorHours            & \SeniorEngineerHour\euro{} & \SeniorEngineerHourTotal\euro{}\\ \hline
		System administrator             & \TutorHours            & \SysAdminHour\euro{}       & \SysAdminHourTotal\euro{}\\ \hline
		Junior Developer                 & \projecthours          & \JuniorDeveloperHour\euro{}& \JuniorDeveloperHourTotal\euro{}\\ \hline
	\end{tabularx}
	\caption{Costs of the human resources for the hours dedicated to the project}
\end{table}
\linej
The total of the last table is \HRTotal\euro{}.

\subsubsection{Hardware resources}
The cost of the hardware is calculated with the amortization and the total price of the item, with the next formula:
\begin{figure}[H]
	\[ \frac{Price\ of\ the\ item}{12 \cdot Duration\ of\ the\ item\ in\ years} \cdot Duration\ of\ the\ project\ in\ months\ \]
	\caption{Formula to calculate the cost of hardware items}
\end{figure}
\linej
In this case it is assumed that the duration of project is 5.5 months, due to the time the project was on hold.
The duration of the hardware items is estimated in 3-5 years, so the average is used in this case for all the items.
\linej
\linej
The hardware items are:
\begin{itemize}
	\item A computer leaning to the higher end in order to run the set of virtual machines: with 20 GB of RAM, an \textit{i5-2500k} processor and about 500GB of free disk storage (of which 100GB were of Solid State Disk). The estimation of the computer is hard to make because it is several years old and made of parts buyed years apart from each other. The value estimated is 1250\euro{}, therefore resulting in a cost of 143.23\euro{}.
	\item A monitor of 24 inches valued in 147.75\euro{} last year. Using the previous formula the cost is 16.93\euro{}.
	\item Other peripheral devices: their collective value is estimated in 20\euro{}, therefore resulting in a cost of 2.23\euro{}.
\end{itemize}
\linej
\newcommand{\HardwareTotal}{162.39}
\newcommand{\DirectCosts}{\fpeval{trunc( \HRTotal + \HardwareTotal ,2)}}
\newcommand{\IndirectCosts}{\fpeval{trunc( \DirectCosts * 0.2 ,2)}}
\newcommand{\CostsTotal}{\fpeval{trunc( \DirectCosts * 1.2 ,2)}}
The total of the hardware cost is \HardwareTotal\euro{}.

\subsection{Indirect costs}
The indirect costs of the project mean hidden costs in the resources used.
For example the cost of the Internet connection, electrical devices, etc.
\linej
According to the General Secretary of the University of Santiago de Compostela the indirect costs in this kind of final degree project should be calculated as an extra 20\% of the direct costs\cite{indirect_costs}.
\linej
\linej
Because the estimation of the direct costs is \DirectCosts\euro{}, the indirect costs adds \IndirectCosts\euro{} over it.

\subsection{Total costs of the project}
The total costs is calculated simply by the addition of direct and indirect costs, resulting in \CostsTotal\euro{}.
