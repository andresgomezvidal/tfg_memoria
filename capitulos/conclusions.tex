\section{Conclusion}
This project has shown how Wazuh can be used to improve the cybersecurity of a system, in some cases without the need of expert knowledge in scripting or pentesting.
\linej
Wazuh has been under development for some years and there are still parts that lack basic funcionality.
This project has tried to show and fix some of them, without changing the source code in order to have more time for other tasks.
\linej
\linej
The main use of a HIDS should be to gather and process data in order to detect security threats.
Wazuh provides a set of features that make it a very good tool for monitoring system behaviour, but not as good in other aspects.
\linej
Stopping the threats can be too much for this kind of system, there are obvious limitations that make it impossible to be as effective as a local antimalware software.
Trusting user scripts for defensive security is very risky and bound to fail sooner or later.
\linej
\linej
Even though in this project only the server runs GNU/Linux is clear from its management and the documentation that Wazuh is less suited for Windows systems.
Many workarounds and third-party tools are needed because of this.
Therefore it is assumed that the project could have advanced faster if it were on GNU/Linux security instead of Windows.
\linej
\linej
Using an incremental methodology proved to be a right choice because there were many desired features that were left behind because lack of time.
It also fits very well with the work flow in this kind project, with very simple development.
The essential requirements and increments were satisfied and a bit more depth than initially planned was applied in many cases, resulting in what we consider a higher quality for the product.

\section{Additions}
Multiple ways to detect and act against malware have been analyzed in this project, but this is just the tip of the iceberg and much work remains to be done.
There are many features and research that could improve the current state of Wazuh as a reliable cybersecurity tool:
\begin{itemize}
	\item All the ideas left behind in this project due to lack of time. They are explained in the exclusions part of the scope section \ref{exclusions}.
	\item Ability to set real variables in the rules, instead of the current ones that are actually constans because they need a restart of the service to be updated.
Unfortunately this can also result in a loss of performance for processing the rules.
	\item A functional database system for all kinds of data, instead of the current CDB that only is useful for IP addresses. This would solve the problem of just monitoring a value, like the case of the free space of the backup volume \ref{free_space_volume}.
	\item Having more decoders for common monitoring commands in Windows, instead of having to write your own or parse the output of the command with an script.
	\item Executing remote commands with direct feedback to the manager, instead of having to use workarounds like writing to a monitored log.
	\item Active response directly with PowerShell, instead of having to use CMD or SSH. This is in development.
	\item Active response with dynamic arguments from the alert, instead of processing the alert manually with a local script and using another program (in this case SSH) for the remote execution.
	\item Option to have only a frequency of 1 in composite rules, which is not such a rare need, instead of needing workarounds for it.
	\item Integration with local antimalware software. This has been done using APIs and custom scripts with some software (like VirusTotal) but not with others as basic as Windows Defender.
	\item Integration with other IDSs. This can provide detection features based on behaviour or network data that Wazuh can not gather on its own.
	\item Further research and development in any of the parts treated in this project.
\end{itemize}
