\chapter{Requirements}

%Especificación de requisitos: debe indicarse, polo miúdo, a especificación do 
%Sistema, xunto coa información que este debe almacenar e as interfaces con outros 
%Sistemas, sexan hardware ou software, e outros requisitos (rendemento, seguridade, 
%etc).


The requirement specification is a full description of the software the project is to develop.
\linej
PMBOK\cite{pmbok} states that requirements are conditions or capabilities that a product must meet to satisfy the contract.
The requirements expose the needs of the client, which have to be accomplished to finish the project successfully.
In this project the requirements will be fullfilled in multiple stages along the project.
\linej
Note that the client in this case is Tarlogic even if the product is a contribution to an open source project.

\linej
\linej
This specification contains:
\begin{itemize}
	\item \textbf{Use cases}: Functionalities that the software will provide.
	\item \textbf{Requirements}: Depending of their type they can describe features, data, relations, properties or any details necessary to explain the system without ambiguity, in a way it can be easily understood.
\end{itemize}

In this project the functional requirements are not included because they can be considered a redundant version of the use cases, because both describe the same functionalities.
Uses cases were chosen over functional requirements because they were considered to be easier to understand and have greater detail. If this project had the need of a very complex requirement specification it would be interesting to have both, as each could help to understand the other better, but in this project the specification should be quite simple.






\section{Use cases}
A use case is a description of all the ways an end-user wants to ``use'' a system. These ``uses'' are like requests of the system, and use cases describe what that system does in response to such requests. In other words, use cases describe the conversation between a system and its user(s), known as actors. Although the system is usually automated (such as an Order system), use cases also apply to equipment, devices, or business processes.\cite{use_case_definition}


\subsection{Use cases actors}
The actors are entities external to the system that interact with it. They can be other systems, persons or even time.

%\subsection{Use cases description}
\subsection{Use cases list}

\section{Requirements analysis}

\subsection{Non functional requirements}

\subsection{Functional requirements}
As mentioned before these are omited because of the redundancy with use cases.

\subsection{Domain requirements}
