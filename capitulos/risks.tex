\section{Risk management}

\subsection{Risk metrics}

\subsection{Risk types}

\subsection{Risk identification}

\begin{table}[H]
	\caption{Project risks}
	\begin{tabularx}{\textwidth}{|l|X|}
		\hline
		\rowcolor{gray!30}
		Identifier & Name \\ \hline
		%R-000 & The scope specified is too big\\ \hline
		R-000 & Optimist planning, ``best case'' (instead of a realistic ``expected case'')\\ \hline
		R-000 & Bad requirement specification\\ \hline
		R-000 & Design errors\\ \hline

		R-000 & Lack of key information from sources\\ \hline 		%articles, documentation, manuals
		R-000 & Lack of feedback or support from the security consultants of Tarlogic\\ \hline 		%eg: holidays or sickness
		R-000 & The learning curve of some technologies is larger than expected\\ \hline
		R-000 & The unexplained parts of the project take more time than expected\\ \hline

		R-000 & Cannot access source material\\ \hline 		%is down, is too old
		R-000 & Unexpected changes in any of the APIs used in the project\\ \hline

		R-000 & Loss of work\\ \hline 	%use git, online repo, local backups for virtual machines?
		R-000 & Wrong management of the project's configuration\\ \hline 	%wrong baseline, wrong identification of the configuration elements, it takes more time than expected, wrong use of the tools, too much time between commits
		R-000 & A delay in one task leads to cascading delays in the dependent tasks\\ \hline

		R-000 & Unnecesary work\\ \hline 		%
		R-000 & The quality of the product is not enough\\ \hline 	%redo
		R-000 & Sickness or overwork\\ \hline
		R-000 & Performance issues\\ \hline 	%redo rules?, get a better machine?
	\end{tabularx}
\end{table}





\subsection{Risk analysis}


\begin{table}[H]
	\begin{tabularx}{\textwidth}{|l|X|}
		\hline
		\rowcolor{gray!30}
		Identifier & \textbf{R-000} \\ \hline
		Description & Bla bla bla bla bla bla bla bla bla bla bla bla bla Bla bla bla bla bla bla bla bla bla bla bla bla bla Bla bla bla bla bla bla bla bla bla bla bla bla bla \\ \hline
		Probability & Low , Medium , High \\ \hline
		Impact & Low , Medium , High \\ \hline
		Exposition & Low , Medium , High \\ \hline
		Indicator & Bla bla bla bla bla bla bla bla bla bla bla bla bla \\ \hline
	\end{tabularx}
\end{table}

\begin{table}[H]
	\begin{tabularx}{\textwidth}{|l|X|}
		\hline
		\rowcolor{gray!30}
		Identifier & \textbf{R-001} \\ \hline
		Description & \\ \hline
		Probability & \\ \hline
		Impact &  \\ \hline
		Exposition &  \\ \hline
		Indicator & \\ \hline
	\end{tabularx}
\end{table}



\subsection{Risk planning}


\subsection{Risk supervision}
