\section{Risk management}

\subsection{Risk metrics}

\begin{table}[H]
	\centering
	\begin{tabular}{|l|l|}
		\hline
		\rowcolor{gray!30}
		Chances of the risk happening & Probability \\ \hline
		$\geq$80\% & \cellcolor{red!60}High\\ \hline
		Between 30\% and 80\% & \cellcolor{yellow!40}Medium\\ \hline
		$\leq$30\% & \cellcolor{green!60}Low\\ \hline
	\end{tabular}
	\caption{Probability classification of risks}
\end{table}


\begin{table}[H]
	\centering
	\begin{tabular}{|l|l|}
		\hline
		\rowcolor{gray!30}
		Resource in Place / Effort / Cost & Impact \\ \hline
		$\geq$20\% & \cellcolor{red!60}High\\ \hline
		Between 10\% and 20\% & \cellcolor{yellow!40}Medium\\ \hline
		$\leq$10\% & \cellcolor{green!60}Low\\ \hline
	\end{tabular}
	\caption{Impact classification of risks}
\end{table}


\begin{table}[H]
	\centering
	\begin{tabular}{|c|c|c|c|c|}
	\hline
		\multicolumn{2}{|c|}{\multirow{2}{*}{\large\textbf{Exposition}}} & \multicolumn{3}{c|}{Probability}\\
		\multicolumn{2}{|c|}{} & \cellcolor{gray!15}\textbf{High} & \cellcolor{gray!15}\textbf{Medium} & \cellcolor{gray!15}\textbf{Low}\\ \hline %\cline{3-5}
		\multirow{3}{*}{Impact} & \cellcolor{gray!15}\textbf{High} & \cellcolor{red!60}High & \cellcolor{red!60}High & \cellcolor{yellow!40}Medium\\
		& \cellcolor{gray!15}\textbf{Medium} & \cellcolor{red!60}High & \cellcolor{yellow!40}Medium & \cellcolor{green!60}Low\\
		& \cellcolor{gray!15}\textbf{Low} & \cellcolor{yellow!40}Medium & \cellcolor{green!60}Low & \cellcolor{green!60}Low\\ \hline
	\end{tabular}
	\caption{Method of calculation of exposition based on probability and impact}
\end{table}

\subsection{Risk types}

\subsection{Risk identification}


\newcommand{\Runo}{Optimist planning, ``best case'' (instead of a realistic ``expected case'')}
\newcommand{\Rdos}{Bad requirement specification}
\newcommand{\Rtres}{Design errors}
\newcommand{\Rcuatro}{Lack of key information from sources}
\newcommand{\Rcinco}{Lack of feedback or support from the security consultants of Tarlogic}
\newcommand{\Rseis}{The learning curve of some technologies is larger than expected}
\newcommand{\Rsiete}{The unexplained parts of the project take more time than expected}
\newcommand{\Rocho}{Can not access source material}
\newcommand{\Rnueve}{Unexpected changes to any of the software used in the project}
\newcommand{\Rdiez}{Loss of work}
\newcommand{\Ronce}{Wrong management of the project's configuration}
\newcommand{\Rdoce}{A delay in one task leads to cascading delays in the dependent tasks}
\newcommand{\Rtrece}{The student can not find a way to code the detection of a certain occurrence}
\newcommand{\Rcatorce}{The quality of the product is not enough}
\newcommand{\Rquince}{Sickness or overwork}
\newcommand{\Rdieciseis}{Performance issues}
\newcommand{\Rdiecisiete}{Unnecessary work}
\newcommand{\Rdieciocho}{Optional requirements delay the project}
\newcommand{\Rdiecinueve}{Unexpected personal events delay the project}


\begin{table}[H]
	\begin{tabularx}{\textwidth}{|l|X|}
		\hline
		\rowcolor{gray!30}
		Identifier & Name \\ \hline
		%R-00 & The scope specified is too big\\ \hline
		R-01 & \Runo \\ \hline
		R-02 & \Rdos \\ \hline
		R-03 & \Rtres \\ \hline
		R-04 & \Rcuatro \\ \hline
		R-05 & \Rcinco \\ \hline
		R-06 & \Rseis \\ \hline
		R-07 & \Rsiete \\ \hline
		R-08 & \Rocho \\ \hline
		R-09 & \Rnueve \\ \hline
		R-10 & \Rdiez \\ \hline
		R-11 & \Ronce \\ \hline
		R-12 & \Rdoce \\ \hline
		R-13 & \Rtrece \\ \hline
		R-14 & \Rcatorce \\ \hline
		R-15 & \Rquince \\ \hline
		R-16 & \Rdieciseis \\ \hline
		R-17 & \Rdiecisiete \\ \hline
		R-18 & \Rdieciocho \\ \hline
		R-19 & \Rdiecinueve \\ \hline
	\end{tabularx}
	\caption{List of the risks of the project}
\end{table}





\newcolumntype{y}{>{\hsize=.15\hsize}X}
\newcolumntype{z}{>{\hsize=.85\hsize}X}
\subsection{Risk analysis and planning}


\begin{table}[H]
	\begin{tabularx}{\textwidth}{|y|z|}
		\hline
		\rowcolor{gray!30}
		Identifier & \textbf{R-001} \\ \hline
		Name & \Runo \\ \hline
		Description &
			An optimistic planning at the start of the project does not take into account problems or delays, and so it does not allocate time for them.
		\\ \hline
		Negative effects &
			Could mean the failure of the project if the objectives can not be accomplished in the time left. \linej
			Cascading delays, because the work done would not fit the planning.
		\\ \hline
		Probability & Medium\\ \hline
		Impact &  High\\ \hline
		Exposition &  High\\ \hline
		Indicator & There are 3 consecutive delays, after the beginning of the project.\\ \hline
		Prevention: Avoid &
			Allocate a bit more time than initially expected for each task, in case something goes wrong.
		\\ \hline
		Correction: Mitigate &
			Redo the planning. \linej
			Reduce the scope of the project, leaving out initially planned optional increments.
		\\ \hline
	\end{tabularx}
\end{table}

\begin{table}[H]
	\begin{tabularx}{\textwidth}{|y|z|}
		\hline
		\rowcolor{gray!30}
		Identifier & \textbf{R-002} \\ \hline
		Name & \Rdos \\ \hline
		Description &
			The requirements specified at the beginning of the project are not specific enough, are not needed or there are new requirements after the beginning of the project.
		\\ \hline
		Negative effects &
			Possible failure of the project if the objectives can not be accomplished in the time left. \linej
			Wasted time, due to lack of comunication in the requirement specification.
		\\ \hline
		Probability & High\\ \hline
		Impact &  High\\ \hline
		Exposition &  High\\ \hline
		Indicator & There are 3 changes in the requirements specification.\\ \hline
		Prevention: Mitigate &
			Confirm that all the requirements have been identified at the beginning of the project.\linej
			Assure that there is no ambiguity in the requirement specification.
		\\ \hline
		Correction: Mitigate &
			Redo the requirement specification. \linej
			Rework of related requirements and work based on them, including the need to test the results. \linej
			Redo the planning.  \linej
			Reduce the scope of the project.
		\\ \hline
	\end{tabularx}
\end{table}

\begin{table}[H]
	\begin{tabularx}{\textwidth}{|y|z|}
		\hline
		\rowcolor{gray!30}
		Identifier & \textbf{R-003} \\ \hline
		Name & \Rtres \\ \hline
		Description &
			A design is not enough or is incorrect. \linej
			This can be found in later stages, when it is clear that the implementation based on the design would not satisfy the requirements.
		\\ \hline
		Negative effects &
			Having to redesign and maybe redo the work based on the design. \linej
			Minor delays.
		\\ \hline
		Probability & Low\\ \hline
		Impact &  Medium\\ \hline
		Exposition & Low\\ \hline
		Indicator & There are 3 designs that need rework.\\ \hline
		Prevention: Mitigate &
			Use design patterns if needed (this project should have very simple designs, so it is possible that there is no need to use them). \linej
			Make the design as simple and modular as possible.
		\\ \hline
		Correction: Mitigate &
			Redesign and probably change and test the work based on the design.
		\\ \hline
	\end{tabularx}
\end{table}

\begin{table}[H]
	\begin{tabularx}{\textwidth}{|y|z|}
		\hline
		\rowcolor{gray!30}
		Identifier & \textbf{R-004} \\ \hline
		Name & \Rcuatro \\ \hline
		Description &
			Not having key information from articles, documentation or manuals.
		\\ \hline
		Negative effects &
			Minor delays. \linej
			Loss of quality. \linej
			Added difficulty, even if the work is done in time. \linej
			Maybe rework and test the functionality, even completely, to follow the desired procedure.
		\\ \hline
		Probability & Medium\\ \hline
		Impact &  Medium\\ \hline
		Exposition &  Medium\\ \hline
		Indicator & The duration of the study of the attack and the related tools takes 50\% than expected. \\ \hline
		Correction: Mitigate &
			Ask the security consultants of Tarlogic for specific information. \linej
			Possibly the need to rework completely some functionality.
		\\ \hline
	\end{tabularx}
\end{table}

\begin{table}[H]
	\begin{tabularx}{\textwidth}{|y|z|}
		\hline
		\rowcolor{gray!30}
		Identifier & \textbf{R-005} \\ \hline
		Name & \Rcinco \\ \hline
		Description &
			Because I do not know enough of some technical aspects of cibersecurity to solve all the problems in this by myself in time, Tarlogic has promised to help (in a tutoring way) if a problem arises. \linej
			This help could be critical to solve or get around some of the most complex problems, which probably happen to be critical points, needing to be dealt with to continue working on that stage.
		\\ \hline
		Negative effects &
			Cascading delays.
		\\ \hline
		Probability & Medium\\ \hline
		Impact &  Medium\\ \hline
		Exposition &  Medium\\ \hline
		Indicator & A simple technical question takes more than 2 working days to be answered or a complex question takes more than 7 working days.\\ \hline
		Prevention: Mitigate &
			Ask in a clear way and with as many details as possible. \linej
			Ask during work hours, to ensure they are available.
		\\ \hline
		Correction: Mitigate &
			Redo planning and possibly change the scope.
		\\ \hline
	\end{tabularx}
\end{table}

\begin{table}[H]
	\begin{tabularx}{\textwidth}{|y|z|}
		\hline
		\rowcolor{gray!30}
		Identifier & \textbf{R-006} \\ \hline
		Name & \Rseis \\ \hline
		Description &
			This is a critical need because not having enough knowledge can result in an inefficient approach to accomplishing the objectives.
		\\ \hline
		Negative effects &
			Loss of quality. \linej
			The work is more complicated.
		\\ \hline
		Probability & Medium\\ \hline
		Impact &  Medium\\ \hline
		Exposition &  Medium\\ \hline
		Indicator & The duration of the study of the technologies takes 50\% than expected. \\ \hline
		Correction: Mitigate &
			Redo planning and possibly change the scope.  \linej
			Ask the security consultants of Tarlogic for specific help. \linej
			Maybe the need to rework completely some functionality.
		\\ \hline
	\end{tabularx}
\end{table}

\begin{table}[H]
	\begin{tabularx}{\textwidth}{|y|z|}
		\hline
		\rowcolor{gray!30}
		Identifier & \textbf{R-007} \\ \hline
		Name & \Rsiete \\ \hline
		Description &
			There is not enough specification on what a tasks implies or not enough planning. \linej
			This means that a part of the project is not understood as it should, and the work done is not what was expected or is not enough, needing more time to finish.
		\\ \hline
		Negative effects &
			Wasted time that should have been easy to avoid. \linej
			Loss of quality. \linej
			Could mean the failure of the project if the objectives can not be accomplished in the time left.
		\\ \hline
		Probability & Low\\ \hline
		Impact &  High\\ \hline
		Exposition &  Medium\\ \hline
		Indicator & A task takes 25\% more time than expected and when the causes are investigated it is revealed that there were ambiguous descriptions or planning.\\ \hline
		Prevention: Avoid &
			Try to detail every part enough, having no obvious ambiguity.
		\\ \hline
		Correction: Mitigate &
			Possible need to redo the specifications.  \linej
			Redo planning and possibly change the scope.  \linej
			Maybe having to redo related work.
		\\ \hline
	\end{tabularx}
\end{table}

\begin{table}[H]
	\begin{tabularx}{\textwidth}{|y|z|}
		\hline
		\rowcolor{gray!30}
		Identifier & \textbf{R-008} \\ \hline
		Name & \Rocho \\ \hline
		Description &
			All or part of the source material can not be accessed, probably because the only host of the resource is down.
		\\ \hline
		Negative effects &
			In some cases this could mean a delay in a critical task, delaying the whole project for an unknown period of time.
		\\ \hline
		Probability & Low\\ \hline
		Impact & Medium\\ \hline
		Exposition & Low\\ \hline
		Indicator & There have been at least 10 failed attempts to download the source material, at least 5 with a computer A in a network X and at least 5 with a computer B in a network Y.\\ \hline
		Prevention: Avoid &
			When possible choose the source with the best uptime.
		\\ \hline
		Correction: Mitigate &
			Redo planning and possibly change the scope. \linej
			Possible need to cut out the part of the project that depends on this source. \linej
			Maybe find another source or wait to the original source to be accessible again.
		\\ \hline
	\end{tabularx}
\end{table}

\begin{table}[H]
	\begin{tabularx}{\textwidth}{|y|z|}
		\hline
		\rowcolor{gray!30}
		Identifier & \textbf{R-009} \\ \hline
		Name & \Rnueve \\ \hline
		Description &
			Changes to base software could affect this project directly or indirectly: programs could fail or not work as expected. \linej
			This could mean any software changes, from simple syntax to API changes. \linej
			Is possible that these changes would eliminate the need of planned or already done work. \linej
			In a project that does not work in a bleeding edge environment, like this, this occurrence should be very rare and even if it were to happen it would have to interfere with the part of the software this project uses, which (as this is not bleeding edge) normally would be backwards compatible.
		\\ \hline
		Negative effects &
			Minor delays. \linej
			Unnecessary work.
		\\ \hline
		Probability & Low\\ \hline
		Impact & Low \\ \hline
		Exposition &  Low\\ \hline
		Indicator & The software is not working as expected due to a change in another software version.\\ \hline
		Prevention: Mitigate &
			When possible use software that follow good design guidelines and try to be backwards compatible. \linej
			Be informed about the roadmap and future functionalities of these software projects.
		\\ \hline
		Correction: Mitigate &
			Need to adapt the software to work as expected or remove the related functionalities.
		\\ \hline
	\end{tabularx}
\end{table}

\begin{table}[H]
	\begin{tabularx}{\textwidth}{|y|z|}
		\hline
		\rowcolor{gray!30}
		Identifier & \textbf{R-010} \\ \hline
		Name & \Rdiez \\ \hline
		Description &
			Due to a bad configuration management or something else, there is a loss of work related to this project.
		\\ \hline
		Negative effects &
			Need to do again the work already done but lost.\linej
			Depending of the time needed to recover the work, there could be minor or very big delays, planning, changes to the scope of the project and even its failure.
		\\ \hline
		Probability & Low\\ \hline
		Impact &  High\\ \hline
		Exposition &  Medium\\ \hline
		Indicator & The need to replicate already done work is greater than 30 minutes.\\ \hline
		Prevention: Mitigate &
			Take snapshots of key status for each virtual machine. \linej
			Automate backing up the data and store the copies both in a cloud storage service and in a local disk.
		\\ \hline
		Correction: Mitigate &
			Recover the last backup available of the work. \linej
			If needed work even outside schedule and in holidays.
		\\ \hline
	\end{tabularx}
\end{table}

\begin{table}[H]
	\begin{tabularx}{\textwidth}{|y|z|}
		\hline
		\rowcolor{gray!30}
		Identifier & \textbf{R-011} \\ \hline
		Name & \Ronce \\ \hline
		Description &
			The project's configuration is inefficient or lacks work. \linej
			For example due to unclear changes or taking too long to commit changes.
		\\ \hline
		Negative effects &
			Maybe the failure of the project if the objectives can not be accomplished in the time left. \linej
			Possibly wrong baselines or identification of the configuration elements. \linej
			It could be that it takes more time than expected to manage the project. \linej
			The project suffer delays because the need to redo management work and/or planned tasks.
		\\ \hline
		Probability & Medium\\ \hline
		Impact &  High\\ \hline
		Exposition &  High\\ \hline
		Indicator & There are 3 delays because of the configuration of the project.\\ \hline
		Prevention: Avoid &
			The configuration of the project should be just complex enough (whithout ambiguity, to ensure a proper management), but not too much complex (which would be hard to follow). \linej
			Use of familiar and standard tools, like Git. \linej
			Optionally use an easier to manage lifecycle. \linej
			Study of the configuration management done in previous final degree projects, to get a proper idea of its scope and details.
		\\ \hline
	\end{tabularx}
\end{table}

\begin{table}[H]
	\begin{tabularx}{\textwidth}{|y|z|}
		\hline
		\rowcolor{gray!30}
		Identifier & \textbf{R-012} \\ \hline
		Name & \Rdoce \\ \hline
		Description &
			A task gets delayed and one or more tasks depends on its completion to start, so they get delayed too.
		\\ \hline
		Negative effects &
			Cascading delays.
		\\ \hline
		Probability & Medium\\ \hline
		Impact &  Medium\\ \hline
		Exposition &  Medium\\ \hline
		Indicator & At least 2 tasks are delayed, due to only one of them needing more time.\\ \hline
		Prevention: Avoid &
			When planning, avoid task dependencies whenever possible. \linej
			Optionally use a lifecycle based on increments.
		\\ \hline
		Correction: Mitigate &
			Redo planning and possibly change the scope.
		\\ \hline
	\end{tabularx}
\end{table}

\begin{table}[H]
	\begin{tabularx}{\textwidth}{|y|z|}
		\hline
		\rowcolor{gray!30}
		Identifier & \textbf{R-013} \\ \hline
		Name & \Rtrece \\ \hline
		Description &
			It could be that the knowledge of the student is too limited or the problem has too much logical or mathematical difficulty.\linej
			Another possibility is that the event is impossible to detect with the current technologies. If so, this impossibility could be hard to assure too, due to the complexity of nowadays technology.
		\\ \hline
		Negative effects &
			High difficulty to estimate the time needed to detect the event. \linej
			Cascading delays.
		\\ \hline
		Probability & Low\\ \hline
		Impact &  Low\\ \hline
		Exposition &  Low\\ \hline
		Indicator & Finding a way to detect the occurrence takes 30\% more time than planned.\\ \hline
		% the code is done later!! first we THINK/DESIGN of a way to detect it
		Prevention: Mitigate &
			Have as much information on the problem as possible, the more detailed the better.
		\\ \hline
		Correction: Mitigate &
			Ask the security consultants of Tarlogic for help. \linej
			Demonstrate that it is possible to detect it.
		\\ \hline
	\end{tabularx}
\end{table}

\begin{table}[H]
	\begin{tabularx}{\textwidth}{|y|z|}
		\hline
		\rowcolor{gray!30}
		Identifier & \textbf{R-014} \\ \hline
		Name & \Rcatorce \\ \hline
		Description &
			The final result is does not comply the quality standard set for this project.
		\\ \hline
		Negative effects &
			The incorporation to the official repository gets rejected.\linej
			Redo planning and possibly change the scope.  \linej
			Analysis of the changes needed to improve the quality.
		\\ \hline
		Probability & Low\\ \hline
		Impact &  High\\ \hline
		Exposition &  Medium\\ \hline
		Indicator & Getting 10 suggestions to rework functionality.\\ \hline
		Prevention: Avoid &
			Follow design patterns. \linej
			Follow the design guidelines of the official repository when possible.
		\\ \hline
		Correction: Mitigate &
			Need to redo and test work. \linej
			Pass some kind of quality control.
		\\ \hline
	\end{tabularx}
\end{table}

\begin{table}[H]
	\begin{tabularx}{\textwidth}{|y|z|}
		\hline
		\rowcolor{gray!30}
		Identifier & \textbf{R-015} \\ \hline
		Name & \Rquince \\ \hline
		Description &
			The health of the student deteriorates to the point it affects the project.
		\\ \hline
		Negative effects &
			Probably the quality of the project drops. \linej
			Possibly delays, that could be hard to specify their limit. \linej
			Analysis of the changes needed to improve the quality. \linej
			In the worst case scenario the project can not continue and fails.
		\\ \hline
		Probability & Medium\\ \hline
		Impact &  High\\ \hline
		Exposition &  High\\ \hline
		Indicator & There is an unexpected delay because the functionality is not done but there has not been any important issues that could explain it but there is a clear deterioration of the student health. \\ \hline
		Prevention: Avoid &
			Stay healthy by following a regular schedule for work and exercising, that includes multiple rest periods. \linej
			Optionally maintain a diet.
		\\ \hline
		Correction: Mitigate &
			Go to the doctor and follow any instructions to improve the recovery.
		\\ \hline
	\end{tabularx}
\end{table}

\begin{table}[H]
	\begin{tabularx}{\textwidth}{|y|z|}
		\hline
		\rowcolor{gray!30}
		Identifier & \textbf{R-016} \\ \hline
		Name & \Rdieciseis \\ \hline
		Description &
			The program is too heavy for the environment and takes too much resources, because there are not good enough optimizations or the problems are poorly approached.
		\\ \hline
		Negative effects &
			Minor delays. \linej
		\\ \hline
		Probability & Low\\ \hline
		Impact &  Low\\ \hline
		Exposition &  Low\\ \hline
		Indicator & The program takes 30\% more resources that at the beginning of the project.\\ \hline
		Prevention: Mitigate &
			If possible use efficient algorithms and check the efficiency after the testing is done for each increment.
		\\ \hline
		Correction: Mitigate &
			Analysis of faster ways to solve the problem.\linej
			Code and test a faster solution.
		\\ \hline
	\end{tabularx}
\end{table}

\begin{table}[H]
	\begin{tabularx}{\textwidth}{|y|z|}
		\hline
		\rowcolor{gray!30}
		Identifier & \textbf{R-017} \\ \hline
		Name & \Rdiecisiete \\ \hline
		Description &
			Resources are wasted in work that latter is not used. \linej
			This could happen because multiple reasons, like wrong assumptions or balancing of the remaining time of the project.
		\\ \hline
		Negative effects &
			Minor delays.
		\\ \hline
		Probability & Low\\ \hline
		Impact &  Low\\ \hline
		Exposition &  Low\\ \hline
		Indicator & There is at least one functionality not necessary or useful for any requirement.\\ \hline
		Prevention: Avoid &
			In the design stage make sure that everything is really needed.
		\\ \hline
		Correction: Mitigate &
			Evaluate again if the work planned is really needed.
		\\ \hline
	\end{tabularx}
\end{table}

\begin{table}[H]
	\begin{tabularx}{\textwidth}{|y|z|}
		\hline
		\rowcolor{gray!30}
		Identifier & \textbf{R-018} \\ \hline
		Name & \Rdieciocho \\ \hline
		Description &
			Optional requirements get too much time or are treated as vital.
		\\ \hline
		Negative effects &
			The task related to these requirements get too much resources.\linej
			Vital requirements get less resources, making the project loss value.
		\\ \hline
		Probability & Low\\ \hline
		Impact &  Low\\ \hline
		Exposition &  Low\\ \hline
		Indicator & There is at least one functionality from an optional requirement, when the project is behind schedule and there are vital requirements not yet accomplished.\\ \hline
		Prevention: Avoid &
			The optional requirements are planned as optional: they are only done if there is enough time left.
		\\ \hline
		Correction: Mitigate &
			Redo the planning.
		\\ \hline
	\end{tabularx}
\end{table}

\begin{table}[H]
	\begin{tabularx}{\textwidth}{|y|z|}
		\hline
		\rowcolor{gray!30}
		Identifier & \textbf{R-019} \\ \hline
		Name & \Rdiecinueve \\ \hline
		Description &
			There are unplanned occurrences that need considerable time from the student, for example family matters.
		\\ \hline
		Negative effects &
			Time loss, resulting in a quality drop and possibly in a smaller scope.\linej
			It can be hard to specify when the event will end, resulting in uncertainty and the failure of the project in the worst case scenario. Even more if is about a chronical or serious sickness from a family member. \linej
			Vital requirements get less resources, making the project loss value.
		\\ \hline
		Probability & Medium\\ \hline
		Impact & High\\ \hline
		Exposition & High\\ \hline
		Indicator & The student stops to work on the project for more than 2 planned weeks, to attend personal matters.\\ \hline
		Prevention: Avoid &
			Always be organized and try to predict time consuming events.
		\\ \hline
		Correction: Mitigate &
			Redo the planning. \linej
			Use personal time like holidays and weekends to work on the project to compensate.
			In extreme cases the project may be put on hold or even fail.
		\\ \hline
	\end{tabularx}
\end{table}







\subsection{Risk supervision}


%\begin{table}[H]
%	\begin{tabularx}{\textwidth}{|l|X|}
%		\hline
%		\rowcolor{gray!30}
%		Identifier & \textbf{R-001} \\ \hline
%		Name & \Runo \\ \hline
%		Date of the beginning of the problem & TODO \\ \hline
%		Date of the solution of the problem & TODO \\ \hline
%		Actions & TODO \\ \hline
%		New probability & TODO \\ \hline
%		New impact &  TODO \\ \hline
%		New exposition &  TODO \\ \hline
%	\end{tabularx}
%\end{table}

\begin{table}[H]
	\begin{tabularx}{\textwidth}{|l|X|}
		\hline
		\rowcolor{gray!30}
		Identifier & \textbf{R-019} \\ \hline
		Name & \Rdiecinueve \\ \hline
		Date of the beginning of the problem & 05/12/2018 \\ \hline
		Date of the solution of the problem & 10/02/2019 \\ \hline
		Actions & After a delay of 2 weeks it was clear that the student could not meet the original planning, or at least without rushing and suffering significant quality loss. \linej
			The project was put on hold and the student notified the tutors, who agreed to the next deadline. \linej
			The student kept working on the project (researching) from time to time. \\ \hline
		New probability & Low (before was Medium) \\ \hline
		New impact &  High (same as before) \\ \hline
		New exposition & Medium (before was High) \\ \hline
		New prevention: Avoid &
			Another person takes responsibility of the family member, freeing the student.
		\\ \hline
	\end{tabularx}
\end{table}
