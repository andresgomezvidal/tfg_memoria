\section{Scope management}

We had to leave behind some ideas because the kind of project this is.

\linej
\linej
Another interesting way to to take advantage from IDS is to set up a honeypot (a false server just to be compromised) and learn from the intrusions suffered, improving the defenses (firewall, IPS and IDS) for the real servers. There are some honeypot implementations that automate (for example with machine learning) the generation of rules for certain IDSs, but is not yet a trend because there are problems:
\begin{itemize}
	\item Experienced attackers have learned to avoid honeypots, because they are easy to identify due to the low security they have.
	\item Is not trivial to automate correctly the defense based on the information of the system, because its state can be very complex (for example due to more than one attack at the same time).
\end{itemize}
\linej
This automation would be a great solution to the need to update manually the rules and depending of the case could even protect against day-zero vulnerabilities. Despite being interesting this was not even included as a possible increment because the complexity of the task and if not it would probably not ended well because it is something that even experts in cybersecurity have some trouble with at the moment.



\subsection{Metodology}





