\section{Configuration management}
The objective of the management of the configuration is to control the changes on the configuration elements, for the duration of the project. This assures the work is always archived, resulting in multiple control advantages.

\linej
\linej
A Git repository\cite{memoria_github} hosted on Github was used to manage the documentation of the project, which revolves mostly around this document.
Another private Git repository was used to keep track of notes, uncertain elements and development in process.
\linej
Git commits are used to keep track of the changes made in the documentation.
Issues, milestones, releases and other features were not used, but they could be if any need appeared.
There is only the master branch, because this project is not about software development and there is only one contributor.

\subsection{Configuration elements}
They are the items that need to be monitored in order to guarantee the consistency of the project.
They can be grouped into:
\begin{itemize}
	\item \textbf{Code}: The scripts and ruleset elements created in the project.
	\item \textbf{Diagrams and images}: They help to explain some of the key concepts of the project.
	\item \textbf{Memory of the project}: This very document. It is mandatory because it explains the whole project.
		%Without this document there would be no way to understand it.
		This item actually makes use of the other two, to fully document the project.
\end{itemize}
\linej
This does not include the virtual machines nor its snapshots because by themselves they are not configuration material, but a result of applying certain configuration over generic images of operative systems. This configuration is already documented by other sources, cited in this document. Still local backups were made because they are still important archives to the project.
