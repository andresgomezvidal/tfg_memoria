\chapter{Planning}

%Planificación e presupostos: debe incluír a estimación do costo (presuposto) e dos 
%recursos necesarios para efectuar a implantación do Traballo, xunto coa planificación 
%temporal do mesmo e a división en fases e tarefas. Recoméndase diferenciar os costos relativos a persoal dos relativos a outros gastos como instalacións e equipos.








%WBS==EDT (translation!)
\section{Initial WBS}
\newpage
{\footnotesize
\begin{forest} for tree={
    grow=east,
    growth parent anchor=east,
    parent anchor=east,
    child anchor=west,
    edge path={\noexpand\path[\forestoption{edge},->, >={latex}] 
         (!u.parent anchor) -- +(5pt,0pt) |- (.child anchor)
         \forestoption{edge label};}
}
[Improvements in IDS: adding functionality to Wazuh, root
    [Closing of the project, onode
        [Project documentation, tnode]
        [Pull request to the official ruleset repository, tnode]
    ]
    [Increment 7: VirusTotal integration, onode
        [Improved integration with antivirus and website scanners, tnode]
        %[Study of the present status, tnode]
    ]
    [Increment 6: Additional detection for GNU/Linux, onode
        [Rules and decoders, tnode]
        %[Analysis and writing of rules/decoders, tnode]
        %[Study of the present status, tnode]
    ]
    [Increment 5: Explore solutions in problems with GPDR, onode
        [Rules and decoders, tnode]
        %[Analysis and writing of rules/decoders, tnode]
        %[Study of GPDR issues, tnode]
    ]
    [Increment 4: Adapt Wazuh configuration to typical requirements from enterprises, onode
        [Configuration changes, tnode]
        %[Analysis of requirements, tnode]
    ]
    [Increment 3: Detection/action against ransomware, onode
        [Rules and decoders, tnode]
        %[Analysis and writing of rules/decoders and actions, tnode]
        %[Study of ransomware attack patterns, tnode]
    ]
    [Increment 2: Use of data from Sysmon, onode
        [Rules and decoders, tnode]
        %[Analysis and writing of rules/decoders, tnode]
        %[Study of Sysmon tools, tnode]
    ]
    [Increment 1: Common attacks in Windows Server, onode
        [Rules and decoders, tnode]
        %[Detailed study of each attack and patterns, tnode]
    ]
    [Beginning of the project, onode
        [Setup of the work environment, tnode]
        [Study of Wazuh documentation and related tools and technologies, tnode]
    ]
    [Project management, onode
        [Cost management, tnode]
        [Configuration management, tnode]
        [Time management, tnode]
        [Risk management, tnode]
        [Requirement management, tnode]
        [Scope management, tnode]
    ]
]
\end{forest}
}


\section{Initial planning}


%TODO
%holidays in gantt
%marked in the calendar or adding more days manually
	%https://tex.stackexchange.com/questions/106419/how-to-colour-the-pgfgantt-canvas-based-on-calendar-dates
	%https://tex.stackexchange.com/questions/163482/pgfgantt-customization-of-canvas-to-mark-vacations/163521






%independent duration in days
%\newcommand{\Duno}{21} %draft
\newcommand{\Duno}{0}
\newcommand{\Ddos}{7}
\newcommand{\Dtres}{28}
\newcommand{\Dcuatro}{7}
\newcommand{\Dcinco}{21}
\newcommand{\Dseis}{7}
\newcommand{\Dsiete}{21}
\newcommand{\Docho}{14}
\newcommand{\Dnueve}{14}
\newcommand{\Ddiez}{14}

%dependent duration (ending time) in weeks
\newcommand{\Funo}{\the\numexpr \Duno /7 \relax}
\newcommand{\Fdos}{\the\numexpr \Funo + \Ddos/7 \relax}
\newcommand{\Ftres}{\the\numexpr \Fdos + \Dtres/7 \relax}
\newcommand{\Fcuatro}{\the\numexpr \Ftres + \Dcuatro/7 \relax}
\newcommand{\Fcinco}{\the\numexpr \Fcuatro + \Dcinco/7 \relax}
\newcommand{\Fseis}{\the\numexpr \Fcinco + \Dseis/7 \relax}
\newcommand{\Fsiete}{\the\numexpr \Fseis + \Dsiete/7 \relax}
\newcommand{\Focho}{\the\numexpr \Fsiete + \Docho/7 \relax}
\newcommand{\Fnueve}{\the\numexpr \Focho + \Dnueve/7 \relax}
\newcommand{\Fdiez}{\the\numexpr \Fnueve + \Ddiez/7 \relax}

%dependent duration (ending time) in days
\newcommand{\Frequirement}{2}

\newcommand{\Plen}{19} %duration of the project in weeks

\newcommand{\Cimportant}{red!70}
\newcommand{\Coptional}{cyan!30}


The next Gantt diagram shows the initial planning, from the draft proposal (31/10/2018) to the end of the project (TODO/02/2019).
\linej
The tasks marked in \colorbox{\Cimportant}{red} are essential to the project, meanwhile the ones marked in \colorbox{\Coptional}{cyan} are considered optional and only will be done if there is enough time left.
\linej
Furthemore the last two weeks are marked with a grey overlay to mark that there are only about 17 weeks before the due date of this project (in February). This difference is because the estimation of the tasks was made by the student and so it is not reliable, which means that it could be optimistic or pessimist. Thus the need to either reduce tasks or have more that there were expected to fit.


\begin{figure}[h!bt]
	\begin{center}
	\begin{ganttchart}[
		vgrid
	]{1}{\Plen}
		\newganttlinktypealias{straight}{f-s}
		\setganttlinklabel{straight}{}
		\gantttitle{Planning of the project in weeks}{\Plen} \\
		\gantttitlelist{1,...,\Plen}{1} \\


	\ganttbar[name=fuera-de-plazo-top,bar/.style={fill=none, draw=none}]{}{18}{\Plen} 			% top node


		\ganttgroup{Project management}{1}{\Plen} \\
		%\ganttbar[name=Draft]{Draft proposal}{1}{\Funo} \\
		\ganttbar[name=Beginning,bar/.append style={fill=\Cimportant}]{Beginning of the project}{\the\numexpr \Funo + 1 \relax}{\Fdos} \\
		\ganttbar[name=Increment1,bar/.append style={fill=\Cimportant}]{Increment 1}{\the\numexpr \Fdos + 1 \relax}{\Ftres} \\
		\ganttbar[name=Increment2,bar/.append style={fill=\Cimportant}]{Increment 2}{\the\numexpr \Ftres + 1 \relax}{\Fcuatro} \\
		\ganttbar[name=Increment3,bar/.append style={fill=\Cimportant}]{Increment 3}{\the\numexpr \Fcuatro + 1 \relax}{\Fcinco} \\
		\ganttbar[name=Increment4,bar/.append style={fill=\Coptional}]{Increment 4}{\the\numexpr \Fcinco + 1 \relax}{\Fseis} \\
		\ganttbar[name=Increment5,bar/.append style={fill=\Coptional}]{Increment 5}{\the\numexpr \Fseis + 1 \relax}{\Fsiete} \\
		\ganttbar[name=Increment6,bar/.append style={fill=\Coptional}]{Increment 6}{\the\numexpr \Fsiete + 1 \relax}{\Focho} \\
		\ganttbar[name=Increment7,bar/.append style={fill=\Coptional}]{Increment 7}{\the\numexpr \Focho + 1 \relax}{\Fnueve} \\
		\ganttbar[name=Closing,bar/.append style={fill=\Cimportant}]{Closing of the project}{\the\numexpr \Fnueve + 1 \relax}{\Fdiez} \\

		%\ganttlink[link type=straight]{Draft}{Beginning}
		\ganttlink[link type=straight]{Beginning}{Increment1}
		\ganttlink[link type=straight]{Increment1}{Increment2}
		\ganttlink[link type=straight]{Increment2}{Increment3}
		\ganttlink[link type=straight]{Increment3}{Increment4}
		\ganttlink[link type=straight]{Increment4}{Increment5}
		\ganttlink[link type=straight]{Increment5}{Increment6}
		\ganttlink[link type=straight]{Increment6}{Increment7}
		\ganttlink[link type=straight]{Increment7}{Closing}


	\ganttbar[name=fuera-de-plazo-bottom,bar/.style={fill=none, draw=none}]{}{18}{\Plen} 			% bottom node
	\begin{scope}
	\draw [opacity=0.2,line width=28] (fuera-de-plazo-top) -- ($(fuera-de-plazo-bottom)+(0,-15pt)$);
	\end{scope}

	\end{ganttchart}
	\end{center}
	\caption{Initial planning}
\end{figure}




%makes no sense by days
%\begin{figure}[h!bt]
%	\begin{center}
%	\begin{ganttchart}[
%	hgrid style/.style={draw=black, line width=.3pt},
%	vgrid={*1{draw=black!5, line width=.75pt}},
%	link/.style={-latex},
%	]{1}{\Duno}
%		\newganttlinktypealias{straight}{f-s}
%		\setganttlinklabel{straight}{}
%		\gantttitle{Planning in days}{\Duno} \\
%		\gantttitlelist{1,...,\Duno}{1} \\
%
%		\ganttgroup[name=Management]{Project management}{1}{\Duno} \\
%
%		\ganttbar[name=Scope]{Scope management}{1}{1}
%		\ganttbar[name=Requirement]{Requirement management}{1}{\Frequirement}
%        \ganttbar[name=Risk]{Risk management}{\Frequirement}{15}
%        \ganttbar[name=Time]{Time management}{\Frequirement}{\Duno}
%        \ganttbar[name=Configuration]{Configuration management}{\Frequirement}{10}
%        \ganttbar[name=Cost]{Cost management}{\Frequirement}{}
%
%		\ganttlink[link type=straight]{Scope}{Requirement}
%		\ganttlink[link type=straight]{Requirement}{Risk}
%		\ganttlink[link type=straight]{Requirement}{Time}
%		\ganttlink[link type=straight]{Requirement}{Configuration}
%		\ganttlink[link type=straight]{Requirement}{Cost}
%
%	\end{ganttchart}
%	\end{center}
%	\caption{Detailed planning of ``Project management''}
%\end{figure}






\section{Final planning}
