%
%  PARA TRABALLOS EN GALLEGO USAR (LINEA 12): \usepackage[galician]{babel}
%  PARA TRABALLOS EN CASTELLANO USAR (LINEA 13): \usepackage[spanish]{babel}
%
% Para los acentos usamos codificacion UTF-8 (LINEA 10): \usepackage[utf8]{inputenc} 
% Si se usase la codificacion es_ES.ISO-8859-1 (LINEA 11): \usepackage[latin1]{inputenc}
% La conversion de acentos se hace con: iconv -f UTF-8 -t ISO-8859-1 filename.tex
%
% Como se incluyen figuras eps hay que compilar con: latex traballo , dvipdf traballo
%

\documentclass[12pt,twoside,a4paper]{book}
%\documentclass[a4paper,10pt]{article}

% pódense engadir todos os packages necesarios
\usepackage[utf8]{inputenc}
%\usepackage[T1]{fontenc}	%Correct accented characters and copy paste them, also not unexpected results with some characters like pipe


%				%es-noquotes]{babel}			si caracteres dan problemas
%				%es-noshorthands]{babel}		si caracteres dan todavia problemas
%				%es-noindentfirst]{babel}		para eliminar sangría
%\usepackage[latin1]{inputenc}
%\usepackage[galician]{babel}
%\usepackage[spanish]{babel}
%\usepackage[spanish, es-tabla,es-noindentfirst]{babel}		%http://www.tex-tipografia.com/spanishopt.html
%\usepackage[spanish, es-tabla]{babel}		%http://www.tex-tipografia.com/spanishopt.html
\usepackage[english]{babel}		%http://www.tex-tipografia.com/spanishopt.html
\addto\captionsenglish{
  \renewcommand{\contentsname}
    {Index} % ToC will show "Index" instead of "Content"
}

%\usepackage[document]{ragged2e}	%centering

\usepackage{enumitem}

\usepackage{graphicx}
\usepackage[dvips]{epsfig}
\usepackage{amssymb}
\usepackage[official]{eurosym}
\usepackage{float}
\usepackage{latexsym}
\usepackage{a4}
\usepackage{hyperref}

%tables
\usepackage{multirow}
\usepackage{multicol}
%\usepackage{changepage}		%margin, eg for tables before tabular \begin{adjustwidth}{-1.4cm}{}
\usepackage{tabularx}

%\usepackage{color}
\usepackage[table,xcdraw,usenames,dvipsnames]{xcolor}


%WBS
\usepackage{forest}
\usetikzlibrary{arrows.meta,shapes,positioning,shadows,trees}
\tikzset{
    basic/.style  = {draw, text width=2cm, drop shadow, font=\sffamily, rectangle},
    root/.style   = {basic, rounded corners=2pt, thin, align=center,
                     fill=green!30, text width=2cm,},
    onode/.style = {basic, thin, rounded corners=2pt, align=left, fill=green!60,text width=6cm,},
    tnode/.style = {basic, thin, align=left, fill=gray!30, text width=5cm},
    edge from parent/.style={draw=black, edge from parent fork right}
}



\usepackage{rotating}
\usepackage{pgfgantt}


\usepackage{amsmath}		%https://www.sharelatex.com/learn/Aligning_equations_with_amsmath
\usepackage{amsfonts}		%Extended set of fonts for use in mathematics

%%\usepackage{afterpage}		%Commands specified in its argument are expanded after the current page is output. EG: \afterpage{\clearpage} and the current page will be filled up with text as usual, but then the \clearpage command will flush out all the floats before the next text page begins

\usepackage{listings}
	\lstset{breaklines}			%line wrap for listings
	\newcommand{\lstlistinginput}{\lstinputlisting}		%alias


\usepackage{courier}		%listings better font
\usepackage[sorting=none]{biblatex} %use with \cite{} and \printbibliography[type=online,heading=subbibliography,title={Referencias web}]
	\addbibresource{config/bib.bib}



%%\usepackage{microtype}		%Improves spacing (words and letters) and more stuff. Load after fonts, because is dependent on this font
%%\usepackage{siunitx}		%SI units
%%\usepackage{xspace}			%Decides whether to insert a space to replace one "eaten" by the command decoder
%%\usepackage{todonotes}		%Mark things to do later
%\usepackage{hyperref}		%\hyperref, \url and \href
%%\usepackage[a4paper]{geometry}	%Margins without needing to remember the particular page dimensions commands. Eg [a4paper], [top=1in, bottom=1.25in, left=1.25in, right=1.25in]
%%\usepackage{cleveref}		%LAST \usepackage in the preamble. If anything else modifies the referencing system (like amsmath) it all goes wrong
%
%%\DisableLigatures{encoding = *, family = *}		%https://en.wikibooks.org/wiki/LaTeX/Text_Formatting#Ligatures
%%\def\spanishoperators{}		%stuff like sin traduced to sen, max to máx, etc




\newcommand\realnumberstyle[1]{}

\makeatletter
\newcommand{\zebra}[3]{%
    {\realnumberstyle{#3}}%
    \begingroup
    \lst@basicstyle
    \ifodd\value{lstnumber}%
        \color{#1}%
    \else
        \color{#2}%
    \fi
        \rlap{\hspace*{\lst@numbersep}%
        \color@block{\linewidth}{\ht\strutbox}{\dp\strutbox}%
        }%
    \endgroup
}
\makeatother

%%%%%%%%%%%%%%%%%%%%%%%%%		MORE	ALIASES		%%%%%%%%%%%%%%%%%%%%%%%%%

\newcommand{\lineh}{\rule{\textwidth}{1pt}\hfill\break}
\newcommand{\linej}{\hfill\break}

\newcommand{\projecthours}{417}

\newcommand{\IncrementoSiete}{VirusTotal integration}
\newcommand{\IncrementoSeis}{Additional detection for GNU/Linux}
\newcommand{\IncrementoCinco}{Explore solutions in problems with GPDR}
\newcommand{\IncrementoCuatro}{Adapt Wazuh configuration to typical requirements from enterprises}
\newcommand{\IncrementoTres}{Detection/action against ransomware}
\newcommand{\IncrementoDos}{Use of more data sources}
\newcommand{\IncrementoUno}{Common attacks in Windows Server}


\newcommand{\RNO}{\cellcolor{red!60}No}
\newcommand{\RYES}{\cellcolor{green!60}Yes}

%%%%%%%%%%%%%%%%%%%%%%%%%		END BASIC CONF		%%%%%%%%%%%%%%%%%%%%%%%%%



%\definecolor{mygreen}{rgb}{0,0.6,0}
%\definecolor{mygray}{rgb}{0.5,0.5,0.5}


%\renewcommand{\thesubsection}{\thesection.\alph{subsection}}		%change subsection style to letters

%https://en.wikibooks.org/wiki/LaTeX/Source_Code_Listings


\input{config/latex-listings-powershell}
\lstdefinestyle{PS}{
	language=Powershell,		%for syntax highlighting
	basicstyle=\footnotesize\ttfamily,
	%numbers=left,
	numbers=none,
	numberstyle=\tiny\zebra{gray!10}{white},
	frame=tb,
	postbreak=\mbox{\textcolor{red}{$\hookrightarrow$}\space},
	tabsize=2,
	columns=fixed,
	showstringspaces=false,
	showtabs=false,
	keepspaces,
	commentstyle=\color{red},
	stringstyle=\color{gray},		%needs xcolor
	keywordstyle=\color{blue}
}

\lstdefinestyle{sh} {
	language=sh,		%for syntax highlighting
	basicstyle=\footnotesize\ttfamily,
	numbers=none,
	frame=tb,
	postbreak=\mbox{\textcolor{red}{$\hookrightarrow$}\space},
	tabsize=2,
	columns=fixed,
	showstringspaces=false,
	showtabs=false,
	keepspaces,
	commentstyle=\color{red},
	stringstyle=\color{gray},		%needs xcolor
	keywordstyle=\color{blue},
	%directivestyle=\color{gray},	%shebang color
	emph={local, export },
	emphstyle=\color{green},
}

\lstdefinestyle{ruby} {
	language=Ruby,		%for syntax highlighting
	basicstyle=\footnotesize\ttfamily,
	%numbers=left,
	numbers=none,
	numberstyle=\tiny\zebra{gray!10}{white},
	frame=tb,
	postbreak=\mbox{\textcolor{red}{$\hookrightarrow$}\space},
	tabsize=4,
	columns=fixed,
	showstringspaces=false,
	showtabs=false,
	keepspaces,
	commentstyle=\color{red},
	stringstyle=\color{gray},		%needs xcolor
	keywordstyle=\color{blue}
}

\lstdefinestyle{xml} {
	language=XML,		%for syntax highlighting
	basicstyle=\footnotesize\ttfamily,
	%numbers=left,
	numbers=none,
	numberstyle=\tiny\zebra{gray!10}{white},
	frame=tb,
	postbreak=\mbox{\textcolor{red}{$\hookrightarrow$}\space},
	tabsize=4,
	columns=fixed,
	showstringspaces=false,
	showtabs=false,
	keepspaces,
	commentstyle=\color{red},
	stringstyle=\color{gray},		%needs xcolor
	keywordstyle=\color{blue}
}

\lstdefinestyle{txt} {
	language=Ruby,
	basicstyle=\scriptsize\ttfamily,
	%numbers=left,
	numbers=none,
	numberstyle=\tiny\zebra{gray!10}{white},
	frame=tb,
	postbreak=\mbox{\textcolor{red}{$\hookrightarrow$}\space},
	tabsize=4,
	columns=fixed,
	showstringspaces=false,
	showtabs=false,
	keepspaces,
	commentstyle=\color{black},
	stringstyle=\color{black},
	keywordstyle=\color{black}
}

%\lstdefinestyle{C} {
%	language=C,		%for syntax highlighting
%	basicstyle=\footnotesize\ttfamily,
%	frame=tb,
%	tabsize=4,
%	columns=fixed,
%	showstringspaces=false,
%	showtabs=false,
%	keepspaces,
	%commentstyle=\color{gray},
	%stringstyle=\color{brown},
	%keywordstyle=\color{blue},
	%directivestyle=\color{gray},			%preprocessor color
	%emph={int,char,double,float,unsigned},
	%emphstyle=\color{green},
%}
%\lstset{ language=C }	%		\begin{lstlisting}[style=C]		\lstinputlisting[style=C]{main.c}



%\lstdefinestyle{sql} {
%	language=sql,		%for syntax highlighting
%	basicstyle=\footnotesize\ttfamily,
%	frame=tb,
%	tabsize=4,
%	columns=fixed,
%	showstringspaces=false,
%	showtabs=false,
%	keepspaces,
%	commentstyle=\color{red},
%	stringstyle=\color{orange},		%needs xcolor
%	keywordstyle=\color{blue}
%}
%\lstset{ language=sql }		\begin{lstlisting}		\lstinputlisting[style=sql]{file.sql}




%\lstdefinestyle{python} {
	%language=python,		%for syntax highlighting
	%basicstyle=\footnotesize\ttfamily,
	%frame=tb,
	%tabsize=4,
	%columns=fixed,
	%showstringspaces=false,
	%showtabs=false,
	%keepspaces,
	%commentstyle=\color{gray},
	%stringstyle=\color{brown},
	%keywordstyle=\color{blue},
	%directivestyle=\color{gray},			%shebang color
	%emphstyle=\color{green},
%}
%\lstset{ language=python }			%\begin{lstlisting}		\lstinputlisting[style=python]{script.py}



%\lstdefinestyle{file} {
	%basicstyle=\footnotesize\ttfamily,
	%frame=tblr,
	%tabsize=4,
	%columns=fixed,
	%showstringspaces=false,
	%showtabs=false,
	%keepspaces,
	%title=\lstname
%}


%\lstset{		%basicstyle=\small\sffamily,
	%basicstyle=\footnotesize\ttfamily,
	%numbers=left,
	%numberstyle=\tiny\zebra{green!10}{white},
	%frame=tb,
	%tabsize=4,
	%columns=fixed,
	%showstringspaces=false,
	%showtabs=false,
	%keepspaces,
	%commentstyle=\color{red},
	%stringstyle=\color{orange},
	%extendedchars=true,
	%literate={á}{{\'a}}1
	%{é}{{\'e}}1
	%%{í}{{\'{\i}}}1
	%{í}{{\'i}}1
	%{ó}{{\'o}}1
	%{ú}{{\'u}}1
	%{Á}{{\'A}}1
	%{É}{{\'E}}1
	%{Í}{{\'I}}1
	%{Ó}{{\'O}}1
	%{Ú}{{\'U}}1
	%%{ü}{{\"u}}1
	%%{Ü}{{\"U}}1
	%{ñ}{{\~n}}1
	%{Ñ}{{\~N}}1
	%{¿}{{?``}}1
	%{¡}{{!``}}1,
	%keywordstyle=\color{blue}
%}




%atm general file listing with
%\lstinputlisting[style=file]{}


%%%%%%%%%%%%%%%%%%%%%%%%%		END CURRENT DOCUMENT CONF		%%%%%%%%%%%%%%%%%%%%%%%%%





\begin{document}

\pagestyle{empty}
\begin{center}
{\bf\Large UNIVERSITY OF SANTIAGO DE COMPOSTELA}

\vspace{0.5cm}
\includegraphics[width=5cm]{figuras/logo_usc.eps}

\vspace{0.5cm}
{\bf\large ESCOLA TÉCNICA SUPERIOR DE ENXEÑARÍA}

\vspace{2cm}
{\bf\LARGE Improvements in IDS: adding functionality to Wazuh}

%\vspace{0.5cm}
%{\bf\LARGE Subtítulo do Traballo de Fin de Grao}
\end{center}

\vspace{2cm}
\hspace{4cm}\begin{tabular}{l}
{\it\Large Autor:} \\
{\bf\Large Andrés Santiago Gómez Vidal} \\
~ \\
{\it\Large Directores:} \\
{\bf\Large Purificación Cariñena Amigo} \\
{\bf\Large Andrés Tarascó Acuña} \\
\end{tabular}

\vspace{2cm}
\begin{center}
{\bf\Large Computer Engineering Degree}

\vspace{0.5cm}
{\bf\large July 2019}

\vspace{0.5cm}
Final degree project presented at the Escola Técnica Superior de Enxeñaría of the University of Santiago de Compostela to obtain the Degree in Computer Engineering
\end{center}



\cleardoublepage
\pagestyle{plain}
\pagenumbering{roman}
\includegraphics[width=4cm]{figuras/logo_usc.eps}

\vspace{1cm}
{\bf Ms. Purificación Cariñena Amigo}, Professor Computing Science and Artificial Intelligence at the University of Santiago de Compostela and {\bf Mr. Andrés Tarascó Acuña}, Managing Director at Tarlogic Security S.L.

\vspace{1cm}
STATE:

\vspace{1cm}
That the present report entitled \textit{Improvements in IDS: adding functionality to Wazuh} written by \textbf{Andrés Santiago Gómez Vidal} in order to obtain the ECTS corresponding to the final degree project of the Computer Engineering degree was conducted under our direction in the department of Computer Science and Artificial Intelligence of the University of Santiago de Compostela.

\vspace{1cm}
For the purpose to be duly recorded, this document was signed in Santiago de Compostela on February TODO, 2019:

\vspace{2cm}
\begin{tabular}{lll}
The director, & The codirector, & The student, \\
~ \\
~ \\
~ \\
~ \\
~ \\
~ \\
~ \\
(Purificación Cariñena Amigo) & (Andrés Tarascó Acuña) & (Andrés Gómez Vidal)
\end{tabular}

 % paxina de certificación (optativa)

%\cleardoublepage
%\include{capitulos/agradecementos} % paxina de agradecementos (optativa) 

%\cleardoublepage
%\include{capitulos/resumo} % páxina de resumo (optativa) 

\cleardoublepage
\pagestyle{plain}
\tableofcontents
\listoffigures
\listoftables



% AGORA INCLUIMOS OS CAPÍTULOS. CAMBIAMOS A NUMERACIÓN E AS CABECEIRAS
\cleardoublepage
\pagenumbering{arabic}
\setcounter{page}{1}
\pagestyle{headings}
%%Introdución: composta por Obxectivos Xerais, Relación da Documentación que conforma a Memoria, Descrición do Sistema, Información Adicional de Interese (métodos, técnicas ou arquitecturas utilizadas, xustificación da súa elección, etc.).

%This project was made in collaboration with the cybersecurity company Tarlogic SL, even though I am not a member of Tarlogic and have never worked with them in the past.
%Due to my lack of professional experience in cybersecurity and the need to research in this project, the scope and planning of the work had to be rescheduled.
%Furthermore in this project there are no absolute constraints or objectives, as it was suggested as a case between investigation (with some coding) and cybersecurity auditing, so the scope can be reduced if the time remaining is too short.

\section{Motivation}

Cybersecurity nowadays is very complex: there are many sub-fields and expert tools and it could be argued that it is impossible to guarantee that any system is totally safe.
In this project we put ourselves in the shoes of a system administrator for an enterprise, that wants to improve the security by detecting intrusions in the servers he works on. This is key to decide which technologies and tools we choose in this project.
\linej
\linej
Cybersecurity measures can be applied in multiple layers of the system, each with different tools, objectives, advantages and costs.
In general the security of a system can be divided into the next parts:
\begin{enumerate}
	\item \textbf{Firewall}: Control the inbound/outbound connections, on the \textbf{network layer}. In our scenario its objective is to reduce the amount of inbound connections, reducing the chance of intrusion.
	\item \textbf{IPS}: Intrusion Prevention System to minimize the chance of intrusions, on the \textbf{network and host layers}. Provides active protection by actions.
	\item \textbf{IDS}: Intrusion Detection System to mitigate the damage of intrusions, on the \textbf{network and host layers}. Provides passive protection by alerts.
\end{enumerate}

\linej
The next table shows a \textbf{simplified} flow on how the information is processed by the security layers and methods.
For example an IDS can monitor the network connections, scanning the whole packet (header and payload) and filing a report if needed, but has worse performance than a firewall because they only scan the header of the packet and just opt to reject them\cite{firewall-ipds-ids_comparison}.
IDS fall into the SIEM category: software that manages information and events in real-time.

\begin{table}[H]
	\centering
	\caption{Simplification of the data flow}
\linej
	\begin{tabular}{|c|c|c|c|}
	\hline
		\textbf{Layer} & \textbf{Network} & \multicolumn{2}{c|}{\textbf{Network and Host}}\\ \hline
		\textbf{Method} & Firewall & IPS & IDS\\ \hline
		\textbf{Measures} & Prevent & Prevent & Mitigate\\ \hline
	\end{tabular}
\end{table}
$\xrightarrow{\makebox[\textwidth]{Direction of the data flow}}$

\linej
\linej
An IDS that focus on network monitoring is a NIDS.
They have become widely used over the past two decades because of the impressive capability to provide a granular view of what is happening on the network.
\linej
Attackers have grown used to NIDSs and have found ways to evade them, like\cite{libro_ossec}:
\begin{enumerate}
	\item Avoid using known patterns in their connections.
	\item Use encrypted connections.
	\item Send the data in pieces accross the network. This does not work against NIDSs that can reassemble them, at a greater computing cost. %session splicing and framentation attacks
	\item Denial of Service attacks: too much traffic overloads the NIDS, blinding it.
\end{enumerate}
\linej
We understand that NIDSs are useful in many situations, and there are many cases in this project where they could be used to complement an HIDS (Host-based IDS).
An HIDS can inspect the full stream of communications, making useless the techniques 2 and 3 in the previous example for evading NIDSs.
\linej
We focus on HIDS because we are more interested about detection at host level, rather than network.
Also IDS is less explored than IPS or firewalls and due to the advance in gathering and processing of data in the last years IDS has become much more viable and reliable.

\linej
\linej
IDSs are different from antivirus or antimalware because the first are systems \textbf{specialized} in detection and the latter usually focus on prevention, however prevention and detection are often meshed together because both are deeply related. There are some cases where a system specialized in detection offers some kind of mitigation functionality or one specialized in prevention offers some kind of detection functionality.

\linej
\linej
It is important to note that in cybersecurity the trend is for the attack to be created first and later some kind of measures, not necessarily by the same teams as they usually are specialized in each role. This means that defensive security that requires manual intervention often lags behind.
\linej
Nowadays there are lots of different attacks, so many that their detection could be almost impossible one by one, but most of them can be detected because they share patterns. If we can determine the patterns of an attack and code a way to detect them we can detect the threat. Some times it is easier to detect the attack and take measures after the intrusion has taken place.
\linej
\linej
IDSs work by analysing the key information available (programs, logs, network information, etc) to determine if there has been an intrusion in the system. The details of the process vary with each IDS but in general they work like an expert system:
\begin{itemize}
	\item The source of the data is the system.
	\item The alerts are set by certain rules when they match.
	\item Rules do not need to throw an alert and there can be dependencies, allowing a stateful approach and complex analysis without false positives (the main annoyance of IDSs).
\end{itemize}

\linej
There are two types of IDS, based on the detection mechanism:
\begin{itemize}
	\item Signature based: The IDS looks for specific data (signature), for example a string. This is often an efficient solution to known attacks, but is fundamentally useless against unknown attacks (attacks without a signature in the IDS database).
	\item Behaviour analysis: After a training period the IDS can detect when an event is rare (by probability) and correlate these suspicious occurrences to an intrusion.
\end{itemize}
In our case we take interest in the signature approach because it is much more used and behavior analysis is more fit for networks than for hosts.
Abnormal or suspicious behavior is called \textit{noise} in cybersecurity jargon.

\linej
\linej
OSSEC is an HIDS solution with detection based on rules and decoders. Both rules and decoders can be defined with numerous options and support dependencies and regular expressions.
\begin{itemize}
	\item The decoders format the data for the rules.
	\item There is a threat if the conditions of the rule are met.
\end{itemize}
\linej
\textbf{OSSEC} stands for \textbf{O}pen \textbf{S}ource HIDS \textbf{SEC}urity and is interesting for this project because\cite{ossec}\cite{wazuh_additional_functionality}:
\begin{itemize}
	\item \textbf{Widely Used}: OSSEC is a growing project, used by many different entities (ISPs, universities, governments, large corporate data centers) as their main HIDS solution. In addition to being deployed as an HIDS, it is commonly used strictly as a log analysis tool, monitoring and analyzing firewalls, IDSs, web servers and authentication logs.
	\item \textbf{Scalable}: Because it is an HIDS and it uses \textbf{agents}. Each monitored host can either install the agent or use an agentless agent\cite{agentless}\cite{ossec_agent}. Agentless agents are processes initiated from the OSSEC manager, which gather information from remote systems, and use any RPC method (e.g. SSH, SNMP, RDP, WMI).
	\item \textbf{Multi-platform}: GNU/Linux, Windows, Mac OS and Solaris. This is important because most professional services are on GNU/Linux or Windows, but it is important to note that rules can only work in one operating system.
	\item \textbf{Free}: OSSEC is a free software and will remain so in the future; you can redistribute it and/or modify it under the terms of the GNU General Public License (version 2) as published by the FSF -- Free Software Foundation.
	\item \textbf{Open source}: The code is open, so you can read, contribute and debug it all you want.
	\item \textbf{Rootkits detection}: This type of malware usually replaces or changes existing operating system components in order to alter the behavior of the system. Rootkits can hide other processes, files or network connections like itself.
	\item \textbf{File integrity monitoring}: To detect access or changes to sensitive data.
\end{itemize}
\linej
There are lots of alternatives to OSSEC for the scenario of a system administrator that wants to reinforce the security of the systems he is responsible for. There exist free of charge and paid solutions. Not all are pure IDSs and often they specialize in a field. For example the next table shows a comparison of the most important ICSs (Industrial Control Systems), which is a genetic type of control system that includes IDS, therefore it shows a comparison of OSSEC with similar software:
\begin{figure}[H]
  \centering
	\includegraphics[width=\textwidth]{figuras/comparison_ics.png}
	\caption{Comparison by attributes of the most important ICSs\cite{comparison_ics}}
\end{figure}

\linej
\linej
One of the problems of a comparison in a table like this is that it fails to show how much a tool excels or lacks in the features it shares with others, how easy it is to use and other factors that can help to choose the right tool. The most relevant alternative technologies to OSSEC for this project are\cite{comparison_tools}:
\begin{itemize}
	\item Sagan: An open source HIDS, but it only supports *nix operating systems (Linux, FreeBSD, OpenBSD, etc) and it lacks in features compared to OSSEC.
	\item YARA: It is not an IDS or IPS, it is just a tool that does pattern/string/signature matching, but it excels at it in performance, results and easiness to write the rules. It can be used to scan the \textbf{memory} for known patterns. YARA is being used widely in cybersecurity, for example by Avast, Kaspersky Lab, VirusTotal and McAfee Advanced Threat Defense\cite{who_is_using_yara}. We could build a system to use YARA to scan files but always combined with at least another tool, but we prefer to stick to a tested IDS.
\end{itemize}
\linej
Due to their popularity it is worth mentioning the next tools, even though they are only for network:
\begin{itemize}
	\item Bro: It is an open source IDS and supports only Linux, FreeBSD, and Mac OS.
	\item Snort: It is the most popular open source IDS/IPS, but can be expensive in processing power.
	\item Suricata: Another open source IDS/IPS solution. It provides hardware acceleration and multi-threading to improve the scanning speed.
\end{itemize}
\linej
Most of the attributes in the previous comparison are not relevant for our work.
We chose OSSEC because of the problems found on the alternatives.
Also OSSEC offers a reliable way to use an already developed and thoroughly tested IDS, which we can enhance to our needs without much work.
To even ease more this we will use Wazuh, a fork of OSSEC.

\section{Objectives}
Quality is valued more than quantity in this project. Therefore anything will be reworked or discarded if it does not fully satisfy the student, Tarlogic or the professor.
\linej
\linej
The main objective is to improve intrusion detection in IDS. This can be accomplished in several ways:
\begin{itemize}
	\item Adding or changing functionality of an already existing technology.
		\subitem Coding on core or additions.
		\subitem Configuration or input of the program.
\end{itemize}
\linej
In this project the focus is on the configuration, particularly of rules to detect certain attacks. It is necessary to fully understand the attacks first to code its detection, therefore preparing the attacks also will need a fair amount of time.
It is important to explain the attacks and their detection clearly, in order to make this work useful for anyone else and ease any changes.

\section{Structure of this document}
%TODO Structure of this document
This document has TODO chapters:
\begin{itemize}
	\item In \textbf{chapter 1} 
	\item In \textbf{chapter 2} 
	\item In \textbf{chapter 3} 
	\item In \textbf{chapter 4} 
	\item In \textbf{chapter 5} 
	\item In \textbf{chapter 6} 
	\item In \textbf{chapter 7} 
\end{itemize}



\cleardoublepage
\chapter{Requirements}

%Especificación de requisitos: debe indicarse, polo miúdo, a especificación do 
%Sistema, xunto coa información que este debe almacenar e as interfaces con outros 
%Sistemas, sexan hardware ou software, e outros requisitos (rendemento, seguridade, 
%etc).


The requirement specification is a full description of the software the project is to develop.
\linej
PMBOK\cite{pmbok} states that requirements are conditions or capabilities that a product must meet to satisfy the contract.
The requirements expose the needs of the client, which have to be accomplished to finish the project successfully.
In this project the requirements will be fullfilled in multiple stages along the project.
\linej
Note that the client in this case is Tarlogic even if the product is a contribution to an open source project.

\linej
\linej
This specification contains:
\begin{itemize}
	\item \textbf{Use cases}: Functionalities that the software will provide.
	\item \textbf{Requirements}: Depending of their type they can describe features, data, relations, properties or any details necessary to explain the system without ambiguity, in a way it can be easily understood.
\end{itemize}

In this project the functional requirements are not included because they can be considered a redundant version of the use cases, because both describe the same functionalities.
Uses cases were chosen over functional requirements because they were considered to be easier to understand and have greater detail. If this project had the need of a very complex requirement specification it would be interesting to have both, as each could help to understand the other better, but in this project the specification should be quite simple.






\section{Use cases}
A use case is a description of all the ways an end-user wants to ``use'' a system. These ``uses'' are like requests of the system, and use cases describe what that system does in response to such requests. In other words, use cases describe the conversation between a system and its user(s), known as actors. Although the system is usually automated (such as an Order system), use cases also apply to equipment, devices, or business processes.\cite{use_case_definition}


\subsection{Use cases actors}
The actors are entities external to the system that interact with it. They can be other systems, persons or even time.

%\subsection{Use cases description}
\subsection{Use cases list}

\section{Requirements analysis}

\subsection{Non functional requirements}

\subsection{Functional requirements}
As mentioned before these are omited because of the redundancy with use cases.

\subsection{Domain requirements}


%\cleardoublepage
%\include{capitulos/design}

\cleardoublepage
\chapter{Planning}

%Planificación e presupostos: debe incluír a estimación do costo (presuposto) e dos 
%recursos necesarios para efectuar a implantación do Traballo, xunto coa planificación 
%temporal do mesmo e a división en fases e tarefas. Recoméndase diferenciar os costos relativos a persoal dos relativos a outros gastos como instalacións e equipos.








%WBS==EDT (translation!)
\section{Initial WBS}

%\node[style1] {Improvements in IDS, adding functionality to Wazuh}
%child {node[style2] (c1) {Project management}}
%child {node[style2] (c2) {Increment 1: Rules/decoders for common attacks in Windows Server}}
%child {node[style2] (c3) {Increment 2: Rules/decoders with data from Sysmon}}
%child {node[style2] (c4) {Increment 3: Detection/action against ransomware}}
%child {node[style2] (c5) {Increment 4: Adapt Wazuh configuration to typical requirements from enterprises}}
%child {node[style2] (c6) {Increment 5: Explore solutions in problems with GPDR}}
%child {node[style2] (c7) {Increment 6: Rules/decoders for GNU/Linux}}
%child {node[style2] (c8) {Increment 7: VirusTotal integration}};

%\begin{tikzpicture}[%
    %grow=right,
    %anchor=west,
    %growth parent anchor=east,
    %parent anchor=east,
    %level 1/.style={sibling distance=4cm},
    %level 2/.style={sibling distance=2.5em},
    %level distance=1cm]

%\node[root] (root) {Drawing diagrams}
    %child {node[onode] (c1) {Defining node and arrow styles}
        %child {node[tnode] (c11) {Setting shape}}
        %child {node[tnode] (c12) {Choosing color}}
        %child {node[tnode] (c13) {Adding shading}}
    %}
    %child {node[onode] (c2) {Positioning the nodes}
        %child {node[tnode] (c21) {Using a Matrix}}
        %child {node[tnode] (c22) {Relatively}}
        %child {node[tnode] (c23) {Absolutely}}
        %child {node[tnode] (c24) {Using overlays}}
    %}
    %child {node[onode] (c3) {Drawing arrows between nodes}
        %child {node[tnode] (c31) {Default arrows}}
        %child {node[tnode] (c32) {Arrow library}}
        %child {node[tnode] (c33) {Resizing tips}}
        %child {node[tnode] (c34) {Shortening}}
        %child {node[tnode] (c35) {Bending}}
%};
%\end{tikzpicture}
\newpage
{\footnotesize
\begin{forest} for tree={
    grow=east,
    growth parent anchor=east,
    parent anchor=east,
    child anchor=west,
    edge path={\noexpand\path[\forestoption{edge},->, >={latex}] 
         (!u.parent anchor) -- +(5pt,0pt) |- (.child anchor)
         \forestoption{edge label};}
}
[Improvements in IDS: adding functionality to Wazuh, root
    [Closing of the project, onode
        [Project documentation, tnode]
        [Pull request to the official ruleset repository, tnode]
    ]
    [Increment 7: VirusTotal integration, onode
        [Improved integration with antivirus and website scanners, tnode]
        %[Study of the present status, tnode]
    ]
    [Increment 6: Additional detection for GNU/Linux, onode
        [Rules and decoders, tnode]
        %[Analysis and writing of rules/decoders, tnode]
        %[Study of the present status, tnode]
    ]
    [Increment 5: Explore solutions in problems with GPDR, onode
        [Rules and decoders, tnode]
        %[Analysis and writing of rules/decoders, tnode]
        %[Study of GPDR issues, tnode]
    ]
    [Increment 4: Adapt Wazuh configuration to typical requirements from enterprises, onode
        [Configuration changes, tnode]
        %[Analysis of requirements, tnode]
    ]
    [Increment 3: Detection/action against ransomware, onode
        [Rules and decoders, tnode]
        %[Analysis and writing of rules/decoders and actions, tnode]
        %[Study of ransomware attack patterns, tnode]
    ]
    [Increment 2: Use of data from Sysmon, onode
        [Rules and decoders, tnode]
        %[Analysis and writing of rules/decoders, tnode]
        %[Study of Sysmon tools, tnode]
    ]
    [Increment 1: Common attacks in Windows Server, onode
        [Rules and decoders, tnode]
        %[Detailed study of each attack and patterns, tnode]
    ]
    [Beginning of the project, onode
        [Setup of the work environment, tnode]
        [Study of Wazuh documentation and related tools and technologies, tnode]
    ]
    [Project management, onode
        [Cost management, tnode]
        [Configuration management, tnode]
        [Time management, tnode]
        [Risk management, tnode]
        [Requirement management, tnode]
        [Scope management, tnode]
    ]
]
\end{forest}
}


\section{Initial planning}






\section{Final planning}


\cleardoublepage
\section{Risk management}

\subsection{Risk metrics}

\begin{table}[H]
	\centering
	\begin{tabular}{|l|l|}
		\hline
		\rowcolor{gray!30}
		Chances of the risk happening & Probability \\ \hline
		$\geq$80\% & \cellcolor{red!60}High\\ \hline
		Between 30\% and 80\% & \cellcolor{yellow!40}Medium\\ \hline
		$\leq$30\% & \cellcolor{green!60}Low\\ \hline
	\end{tabular}
	\caption{Probability classification of risks}
\end{table}


\begin{table}[H]
	\centering
	\begin{tabular}{|l|l|}
		\hline
		\rowcolor{gray!30}
		Resource in Place / Effort / Cost & Impact \\ \hline
		$\geq$20\% & \cellcolor{red!60}High\\ \hline
		Between 10\% and 20\% & \cellcolor{yellow!40}Medium\\ \hline
		$\leq$10\% & \cellcolor{green!60}Low\\ \hline
	\end{tabular}
	\caption{Impact classification of risks}
\end{table}


\begin{table}[H]
	\centering
	\begin{tabular}{|c|c|c|c|c|}
	\hline
		\multicolumn{2}{|c|}{\multirow{2}{*}{\large\textbf{Exposition}}} & \multicolumn{3}{c|}{Probability}\\
		\multicolumn{2}{|c|}{} & \cellcolor{gray!15}\textbf{High} & \cellcolor{gray!15}\textbf{Medium} & \cellcolor{gray!15}\textbf{Low}\\ \hline %\cline{3-5}
		\multirow{3}{*}{Impact} & \cellcolor{gray!15}\textbf{High} & \cellcolor{red!60}High & \cellcolor{red!60}High & \cellcolor{yellow!40}Medium\\
		& \cellcolor{gray!15}\textbf{Medium} & \cellcolor{red!60}High & \cellcolor{yellow!40}Medium & \cellcolor{green!60}Low\\
		& \cellcolor{gray!15}\textbf{Low} & \cellcolor{yellow!40}Medium & \cellcolor{green!60}Low & \cellcolor{green!60}Low\\ \hline
	\end{tabular}
	\caption{Method of calculation of Exposition based of Probability and Impact}
\end{table}

\subsection{Risk types}

\subsection{Risk identification}

\begin{table}[H]
	\caption{Project risks}
	\begin{tabularx}{\textwidth}{|l|X|}
		\hline
		\rowcolor{gray!30}
		Identifier & Name \\ \hline
		%R-000 & The scope specified is too big\\ \hline
		R-001 & Optimist planning, ``best case'' (instead of a realistic ``expected case'')\\ \hline
		R-002 & Bad requirement specification\\ \hline
		R-003 & Design errors\\ \hline

		R-004 & Lack of key information from sources\\ \hline 		%articles, documentation, manuals
		R-005 & Lack of feedback or support from the security consultants of Tarlogic\\ \hline 		%eg: holidays or sickness
		R-006 & The learning curve of some technologies is larger than expected\\ \hline
		R-007 & The unexplained parts of the project take more time than expected\\ \hline

		R-008 & Cannot access source material\\ \hline 		%is down, is too old
		R-009 & Unexpected changes to any of the APIs used in the project\\ \hline

		R-010 & Loss of work\\ \hline 	%use git, online repo, local backups for virtual machines?
		R-011 & Wrong management of the project's configuration\\ \hline 	%wrong baseline, wrong identification of the configuration elements, it takes more time than expected, wrong use of the tools, too much time between commits
		R-012 & A delay in one task leads to cascading delays in the dependent tasks\\ \hline

		R-013 & Unnecessary work\\ \hline 		%
		R-014 & The quality of the product is not enough\\ \hline 	%redo
		R-015 & Sickness or overwork\\ \hline
		R-016 & Performance issues\\ \hline 	%redo rules?, get a better machine?
	\end{tabularx}
\end{table}





\subsection{Risk analysis}


\begin{table}[H]
	\begin{tabularx}{\textwidth}{|l|X|}
		\hline
		\rowcolor{gray!30}
		Identifier & \textbf{R-000} \\ \hline
		Name & Bla \\ \hline
		Description & Bla bla bla bla bla bla bla bla bla bla bla bla bla Bla bla bla bla bla bla bla bla bla bla bla bla bla Bla bla bla bla bla bla bla bla bla bla bla bla bla \\ \hline
		Probability & Low , Medium , High \\ \hline
		Impact & Low , Medium , High \\ \hline
		Exposition & Low , Medium , High \\ \hline
		Indicator & Bla bla bla bla bla bla bla bla bla bla bla bla bla \\ \hline
	\end{tabularx}
\end{table}

\begin{table}[H]
	\begin{tabularx}{\textwidth}{|l|X|}
		\hline
		\rowcolor{gray!30}
		Identifier & \textbf{R-001} \\ \hline
		Name & Optimist planning, ``best case'' (instead of a realistic ``expected case'')\\ \hline
		Description & An optimistic planning at the start of the project does not take into account problems or delays, and so it does not allocate time for them, leading to cascading delays if they happen. \\ \hline
		Probability & Medium\\ \hline
		Impact &  High\\ \hline
		Exposition &  High\\ \hline
		Indicator & There are 3 consecutive delays, after the beginning of the project.\\ \hline
	\end{tabularx}
\end{table}



\begin{table}[H]
	\begin{tabularx}{\textwidth}{|l|X|}
		\hline
		\rowcolor{gray!30}
		Identifier & \textbf{R-002} \\ \hline
		Name & Bad requirement specification\\ \hline
		Description & The requirements specified at the beginning of the project are not specific enough, are not needed or there are new requirements after the beginning of the project. \\ \hline
		Probability & High\\ \hline
		Impact &  High\\ \hline
		Exposition &  High\\ \hline
		Indicator & There are 3 changes in the requirements specification.\\ \hline
	\end{tabularx}
\end{table}



\begin{table}[H]
	\begin{tabularx}{\textwidth}{|l|X|}
		\hline
		\rowcolor{gray!30}
		Identifier & \textbf{R-003} \\ \hline
		Name & Design errors\\ \hline
		Description & A design is not enough or is incorrect, needing a re-design and probably changes in the next steps it was used. \\ \hline
		Probability & Low\\ \hline
		Impact &  Medium\\ \hline
		Exposition & Medium\\ \hline
		Indicator & There are 3 designs that need rework.\\ \hline
	\end{tabularx}
\end{table}



\begin{table}[H]
	\begin{tabularx}{\textwidth}{|l|X|}
		\hline
		\rowcolor{gray!30}
		Identifier & \textbf{R-004} \\ \hline
		Name & Lack of key information from sources\\ \hline
		Description & Not having key information from articles, documentation or manuals can result in unexpected delays, added difficulty or the need to rework completely the functionality.\\ \hline
		Probability & Medium\\ \hline
		Impact &  High\\ \hline
		Exposition &  High\\ \hline
		Indicator & The duration of the study of the attack and the needed tools takes 50\% than expected. \\ \hline
	\end{tabularx}
\end{table}



\begin{table}[H]
	\begin{tabularx}{\textwidth}{|l|X|}
		\hline
		\rowcolor{gray!30}
		Identifier & \textbf{R-005} \\ \hline
		Name & Lack of feedback or support from the security consultants of Tarlogic\\ \hline
		Description & Because I do not know enough of some technical aspects of cibersecurity to solve all the problems in this by myself in time, Tarlogic has promised to help (in a tutoring way) if a problem arises.
		This help could be critical to solve or get around some of the most complex problems, which probably happen to be critical points, needing to be dealt with to continue working on that stage.\\ \hline
		Probability & Medium\\ \hline
		Impact &  High\\ \hline
		Exposition &  High\\ \hline
		Indicator & A simple technical question takes more than 2 working days to be answered or a complex question takes more than 7 working days.\\ \hline
	\end{tabularx}
\end{table}



\begin{table}[H]
	\begin{tabularx}{\textwidth}{|l|X|}
		\hline
		\rowcolor{gray!30}
		Identifier & \textbf{R-006} \\ \hline
		Name & The learning curve of some technologies is larger than expected\\ \hline
		Description & This is a critical need because not having enough knowledge can result in unexpected delays, added difficulty or the need to rework completely the functionality.\\ \hline
		Probability & Medium\\ \hline
		Impact &  Medium\\ \hline
		Exposition &  Medium\\ \hline
		Indicator & The duration of the study of the technologies takes 50\% than expected. \\ \hline
	\end{tabularx}
\end{table}

\begin{table}[H]
	\begin{tabularx}{\textwidth}{|l|X|}
		\hline
		\rowcolor{gray!30}
		Identifier & \textbf{R-007} \\ \hline
		Name & The unexplained parts of the project take more time than expected\\ \hline
		Description & There is not enough specification on what a tasks implies or not enough planning. This means that a part of the project is not understood as it should, and the work done is not what was expected or is not enough, needing more time to finish. \\ \hline
		Probability & Low\\ \hline
		Impact &  High\\ \hline
		Exposition &  Medium\\ \hline
		Indicator & A task takes 15\% more time than expected and when the causes are investigated it is revealed that there were ambiguous descriptions or planning.\\ \hline
	\end{tabularx}
\end{table}
\begin{table}[H]
	\begin{tabularx}{\textwidth}{|l|X|}
		\hline
		\rowcolor{gray!30}
		Identifier & \textbf{R-008} \\ \hline
		Name & Cannot access source material\\ \hline
		Description & All or part of the source material can not be accessed, probably because the only host of the resource is down. In some cases this could mean a delay in a critical task, cascading in other delays and delaying the project for a period unknown.\\ \hline
		Probability & Low\\ \hline
		Impact & High\\ \hline
		Exposition & Medium\\ \hline
		Indicator & There have been at least 10 failed attempts to download the source material, at least 5 with a computer A in a network X and at least 5 with a computer B in a network Y.\\ \hline
	\end{tabularx}
\end{table}
\begin{table}[H]
	\begin{tabularx}{\textwidth}{|l|X|}
		\hline
		\rowcolor{gray!30}
		Identifier & \textbf{R-009} \\ \hline
		Name & Unexpected changes to any of the APIs used in the project\\ \hline
		Description & Changes to an API could affect this project directly or indirectly. Programs could fail or not work as expected.
		In a project that does not work in a bleeding edge environment, like this, this should be very rare and even if it were to happen it would have to interfere with the part of the API this project uses, which (as this is not bleeding edge) normally would be backwards compatible.\\ \hline
		Probability & Low\\ \hline
		Impact & Low \\ \hline
		Exposition &  Low\\ \hline
		Indicator & There are 3 failures due to changes in APIs.\\ \hline
	\end{tabularx}
\end{table}
\begin{table}[H]
	\begin{tabularx}{\textwidth}{|l|X|}
		\hline
		\rowcolor{gray!30}
		Identifier & \textbf{R-010} \\ \hline
		Name & Loss of work\\ \hline
		Description & Due to a bad configuration management or something else, there is a loss of work related to this project.\\ \hline
		Probability & Low\\ \hline
		Impact &  High\\ \hline
		Exposition &  Medium\\ \hline
		Indicator & The need to replicate already done work is greater than 30 minutes.\\ \hline
	\end{tabularx}
\end{table}
\begin{table}[H]
	\begin{tabularx}{\textwidth}{|l|X|}
		\hline
		\rowcolor{gray!30}
		Identifier & \textbf{R-011} \\ \hline
		Name & Wrong management of the project's configuration\\ \hline
		Description & The project's configuration is inefficient or lacks work.
		Some of the problems could be:
		\vspace{-0.5em}
		\begin{itemize}
		\setlength\itemsep{0em}
			\item Wrong baselines
			\item Wrong identification of the configuration elements
			\item It takes more time than expected to manage the project
			\item Wrong use of the tools
			\item Too much time between commits
			\item Changes are unclear
		\end{itemize}
		\vspace{-0.5em}
		This means the project suffer delays because the need to redo management work and/or planned tasks. \\ \hline
		Probability & Medium\\ \hline
		Impact &  High\\ \hline
		Exposition &  High\\ \hline
		Indicator & There are 3 delays because of the configuration of the project.\\ \hline
	\end{tabularx}
\end{table}
\begin{table}[H]
	\begin{tabularx}{\textwidth}{|l|X|}
		\hline
		\rowcolor{gray!30}
		Identifier & \textbf{R-012} \\ \hline
		Name & A delay in one task leads to cascading delays in the dependent tasks\\ \hline
		Description & A task gets delayed and one or more tasks depends on its completion to start, so they get delayed too.\\ \hline
		Probability & Medium\\ \hline
		Impact &  Medium\\ \hline
		Exposition &  Medium\\ \hline
		Indicator & At least 2 tasks are delayed, due to only one of them needing more time.\\ \hline
	\end{tabularx}
\end{table}
\begin{table}[H]
	\begin{tabularx}{\textwidth}{|l|X|}
		\hline
		\rowcolor{gray!30}
		Identifier & \textbf{R-013} \\ \hline
		Name & Unnecessary work\\ \hline
		Description & Resources are wasted in work that latter is not used. This could happen because multiple reasons, like wrong assumptions or balancing of the remaining time of the project.\\ \hline
		Probability & Low\\ \hline
		Impact &  Low\\ \hline
		Exposition &  Low\\ \hline
		Indicator & At least 3 commits are reverted.\\ \hline
	\end{tabularx}
\end{table}
\begin{table}[H]
	\begin{tabularx}{\textwidth}{|l|X|}
		\hline
		\rowcolor{gray!30}
		Identifier & \textbf{R-014} \\ \hline
		Name & The quality of the product is not enough\\ \hline
		Description & The final result is does not comply the quality standard set for this project. This could mean the need to redo work in a later stage or the incorporation to the official repository being rejected.\\ \hline
		Probability & Low\\ \hline
		Impact &  High\\ \hline
		Exposition &  Medium\\ \hline
		Indicator & Getting 10 suggestions to rework functionality.\\ \hline
	\end{tabularx}
\end{table}
\begin{table}[H]
	\begin{tabularx}{\textwidth}{|l|X|}
		\hline
		\rowcolor{gray!30}
		Identifier & \textbf{R-015} \\ \hline
		Name & Sickness or overwork\\ \hline
		Description & The health of the student deteriorates to the point it affects the project, and it is caused by sickness or overwork.\\ \hline
		Probability & Medium\\ \hline
		Impact &  Medium\\ \hline
		Exposition &  Medium\\ \hline
		Indicator & There is an unexpected delay because the functionality is not done but there has not been any important issues that could explain it. \\ \hline
	\end{tabularx}
\end{table}
\begin{table}[H]
	\begin{tabularx}{\textwidth}{|l|X|}
		\hline
		\rowcolor{gray!30}
		Identifier & \textbf{R-016} \\ \hline
		Name & Performance issues\\ \hline
		Description & The program is too heavy for the environment and takes too much resources, because there are not good enough optimizations or the problems are poorly approached.\\ \hline
		Probability & Low\\ \hline
		Impact &  Low\\ \hline
		Exposition &  Low\\ \hline
		Indicator & The program takes 30\% more resources that at the beginning of the project.\\ \hline
	\end{tabularx}
\end{table}


\subsection{Risk planning}


\subsection{Risk supervision}


%\cleardoublepage
%\section{Conclusion}
This project has shown how Wazuh can be used to improve the cybersecurity of a system, in some cases without the need of expert knowledge in scripting or pentesting.
\linej
Wazuh has been under development for some years and there are still parts that lack basic funcionality.
This project has tried to show and fix some of them, without changing the source code in order to have more time for other tasks.
\linej
\linej
The main use of a HIDS should be to gather and process data in order to detect security threats.
Wazuh provides a set of features that make it a very good tool for monitoring system behaviour, but not as good in other aspects.
\linej
Stopping the threats can be too much for this kind of system, there are obvious limitations that make it impossible to be as effective as a local antimalware software.
Trusting user scripts for defensive security is very risky and bound to fail sooner or later.
\linej
\linej
Even though in this project only the server runs GNU/Linux is clear from its management and the documentation that Wazuh is less suited for Windows systems.
Many workarounds and third-party tools are needed because of this.
Therefore it is assumed that the project could have advanced faster if it were on GNU/Linux security instead of Windows.
\linej
\linej
Using an incremental methodology proved to be a right choice because there were many desired features that were left behind because lack of time.
It also fits very well with the work flow in this kind project, with very simple development.
The essential requirements and increments were satisfied and a bit more depth than initially planned was applied in many cases, resulting in what we consider a higher quality for the product.

\section{Additions}
Multiple ways to detect and act against malware have been analyzed in this project, but this is just the tip of the iceberg and much work remains to be done.
There are many features and research that could improve the current state of Wazuh as a reliable cybersecurity tool:
\begin{itemize}
	\item All the ideas left behind in this project due to lack of time. They are explained in the exclusions part of the scope section \ref{exclusions}.
	\item Ability to set real variables in the rules, instead of the current ones that are actually constans because they need a restart of the service to be updated.
Unfortunately this can also result in a loss of performance for processing the rules.
	\item A functional database system for all kinds of data, instead of the current CDB that only is useful for IP addresses. This would solve the problem of just monitoring a value, like the case of the free space of the backup volume \ref{free_space_volume}.
	\item Having more decoders for common monitoring commands in Windows, instead of having to write your own or parse the output of the command with an script.
	\item Executing remote commands with direct feedback to the manager, instead of having to use workarounds like writing to a monitored log.
	\item Active response directly with PowerShell, instead of having to use CMD or SSH. This is in development.
	\item Active response with dynamic arguments from the alert, instead of processing the alert manually with a local script and using another program (in this case SSH) for the remote execution.
	\item Option to have only a frequency of 1 in composite rules, which is not such a rare need, instead of needing workarounds for it.
	\item Integration with local antimalware software. This has been done using APIs and custom scripts with some software (like VirusTotal) but not with others as basic as Windows Defender.
	\item Integration with other IDSs. This can provide detection features based on behaviour or network data that Wazuh can not gather on its own.
	\item Further research and development in any of the parts treated in this project.
\end{itemize}




% AQUÍ EMPEZAN OS APÉNDICES
%\appendix
%\cleardoublepage
%\include{capitulos/apendicea}

%\cleardoublepage
%\include{capitulos/apendiceb}

%\cleardoublepage
%\include{capitulos/licenza}

%\cleardoublepage
%\include{capitulos/bibliografia}

\end{document}


