%
%  PARA TRABALLOS EN GALLEGO USAR (LINEA 12): \usepackage[galician]{babel}
%  PARA TRABALLOS EN CASTELLANO USAR (LINEA 13): \usepackage[spanish]{babel}
%
% Para los acentos usamos codificacion UTF-8 (LINEA 10): \usepackage[utf8]{inputenc} 
% Si se usase la codificacion es_ES.ISO-8859-1 (LINEA 11): \usepackage[latin1]{inputenc}
% La conversion de acentos se hace con: iconv -f UTF-8 -t ISO-8859-1 filename.tex
%
% Como se incluyen figuras eps hay que compilar con: latex traballo , dvipdf traballo
%

\documentclass[12pt,twoside,a4paper]{book}
%\documentclass[a4paper,10pt]{article}

% pódense engadir todos os packages necesarios
\usepackage[utf8]{inputenc}
%\usepackage[T1]{fontenc}	%Correct accented characters and copy paste them, also not unexpected results with some characters like pipe


%				%es-noquotes]{babel}			si caracteres dan problemas
%				%es-noshorthands]{babel}		si caracteres dan todavia problemas
%				%es-noindentfirst]{babel}		para eliminar sangría
%\usepackage[latin1]{inputenc}
%\usepackage[galician]{babel}
%\usepackage[spanish]{babel}
%\usepackage[spanish, es-tabla,es-noindentfirst]{babel}		%http://www.tex-tipografia.com/spanishopt.html
%\usepackage[spanish, es-tabla]{babel}		%http://www.tex-tipografia.com/spanishopt.html
\usepackage[english]{babel}		%http://www.tex-tipografia.com/spanishopt.html
\addto\captionsenglish{
  \renewcommand{\contentsname}
    {Index} % ToC will show "Index" instead of "Content"
}

%\usepackage[document]{ragged2e}	%centering

\usepackage{enumitem}

\usepackage{graphicx}
\usepackage[dvips]{epsfig}
\usepackage{amssymb}
\usepackage[official]{eurosym}
\usepackage{float}
\usepackage{latexsym}
\usepackage{a4}
\usepackage{hyperref}

%tables
\usepackage{multirow}
\usepackage{multicol}
%\usepackage{changepage}		%margin, eg for tables before tabular \begin{adjustwidth}{-1.4cm}{}
\usepackage{tabularx}

%\usepackage{color}
\usepackage[table,xcdraw,usenames,dvipsnames]{xcolor}


%WBS
\usepackage{forest}
\usetikzlibrary{arrows.meta,shapes,positioning,shadows,trees}
\tikzset{
    basic/.style  = {draw, text width=2cm, drop shadow, font=\sffamily, rectangle},
    root/.style   = {basic, rounded corners=2pt, thin, align=center,
                     fill=green!30, text width=2cm,},
    onode/.style = {basic, thin, rounded corners=2pt, align=left, fill=green!60,text width=6cm,},
    tnode/.style = {basic, thin, align=left, fill=gray!30, text width=5cm},
    edge from parent/.style={draw=black, edge from parent fork right}
}



\usepackage{rotating}
\usepackage{pgfgantt}


\usepackage{amsmath}		%https://www.sharelatex.com/learn/Aligning_equations_with_amsmath
\usepackage{amsfonts}		%Extended set of fonts for use in mathematics

%%\usepackage{afterpage}		%Commands specified in its argument are expanded after the current page is output. EG: \afterpage{\clearpage} and the current page will be filled up with text as usual, but then the \clearpage command will flush out all the floats before the next text page begins

\usepackage{listings}
	\lstset{breaklines}			%line wrap for listings
	\newcommand{\lstlistinginput}{\lstinputlisting}		%alias


\usepackage{courier}		%listings better font
\usepackage[sorting=none]{biblatex} %use with \cite{} and \printbibliography[type=online,heading=subbibliography,title={Referencias web}]
	\addbibresource{config/bib.bib}



%%\usepackage{microtype}		%Improves spacing (words and letters) and more stuff. Load after fonts, because is dependent on this font
%%\usepackage{siunitx}		%SI units
%%\usepackage{xspace}			%Decides whether to insert a space to replace one "eaten" by the command decoder
%%\usepackage{todonotes}		%Mark things to do later
%\usepackage{hyperref}		%\hyperref, \url and \href
%%\usepackage[a4paper]{geometry}	%Margins without needing to remember the particular page dimensions commands. Eg [a4paper], [top=1in, bottom=1.25in, left=1.25in, right=1.25in]
%%\usepackage{cleveref}		%LAST \usepackage in the preamble. If anything else modifies the referencing system (like amsmath) it all goes wrong
%
%%\DisableLigatures{encoding = *, family = *}		%https://en.wikibooks.org/wiki/LaTeX/Text_Formatting#Ligatures
%%\def\spanishoperators{}		%stuff like sin traduced to sen, max to máx, etc




\newcommand\realnumberstyle[1]{}

\makeatletter
\newcommand{\zebra}[3]{%
    {\realnumberstyle{#3}}%
    \begingroup
    \lst@basicstyle
    \ifodd\value{lstnumber}%
        \color{#1}%
    \else
        \color{#2}%
    \fi
        \rlap{\hspace*{\lst@numbersep}%
        \color@block{\linewidth}{\ht\strutbox}{\dp\strutbox}%
        }%
    \endgroup
}
\makeatother

%%%%%%%%%%%%%%%%%%%%%%%%%		MORE	ALIASES		%%%%%%%%%%%%%%%%%%%%%%%%%

\newcommand{\lineh}{\rule{\textwidth}{1pt}\hfill\break}
\newcommand{\linej}{\hfill\break}

\newcommand{\projecthours}{417}

\newcommand{\IncrementoSiete}{VirusTotal integration}
\newcommand{\IncrementoSeis}{Additional detection for GNU/Linux}
\newcommand{\IncrementoCinco}{Explore solutions in problems with GPDR}
\newcommand{\IncrementoCuatro}{Adapt Wazuh configuration to typical requirements from enterprises}
\newcommand{\IncrementoTres}{Detection/action against ransomware}
\newcommand{\IncrementoDos}{Use of more data sources}
\newcommand{\IncrementoUno}{Common attacks in Windows Server}


\newcommand{\RNO}{\cellcolor{red!60}No}
\newcommand{\RYES}{\cellcolor{green!60}Yes}

%%%%%%%%%%%%%%%%%%%%%%%%%		END BASIC CONF		%%%%%%%%%%%%%%%%%%%%%%%%%



%\definecolor{mygreen}{rgb}{0,0.6,0}
%\definecolor{mygray}{rgb}{0.5,0.5,0.5}


%\renewcommand{\thesubsection}{\thesection.\alph{subsection}}		%change subsection style to letters

%https://en.wikibooks.org/wiki/LaTeX/Source_Code_Listings


\input{config/latex-listings-powershell}
\lstdefinestyle{PS}{
	language=Powershell,		%for syntax highlighting
	basicstyle=\footnotesize\ttfamily,
	%numbers=left,
	numbers=none,
	numberstyle=\tiny\zebra{gray!10}{white},
	frame=tb,
	postbreak=\mbox{\textcolor{red}{$\hookrightarrow$}\space},
	tabsize=2,
	columns=fixed,
	showstringspaces=false,
	showtabs=false,
	keepspaces,
	commentstyle=\color{red},
	stringstyle=\color{gray},		%needs xcolor
	keywordstyle=\color{blue}
}

\lstdefinestyle{sh} {
	language=sh,		%for syntax highlighting
	basicstyle=\footnotesize\ttfamily,
	numbers=none,
	frame=tb,
	postbreak=\mbox{\textcolor{red}{$\hookrightarrow$}\space},
	tabsize=2,
	columns=fixed,
	showstringspaces=false,
	showtabs=false,
	keepspaces,
	commentstyle=\color{red},
	stringstyle=\color{gray},		%needs xcolor
	keywordstyle=\color{blue},
	%directivestyle=\color{gray},	%shebang color
	emph={local, export },
	emphstyle=\color{green},
}

\lstdefinestyle{ruby} {
	language=Ruby,		%for syntax highlighting
	basicstyle=\footnotesize\ttfamily,
	%numbers=left,
	numbers=none,
	numberstyle=\tiny\zebra{gray!10}{white},
	frame=tb,
	postbreak=\mbox{\textcolor{red}{$\hookrightarrow$}\space},
	tabsize=4,
	columns=fixed,
	showstringspaces=false,
	showtabs=false,
	keepspaces,
	commentstyle=\color{red},
	stringstyle=\color{gray},		%needs xcolor
	keywordstyle=\color{blue}
}

\lstdefinestyle{xml} {
	language=XML,		%for syntax highlighting
	basicstyle=\footnotesize\ttfamily,
	%numbers=left,
	numbers=none,
	numberstyle=\tiny\zebra{gray!10}{white},
	frame=tb,
	postbreak=\mbox{\textcolor{red}{$\hookrightarrow$}\space},
	tabsize=4,
	columns=fixed,
	showstringspaces=false,
	showtabs=false,
	keepspaces,
	commentstyle=\color{red},
	stringstyle=\color{gray},		%needs xcolor
	keywordstyle=\color{blue}
}

\lstdefinestyle{txt} {
	language=Ruby,
	basicstyle=\scriptsize\ttfamily,
	%numbers=left,
	numbers=none,
	numberstyle=\tiny\zebra{gray!10}{white},
	frame=tb,
	postbreak=\mbox{\textcolor{red}{$\hookrightarrow$}\space},
	tabsize=4,
	columns=fixed,
	showstringspaces=false,
	showtabs=false,
	keepspaces,
	commentstyle=\color{black},
	stringstyle=\color{black},
	keywordstyle=\color{black}
}

%\lstdefinestyle{C} {
%	language=C,		%for syntax highlighting
%	basicstyle=\footnotesize\ttfamily,
%	frame=tb,
%	tabsize=4,
%	columns=fixed,
%	showstringspaces=false,
%	showtabs=false,
%	keepspaces,
	%commentstyle=\color{gray},
	%stringstyle=\color{brown},
	%keywordstyle=\color{blue},
	%directivestyle=\color{gray},			%preprocessor color
	%emph={int,char,double,float,unsigned},
	%emphstyle=\color{green},
%}
%\lstset{ language=C }	%		\begin{lstlisting}[style=C]		\lstinputlisting[style=C]{main.c}



%\lstdefinestyle{sql} {
%	language=sql,		%for syntax highlighting
%	basicstyle=\footnotesize\ttfamily,
%	frame=tb,
%	tabsize=4,
%	columns=fixed,
%	showstringspaces=false,
%	showtabs=false,
%	keepspaces,
%	commentstyle=\color{red},
%	stringstyle=\color{orange},		%needs xcolor
%	keywordstyle=\color{blue}
%}
%\lstset{ language=sql }		\begin{lstlisting}		\lstinputlisting[style=sql]{file.sql}




%\lstdefinestyle{python} {
	%language=python,		%for syntax highlighting
	%basicstyle=\footnotesize\ttfamily,
	%frame=tb,
	%tabsize=4,
	%columns=fixed,
	%showstringspaces=false,
	%showtabs=false,
	%keepspaces,
	%commentstyle=\color{gray},
	%stringstyle=\color{brown},
	%keywordstyle=\color{blue},
	%directivestyle=\color{gray},			%shebang color
	%emphstyle=\color{green},
%}
%\lstset{ language=python }			%\begin{lstlisting}		\lstinputlisting[style=python]{script.py}



%\lstdefinestyle{file} {
	%basicstyle=\footnotesize\ttfamily,
	%frame=tblr,
	%tabsize=4,
	%columns=fixed,
	%showstringspaces=false,
	%showtabs=false,
	%keepspaces,
	%title=\lstname
%}


%\lstset{		%basicstyle=\small\sffamily,
	%basicstyle=\footnotesize\ttfamily,
	%numbers=left,
	%numberstyle=\tiny\zebra{green!10}{white},
	%frame=tb,
	%tabsize=4,
	%columns=fixed,
	%showstringspaces=false,
	%showtabs=false,
	%keepspaces,
	%commentstyle=\color{red},
	%stringstyle=\color{orange},
	%extendedchars=true,
	%literate={á}{{\'a}}1
	%{é}{{\'e}}1
	%%{í}{{\'{\i}}}1
	%{í}{{\'i}}1
	%{ó}{{\'o}}1
	%{ú}{{\'u}}1
	%{Á}{{\'A}}1
	%{É}{{\'E}}1
	%{Í}{{\'I}}1
	%{Ó}{{\'O}}1
	%{Ú}{{\'U}}1
	%%{ü}{{\"u}}1
	%%{Ü}{{\"U}}1
	%{ñ}{{\~n}}1
	%{Ñ}{{\~N}}1
	%{¿}{{?``}}1
	%{¡}{{!``}}1,
	%keywordstyle=\color{blue}
%}




%atm general file listing with
%\lstinputlisting[style=file]{}


%%%%%%%%%%%%%%%%%%%%%%%%%		END CURRENT DOCUMENT CONF		%%%%%%%%%%%%%%%%%%%%%%%%%





\begin{document}

\pagestyle{empty}
\begin{center}
{\bf\Large UNIVERSITY OF SANTIAGO DE COMPOSTELA}

\vspace{0.5cm}
\includegraphics[width=5cm]{figuras/logo_usc.eps}

\vspace{0.5cm}
{\bf\large ESCOLA TÉCNICA SUPERIOR DE ENXEÑARÍA}

\vspace{2cm}
{\bf\LARGE Improvements in IDS: adding functionality to Wazuh}

%\vspace{0.5cm}
%{\bf\LARGE Subtítulo do Traballo de Fin de Grao}
\end{center}

\vspace{2cm}
\hspace{4cm}\begin{tabular}{l}
{\it\Large Autor:} \\
{\bf\Large Andrés Santiago Gómez Vidal} \\
~ \\
{\it\Large Directores:} \\
{\bf\Large Purificación Cariñena Amigo} \\
{\bf\Large Andrés Tarascó Acuña} \\
\end{tabular}

\vspace{2cm}
\begin{center}
{\bf\Large Computer Engineering Degree}

\vspace{0.5cm}
{\bf\large July 2019}

\vspace{0.5cm}
Final degree project presented at the Escola Técnica Superior de Enxeñaría of the University of Santiago de Compostela to obtain the Degree in Computer Engineering
\end{center}



\cleardoublepage
\pagestyle{plain}
\pagenumbering{roman}
\includegraphics[width=4cm]{figuras/logo_usc.eps}

\vspace{1cm}
{\bf Ms. Purificación Cariñena Amigo}, Professor Computing Science and Artificial Intelligence at the University of Santiago de Compostela and {\bf Mr. Andrés Tarascó Acuña}, Managing Director at Tarlogic Security S.L.

\vspace{1cm}
STATE:

\vspace{1cm}
That the present report entitled \textit{Improvements in IDS: adding functionality to Wazuh} written by \textbf{Andrés Santiago Gómez Vidal} in order to obtain the ECTS corresponding to the final degree project of the Computer Engineering degree was conducted under our direction in the department of Computer Science and Artificial Intelligence of the University of Santiago de Compostela.

\vspace{1cm}
For the purpose to be duly recorded, this document was signed in Santiago de Compostela on February TODO, 2019:

\vspace{2cm}
\begin{tabular}{lll}
The director, & The codirector, & The student, \\
~ \\
~ \\
~ \\
~ \\
~ \\
~ \\
~ \\
(Purificación Cariñena Amigo) & (Andrés Tarascó Acuña) & (Andrés Gómez Vidal)
\end{tabular}

 % paxina de certificación (optativa)

%\cleardoublepage
%\include{capitulos/agradecementos} % paxina de agradecementos (optativa) 

%\cleardoublepage
%\include{capitulos/resumo} % páxina de resumo (optativa) 

\cleardoublepage
\pagestyle{plain}
\tableofcontents
\listoffigures
\listoftables



% AGORA INCLUIMOS OS CAPÍTULOS. CAMBIAMOS A NUMERACIÓN E AS CABECEIRAS
\cleardoublepage
\pagenumbering{arabic}
\setcounter{page}{1}
\pagestyle{headings}
\chapter{Introduction}
%Introdución: composta por Obxectivos Xerais, Relación da Documentación que conforma a Memoria, Descrición do Sistema, Información Adicional de Interese (métodos, técnicas ou arquitecturas utilizadas, xustificación da súa elección, etc.).

%This project was made in collaboration with the cybersecurity company Tarlogic SL, even though I am not a member of Tarlogic and have never worked with them in the past.
%Due to my lack of professional experience in cybersecurity and the need to research in this project, the scope and planning of the work had to be rescheduled.
%Furthermore in this project there are no absolute constraints or objectives, as it was suggested as a case between investigation (with some coding) and cybersecurity auditing, so the scope can be reduced if the time remaining is too short.

\section{Motivation}

Cybersecurity nowadays is very complex: there are many sub-fields and expert tools and it could be argued that it is impossible to guarantee that any system is totally safe.
In this project we put ourselves in the shoes of a system administrator for an enterprise, that wants to improve the security by detecting intrusions in the servers he works on. This is key to decide which technologies and tools we choose in this project.
\linej
\linej
Cybersecurity measures can be applied in multiple layers of the system, each with different tools, objectives, advantages and costs.
In general the security of a system can be divided into the next parts:
\begin{enumerate}
	\item \textbf{Firewall}: Control the inbound/outbound connections, on the \textbf{network layer}. In our scenario its objective is to reduce the amount of inbound connections, reducing the chance of intrusion.
	\item \textbf{IPS}: Intrusion Prevention System to minimize the chance of intrusions, on the \textbf{network and host layers}. Provides active protection by actions.
	\item \textbf{IDS}: Intrusion Detection System to mitigate the damage of intrusions, on the \textbf{network and host layers}. Provides passive protection by alerts.
\end{enumerate}

\linej
The next table shows a \textbf{simplified} flow on how the information is processed by the security layers and methods.
For example an IDS can monitor the network connections, scanning the whole packet (header and payload) and filing a report if needed, but has worse performance than a firewall because they only scan the header of the packet and just opt to reject them\cite{firewall-ipds-ids_comparison}.
IDS fall into the SIEM category: software that manages information and events in real-time.

\begin{table}[H]
	\centering
	\caption{Simplification of the data flow}
\linej
	\begin{tabular}{|c|c|c|c|}
	\hline
		\textbf{Layer} & \textbf{Network} & \multicolumn{2}{c|}{\textbf{Network and Host}}\\ \hline
		\textbf{Method} & Firewall & IPS & IDS\\ \hline
		\textbf{Measures} & Prevent & Prevent & Mitigate\\ \hline
	\end{tabular}
\end{table}
$\xrightarrow{\makebox[\textwidth]{Direction of the data flow}}$

\linej
\linej
An IDS that focus on network monitoring is a NIDS.
They have become widely used over the past two decades because of the impressive capability to provide a granular view of what is happening on the network.
\linej
Attackers have grown used to NIDSs and have found ways to evade them, like\cite{libro_ossec}:
\begin{enumerate}
	\item Avoid using known patterns in their connections.
	\item Use encrypted connections.
	\item Send the data in pieces accross the network. This does not work against NIDSs that can reassemble them, at a greater computing cost. %session splicing and framentation attacks
	\item Denial of Service attacks: too much traffic overloads the NIDS, blinding it.
\end{enumerate}
\linej
We understand that NIDSs are useful in many situations, and there are many cases in this project where they could be used to complement an HIDS (Host-based IDS).
An HIDS can inspect the full stream of communications, making useless the techniques 2 and 3 in the previous example for evading NIDSs.
\linej
We focus on HIDS because we are more interested about detection at host level, rather than network.
Also IDS is less explored than IPS or firewalls and due to the advance in gathering and processing of data in the last years IDS has become much more viable and reliable.

\linej
\linej
IDSs are different from antivirus or antimalware because the first are systems \textbf{specialized} in detection and the latter usually focus on prevention, however prevention and detection are often meshed together because both are deeply related. There are some cases where a system specialized in detection offers some kind of mitigation functionality or one specialized in prevention offers some kind of detection functionality.

\linej
\linej
It is important to note that in cybersecurity the trend is for the attack to be created first and later some kind of measures, not necessarily by the same teams as they usually are specialized in each role. This means that defensive security that requires manual intervention often lags behind.
\linej
Nowadays there are lots of different attacks, so many that their detection could be almost impossible one by one, but most of them can be detected because they share patterns. If we can determine the patterns of an attack and code a way to detect them we can detect the threat. Some times it is easier to detect the attack and take measures after the intrusion has taken place.
\linej
\linej
IDSs work by analysing the key information available (programs, logs, network information, etc) to determine if there has been an intrusion in the system. The details of the process vary with each IDS but in general they work like an expert system:
\begin{itemize}
	\item The source of the data is the system.
	\item The alerts are set by certain rules when they match.
	\item Rules do not need to throw an alert and there can be dependencies, allowing a stateful approach and complex analysis without false positives (the main annoyance of IDSs).
\end{itemize}

\linej
There are two types of IDS, based on the detection mechanism:
\begin{itemize}
	\item Signature based: The IDS looks for specific data (signature), for example a string. This is often an efficient solution to known attacks, but is fundamentally useless against unknown attacks (attacks without a signature in the IDS database).
	\item Behaviour analysis: After a training period the IDS can detect when an event is rare (by probability) and correlate these suspicious occurrences to an intrusion.
\end{itemize}
In our case we take interest in the signature approach because it is much more used and behavior analysis is more fit for networks than for hosts.
Abnormal or suspicious behavior is called \textit{noise} in cybersecurity jargon.

\linej
\linej
OSSEC is an HIDS solution with detection based on rules and decoders. Both rules and decoders can be defined with numerous options and support dependencies and regular expressions.
\begin{itemize}
	\item The decoders format the data for the rules.
	\item There is a threat if the conditions of the rule are met.
\end{itemize}
\linej
\textbf{OSSEC} stands for \textbf{O}pen \textbf{S}ource HIDS \textbf{SEC}urity and is interesting for this project because\cite{ossec}\cite{wazuh_additional_functionality}:
\begin{itemize}
	\item \textbf{Widely Used}: OSSEC is a growing project, used by many different entities (ISPs, universities, governments, large corporate data centers) as their main HIDS solution. In addition to being deployed as an HIDS, it is commonly used strictly as a log analysis tool, monitoring and analyzing firewalls, IDSs, web servers and authentication logs.
	\item \textbf{Scalable}: Because it is an HIDS and it uses \textbf{agents}. Each monitored host can either install the agent or use an agentless agent\cite{agentless}\cite{ossec_agent}. Agentless agents are processes initiated from the OSSEC manager, which gather information from remote systems, and use any RPC method (e.g. SSH, SNMP, RDP, WMI).
	\item \textbf{Multi-platform}: GNU/Linux, Windows, Mac OS and Solaris. This is important because most professional services are on GNU/Linux or Windows, but it is important to note that rules can only work in one operating system.
	\item \textbf{Free}: OSSEC is a free software and will remain so in the future; you can redistribute it and/or modify it under the terms of the GNU General Public License (version 2) as published by the FSF -- Free Software Foundation.
	\item \textbf{Open source}: The code is open, so you can read, contribute and debug it all you want.
	\item \textbf{Rootkits detection}: This type of malware usually replaces or changes existing operating system components in order to alter the behavior of the system. Rootkits can hide other processes, files or network connections like itself.
	\item \textbf{File integrity monitoring}: To detect access or changes to sensitive data.
\end{itemize}
\linej
There are lots of alternatives to OSSEC for the scenario of a system administrator that wants to reinforce the security of the systems he is responsible for. There exist free of charge and paid solutions. Not all are pure IDSs and often they specialize in a field. For example the next table shows a comparison of the most important ICSs (Industrial Control Systems), which is a genetic type of control system that includes IDS, therefore it shows a comparison of OSSEC with similar software:
\begin{figure}[H]
  \centering
	\includegraphics[width=\textwidth]{figuras/comparison_ics.png}
	\caption{Comparison by attributes of the most important ICSs\cite{comparison_ics}}
\end{figure}

\linej
\linej
One of the problems of a comparison in a table like this is that it fails to show how much a tool excels or lacks in the features it shares with others, how easy it is to use and other factors that can help to choose the right tool. The most relevant alternative technologies to OSSEC for this project are\cite{comparison_tools}:
\begin{itemize}
	\item Sagan: An open source HIDS, but it only supports *nix operating systems (Linux, FreeBSD, OpenBSD, etc) and it lacks in features compared to OSSEC.
	\item YARA: It is not an IDS or IPS, it is just a tool that does pattern/string/signature matching, but it excels at it in performance, results and easiness to write the rules. It can be used to scan the \textbf{memory} for known patterns. YARA is being used widely in cybersecurity, for example by Avast, Kaspersky Lab, VirusTotal and McAfee Advanced Threat Defense\cite{who_is_using_yara}. We could build a system to use YARA to scan files but always combined with at least another tool, but we prefer to stick to a tested IDS.
\end{itemize}
\linej
Due to their popularity it is worth mentioning the next tools, even though they are only for network:
\begin{itemize}
	\item Bro: It is an open source IDS and supports only Linux, FreeBSD, and Mac OS.
	\item Snort: It is the most popular open source IDS/IPS, but can be expensive in processing power.
	\item Suricata: Another open source IDS/IPS solution. It provides hardware acceleration and multi-threading to improve the scanning speed.
\end{itemize}
\linej
Most of the attributes in the previous comparison are not relevant for our work.
We chose OSSEC because of the problems found on the alternatives.
Also OSSEC offers a reliable way to use an already developed and thoroughly tested IDS, which we can enhance to our needs without much work.
To even ease more this we will use Wazuh, a fork of OSSEC.

\section{Objectives}
Quality is valued more than quantity in this project. Therefore anything will be reworked or discarded if it does not fully satisfy the student, Tarlogic or the professor.
\linej
\linej
The main objective is to improve intrusion detection in IDS. This can be accomplished in several ways:
\begin{itemize}
	\item Adding or changing functionality of an already existing technology.
		\subitem Coding on core or additions.
		\subitem Configuration or input of the program.
\end{itemize}
\linej
In this project the focus is on the configuration, particularly of rules to detect certain attacks. It is necessary to fully understand the attacks first to code its detection, therefore preparing the attacks also will need a fair amount of time.
It is important to explain the attacks and their detection clearly, in order to make this work useful for anyone else and ease any changes.

\section{Structure of this document}
%TODO Structure of this document
This document has TODO chapters:
\begin{itemize}
	\item In \textbf{chapter 1} 
	\item In \textbf{chapter 2} 
	\item In \textbf{chapter 3} 
	\item In \textbf{chapter 4} 
	\item In \textbf{chapter 5} 
	\item In \textbf{chapter 6} 
	\item In \textbf{chapter 7} 
\end{itemize}



\cleardoublepage
\chapter{Requirements}
\chapter{Requirements}

%Especificación de requisitos: debe indicarse, polo miúdo, a especificación do 
%Sistema, xunto coa información que este debe almacenar e as interfaces con outros 
%Sistemas, sexan hardware ou software, e outros requisitos (rendemento, seguridade, 
%etc).


The requirement specification is a full description of the software the project is to develop.
\linej
PMBOK\cite{pmbok} states that requirements are conditions or capabilities that a product must meet to satisfy the contract.
The requirements expose the needs of the client, which have to be accomplished to finish the project successfully.
In this project the requirements will be fullfilled in multiple stages along the project.
\linej
Note that the client in this case is Tarlogic even if the product is a contribution to an open source project.

\linej
\linej
This specification contains:
\begin{itemize}
	\item \textbf{Use cases}: Functionalities that the software will provide.
	\item \textbf{Requirements}: Depending of their type they can describe features, data, relations, properties or any details necessary to explain the system without ambiguity, in a way it can be easily understood.
\end{itemize}

In this project the functional requirements are not included because they can be considered a redundant version of the use cases, because both describe the same functionalities.
Uses cases were chosen over functional requirements because they were considered to be easier to understand and have greater detail. If this project had the need of a very complex requirement specification it would be interesting to have both, as each could help to understand the other better, but in this project the specification should be quite simple.






\section{Use cases}
A use case is a description of all the ways an end-user wants to ``use'' a system. These ``uses'' are like requests of the system, and use cases describe what that system does in response to such requests. In other words, use cases describe the conversation between a system and its user(s), known as actors. Although the system is usually automated (such as an Order system), use cases also apply to equipment, devices, or business processes.\cite{use_case_definition}


\subsection{Use cases actors}
The actors are entities external to the system that interact with it. They can be other systems, persons or even time.

%\subsection{Use cases description}
\subsection{Use cases list}

\section{Requirements analysis}

\subsection{Non functional requirements}

\subsection{Functional requirements}
As mentioned before these are omited because of the redundancy with use cases.

\subsection{Domain requirements}


\cleardoublepage
\chapter{Technologies and tools}
\section{Introduction}
Wazuh is a fork of OSSEC. It adds a RESTFul API, has a more updated ruleset and is easier to install (providing ELK over OSSEC).
\begin{figure}[H]
  \centering
	\includegraphics[width=.6\textwidth]{figuras/wazuh_stack.png}
	\caption{The different parts of Wazuh\cite{wazuh_stack}}
\end{figure}
\linej
The most interesting qualities of Wazuh for this project are\cite{wazuh_index}\cite{wazuh_documentation}: %necessary redundancy
\begin{itemize}
	\item \textbf{Rootkits detection}: Rootkits are commonly used after an attack has succeeded to use the computer of the victim leaving no traces.
	\item \textbf{File integrity monitoring}: It can provide detection of intrusions by identifying changes in content, permissions, ownership, and attributes on the monitored files. It can be used to comply with GDPR (General Data Protection Regulation).
	\item \textbf{Scalability and multi-platform}: This means that the work on this project could really be used in real work environments.
	\item \textbf{Configuration management}: The configuration is managed by the Wazuh server (Wazuh manager) and the agents can be grouped, allowing custom, group or global gathering and detection for each agent.
	\item \textbf{Multiple sources of data}: The scanned data can be from logs, output of commands or databases.
	\item \textbf{Active response}: Automated remediation to security violations and threats, to mitigate more the possible damage. For example to stop the Internet connection to isolate a compromised system.
	\item \textbf{Improved ruleset}: It is the combination of rules and decoders. Having this ruleset out of the box reduces the workload of this project. It can also serve as reference and complement some of the rules and decoders that this project intends to work on.
	\item \textbf{Open source, free and easy to contribute to}: This is optional but nice, as it offers a chance to an inexperienced student to contribute in a real and useful project. The project is hosted on Github and Google Groups. In this project the contribution would be to the ruleset\cite{wazuh_ruleset} and not to the core of Wazuh\cite{wazuh} or the documentation\cite{wazuh_documentation2}.
		%TODO decidir si hacer contribución al ruleset y si cambiar/dejar esto y otras menciones similares
\end{itemize}
\linej
The RESTFul API interacts using OSSEC commands and would be interesting if this project were related to a tool issuing queries to Wazuh, but this is not the case. Anyway it is still something valuable to have as these kind of tools are very common nowadays.

\linej
\linej
Wazuh provides support and integration with multiple important tools and technologies:
\begin{itemize}
	\item Docker container for OSSEC: An ossec-server image with the ability to separate the ossec configuration/data from the container.
	\item Puppet and Ansible: For massive deployment. This can be very helpful to setup a big environment mostly because even being no need to put configuration files in the agents for Wazuh often is necessary to configure other things and the process of registering agents can be tedious manually.
	\item Network IDS integration: Gives the option to use OwlH and integrate Suricata and Bro to generate alerts in Wazuh.
	%\item Splunk infrastructure: Comprised of a Splunk Enterprise instance as indexer and a Splunk Forwarder node, as well as the Wazuh app for Splunk.
	\item VirusTotal: A free virus, malware and URL online scanning service that combines more than 40 antivirus solutions.
	\item OSQuery: Osquery can be used to expose an operating system as a high-performance relational database. This allows you to write SQL-based queries to explore operating system data.
\end{itemize}
\linej
The use of these depends on the scenario, but we only take interest in VirusTotal and Network integration. They can work as secondary detection methods for the most critical or complicated cases.

\section{Wazuh architecture}

A basic Wazuh setup has the next components\cite{wazuh_architecture}:
\begin{itemize}
	\item Wazuh server: Runs the Wazuh manager, API and Filebeat (Filebeat is only necessary in distributed architecture). It collects and analyzes data from deployed agents.
	\item ELK stack: It reads, parses, indexes, and stores alert data generated by the Wazuh server. The ELK stack is flexible, highly configurable and very used in big data.
	\item Wazuh agent: Runs on the monitored host, collecting system log and configuration data and detecting intrusions and anomalies. It communicates with the Wazuh server, to which it forwards collected data for further analysis.
\end{itemize}
\linej
The main difference with the architecture of OSSEC is the ELK stack, because OSSEC leaves the choice of tools to the user. ELK stands for the combination of:
\begin{itemize}
	\item Elasticsearch: Gets the data and allows search queries and analysis.
	\item Logstash: Transforms the data to the desired format. This step can make alike data from different log and output formats, trivializing the decoders work.
	\item Kibana: Shows the data in a web browser, with graphs and options like grouping and time interval. This is often easier than to write commands to scan the OSSEC log in the Wazuh server, as the data of interest tends to stay the same.
\end{itemize}
\linej
\label{singlehost}
There are two possible architectures for this setup: having the ELK stack in the same machine that the Wazuh server (single host) or in a separated one (distributed). Each has advantages and disadvantages and in this project we will use the single host because in our case there are no constraints and is easier to set up and is more efficient.
\begin{figure}[H]
  \centering
	\includegraphics[width=\textwidth]{figuras/wazuh_singlehost.png}
	\caption{Single host architecture}
\end{figure}

\begin{figure}[H]
  \centering
	\includegraphics[width=\textwidth]{figuras/wazuh_distributed.png}
	\caption{Distributed architecture}
\end{figure}
\linej
To understand better the communications and data flow in Wazuh we will now get into more detail on the process\cite{wazuh_architecture2}\cite{wazuh_data_flow}.
\linej
\linej
Wazuh agents use the OSSEC message protocol to send collected events to the Wazuh server over port 1514 (UDP or TCP). The Wazuh server then decodes and rule-checks the received events with the analysis engine. Events that trip a rule are augmented with alert data such as rule id and rule name. The Wazuh message protocol uses a 192-bit Blowfish encryption with a full 16-round implementation, or AES encryption with 128 bits per block and 256-bit keys.
\linej
Logstash formats the incoming data and optionally enriches it with GeoIP information before sending it to Elasticsearch (port 9200/TCP). Once the data is indexed into Elasticsearch, Kibana (port 5601/TCP) is used to mine and visualize the information.
\linej
The Wazuh App runs inside Kibana constantly querying the RESTful API (port 55000/TCP on the Wazuh manager) in order to display configuration and status related information of the server and agents, as well to restart agents when desired. This communication is encrypted with TLS and authenticated with username and password.

\begin{figure}[H]
  \centering
	\makebox[\textwidth][c]{\includegraphics[width=1.2\textwidth]{figuras/wazuh_data_flow1.png}}
	\caption{Communications and data flow}
\end{figure}
\linej
Both alerts and non-alert events are stored in files on the Wazuh server in addition to being sent to Elasticsearch. These files can be written in JSON format and/or in plain text format (.log, with no decoded fields but more compact). These files are daily compressed and signed using MD5 and SHA1 checksums.
There is also the option to store the alerts in a database if OSSEC is compiled with database support (for example MySQL or PostgreSQL)\cite{libro_ossec}.

\section{Rules and decoders}
They constitute the main part of this project and they can be used to detect application or system errors, misconfigurations, attempted and/or successful malicious activities, policy violations and a variety of other security and operational issues\cite{wazuh_index}. Wazuh is quite helpful with the features and documentation of the ruleset and in this project the already existing rules and decoders were a great help as examples.
\linej
\linej
Most of the time we forget about the rest of the details of the previous figure, focusing just on the inmediate elements to the ruleset:
\begin{figure}[H]
  \centering
	\includegraphics[width=.4\textwidth]{figuras/wazuh_data_flow1_zoom.png}
\end{figure}
\linej
When an event is received in the manager first it gets decoded. The process of predecoding is very simple and is meant to extract only static information from well-known fi elds of an event.
Decoding is used for extracting the data that is not static. This means later we can make rules that process this information. It can also be useful to differentiate very similar events, to be later processed by different rules.
\begin{figure}[H]
  \centering
	\includegraphics[width=\textwidth]{figuras/Event_Flow.png}
	\caption{Event Flow Diagram}
\end{figure}
\linej
At least a rule in the hierarchy has to be related to the decoder of the event for it to be able to trigger an alert. An Alert is generated if the conditions of the rule are true an the level of the rule is greater than 1.
\linej
When alerts are triggered they are recorded into the log, and also can be stored in a database, send mails and execute commands\cite{libro_ossec}.

\linej
\linej
There are two tipes of rules\cite{libro_ossec}:
\begin{itemize}
	\item \textbf{Atomic}: They are based on simple events, without any correlation. They are by far the most used.
	\item \textbf{Composite}: Those with multiple events. They have a time window and a number of times the rule has to be true before triggering the alert. They can group multiple atomic rules, all of which have to be true for the composite rule to be true.
\end{itemize}
\linej
The level parameter of the rule marks the severity of the alert. These are some examples\cite{libro_ossec}:
\begin{itemize}
	\item 0: Ignored, no action taken. Primarily used to avoid false positives. These rules are scanned before all the others and include events with no security relevance.
	\item 1: They are like 0, but for composite rules. Atomic rules for composite rules need to be of level 1 to be used by composite rules and not generate any alert on their own.
	\item 2: System low priority notifications or status messages that have no security relevance.
	\item 3: Successful/authorized events. Successful login attempts, firewall allow events, etc.
	\item 4: System low priority errors. Errors related to bad configurations or unused devices/applications.
	\item 5: User-generated errors. Missed passwords, denied actions, etc. These messages typically have no security relevance.
	\item 6: Low relevance attacks. Indicate a worm or a virus that provide no threat to the system such as a Windows worm attacking a Linux server. They also include frequently triggered IDS events and common error events.
\end{itemize}
In this project the same level is used for all the rules that are meant to trigger alerts, to keep it simple.

\linej
\linej
Rules can be added in \textit{/var/ossec/etc/rules/} and decoders in \textit{/var/ossec/etc/decoders/} without any issue, but to change the already existing ones in \textit{/var/ossec/ruleset/rules/} or \textit{/var/ossec/ruleset/decoders/} is a bad idea because the next changes in those files from updates would overwrite them.
%The solution is to copy the code (actually only the id is needed) of the existing item to the folder where we can add new ones, make the desired changes add \textit{overwrite=``yes''}\cite{wazuh_custom}.
\linej
\linej
As mentioned before Wazuh adds its own ruleset over the one provided by the OSSEC project. The next table shows about 20\% of the combined ruleset that Wazuh uses, where ``Out of the box'' means that the source was the OSSEC project.
\begin{figure}[H]
  \centering
	\makebox[\textwidth][c]{\includegraphics[width=1.1\textwidth]{figuras/wazuh_ruleset.png}}
	\caption{Portion of the ruleset used by Wazuh\cite{wazuh_ossec_ruleset}}
\end{figure}
\linej
Wazuh provides a way to manually test how an event is decoded and if an alert is generated with the tool \textit{/var/ossec/bin/ossec-logtest}\cite{wazuh_testing}, which is very useful for debugging.
To use it you only need to introduce the data as it would be received by the Wazuh manager.
Is possible to show which rules are tried and which trigger an alert for each event.
This tools does not need a restart of the wazuh-manager service whenever changes want to be tested because it reads the configuration directly.
\linej
But is also worth to mention that some times it can be misleading because it does not work in the same way as the manager.
For example the logtest may show that the log matches a certain rule but actually it has matched a previous one silently.
\linej
\linej
For example for this input:
\linej
\includegraphics[width=\textwidth]{figuras/ossec-logtest_input.png}
\linej
We get the next output:
\begin{figure}[H]
  \centering
	\includegraphics[width=\textwidth]{figuras/ossec-logtest_output.png}
	\caption{Example of output for ossec-logtest}
\end{figure}
\linej
%TODO cambiar currently in version
After version 3.0.0 (we are currently in 3.9) Wazuh incorporates an integrated decoder for JSON logs enabling the extraction of data from any source in this format. This can be very useful in many situations, for example trivializing the generation of alerts for programs reporting in JSON, without the need for a decoder for each one\cite{wazuh_json}.
\linej
\linej
Another interesting feature is to check if a field extracted during the decoding phase is in a CDB list (constant database). The main use case of this feature is to create a white/black list of users, IPs or domain names.\cite{wazuh_cdb}.

Usually this chapter would be ''Design and Implementation``, but it was not needed because there was no serious software development during this project, as explained before.
Still a description of the components and their communication in the system is needed to fully understand it.

\linej
\linej
This project was run under a GNU/Linux distribution in one of the personal computers of the student.
Said computer has 20GB of RAM, an \textit{i5-2500k} processor and about 500GB of free disk storage for this project.

\section{Virtual machines}
VirtualBox was chosen as the host program of the virtual machines because it was the one the student had the most experience with.
\linej
This laboratory consist of multiple virtual machines, that represent:
\begin{itemize}
	\item An enterprise network of Windows computers, managed by Windows' Active Directory:
		\begin{itemize}
			\item Windows Server 2019, as Domain Controller of the Active Directory.
			\item Windows Server 2019, as a file server.
			\item Windows 10, as a basic workstation.
		\end{itemize}
	\item The Wazuh server: A CentOS 7. As mentioned before\ref{singlehost}, we are using a single server to host both the Wazuh manager and the ELK stack.
	\item An offensive security box: In this case with Kali Linux. This is used to access the Windows boxes in some of the attacks.
		%TODO version Kali
\end{itemize}

\begin{figure}[H]
  \centering
	\includegraphics[width=\textwidth]{figuras/virtual_machines.jpg}
	\caption{Virtual machines in the project}
\end{figure}
\linej
Every machine has two network interfaces, one for the internal network (10.0.3.0) and another for accessing the internet connection of the host.
Each of the Windows boxes have a Wazuh agent installed, that reports to their manager (server) in the CentOS box.
All the ruleset changes and log processing was done directly on the CentOS machine.
\linej
\linej
Most of the time only the DC and the CentOS machine were powered on, but when all of them were being used at the same time they consumed about 12GB of RAM. Leavin aside booting, they did not affect performance in a perceptible way, either one to each other or to the host.
\linej
Snapshots were taken when significant changes were made, like relevant installation or configuration. Then the virtual machines with their snapshots were copied to a couple of external disks as backup.

%TODO
\section{For development and configuration}
\begin{itemize}
	\item Wazuh: Wazuh\cite{wazuh} is the core of this project and some of the configuration of the system had to be done with it. In this project ruleset files were created to detect the targeted threats.
	\item Sysmon: Sysmon\cite{sysmon} reports events based on its configuration, which in most cases give more insight of the status of the system than normal Windows events.
	\item Powershell: Some basic scripting\cite{memoria_github} was done in powershell to mimic attacks and to perform a particular detection with a remote command.
\end{itemize}

\section{For pentesting}
\begin{itemize}
	\item Powershell scripts: Third-party tools writen in powershell were used to mimic attacks.
	\item Powershell and Windows builtins: Some harmless powershell commands were used in combination with Windows builtins in order to extract useful data for an attacker.
	\item Linux shell programs in GNU/Linux: They were used in order to fully understand how some of the security was implemented in the Windows systems. For example their network share was first tested with the Samba command \textit{smbclient}.
	\item Metasploit: It was used as framework\cite{metasploit} for getting information and to exploit security vulnerabilities, mimicking an attacker.
	\item Mimikatz: Mimikatz\cite{mimikatz_github} was one of the most used tools for extracting data and gaining privileges.
	\item Other third-party tools: Other programs were used to dump data from memory, in order to have assure the identification of attack patterns from different sources.
\end{itemize}

\section{For processing logs}
\begin{itemize}
	\item The Kibana plugin for Wazuh: It provides an easy and fashionable way to show the triggered alerts with a web browser. It was only used at first, before it felt too slow and shallow.
	\item Linux shell programs in GNU/Linux: The most used were Grep, AWK and Tmux. They were used to process the logs of Wazuh for most of the project. These logs include the received events and the triggered alerts. They were a good fit because they are easy to write commands on, they are fast and the queries needed for this project were quite simple. Grep was used for finding events with certain patterns. AWK to parse logs for easier comprehension. Tmux is a terminal multiplexer, which is basically a program that controls a bunch of terminals, and was used to manage the shells and to find and copy strings in them, similar to an Integrated Development Environment (IDE).
	\item The Windows Event Viewer: To troubleshoot inspecting the logs generated from Sysmon. Some times under heavy load (particularly after booting) the agent could not send the data to the manager without a significant delay of a couple of minutes. From the manager it made it seem like none of the exepect events were generated.
\end{itemize}

\section{For the documentation}
\begin{itemize}
	\item Git: Was used to store the memory and manage its changes, through a Github repository\cite{memoria_github}.
	\item Vim + \LaTeX + Latexmk: Vim was used as the text editor to write all the documentation and most of the ruleset and scripts. This was easier for me than using other editos because I have a fair amount of customization for Vim, in my general purpose dotfiles\cite{andresgomezvidal_gitlab}. \LaTeX was the medium the memory was writen in, with Latexmk for its compilation.
	\item Draw.io: A web application for general purpose drawing\cite{drawio}. It was used for making diagrams in this project. If these diagrams were to be more complex this probably would have been replaced by other tool (at least for those cases).
	\item Simplescreenrecorder: Is a GNU/Linux program. It was used to record videos during critical tasks using the virtual machines. These serve just as a way to assure the student these tasks were done exactly as he remembers them, or how to reproduce them.
	%TODO comentar valores calidad y que una hora son $\sim$110MB
	\item Linux shell programs: Some simple commands to make sure no minor elements are forgotten. For example that all references and images are used, and that any relevant acronym are explained in the glossary.
	\item Aspell: Is a spell checker\cite{aspell}, this is a program for reviewing that words are writen correctly in the specified language. This program has a mode for \LaTeX that ignores its commands without the need of a special dictionary.
\end{itemize}

\section{Other technologies and tools}


\cleardoublepage
\chapter{Project management}
A project is temporary in that it has a defined beginning and end in time, and therefore defined scope and resources.
And a project is unique in that it is not a routine operation, but a specific set of operations designed to accomplish a singular goal.
\linej
Project management, is the application of knowledge, skills, tools, and techniques to project activities to meet the project requirements\cite{pmi_project_management}.
It is important to note that the actions on each area of the project may affect other areas, increasing the difficulty of the management.

\section{Scope management}
The management of the scope of the project has the necessary processes to guarantee that the objectives are met.
The scope management allows the project to start focused in what really matters, not losing time in irrelevant details or desirable additions that we can not implementent, by identifying and describing the necessary tasks.


\subsection{Description of the scope}
This project tries to improve the detection of intrusions on the already existing HIDS Wazuh. This kind of objective can be accomplished by very different approaches, because the software to work on can be used in many scenarios, is very related to other software and is in active development.
%TODO
\linej
Though some increments can be considered difficult due to the amount of new technologies and tools there should be no problem to met the basic objectives because we have the freedom to adapt the scope at any time and there is more than enough time for the essential parts of the project.

\subsection{Acceptation criteria}
In order to the product to be accepted the essential requeriments need to have been accomplished before the time limit of the project.

\subsection{Increments}
The essential increments to the project are:
\begin{itemize}
	\item Increment 1: \IncrementoUno.
	\item Increment 2: \IncrementoDos. By itself is not really a must have, but it could make a different for certain attacks and the more data the better.
	\item Increment 3: \IncrementoTres.
\end{itemize}
\linej
These were chosen because we think Wazuh needs tangible security measures against common threats more than anything else at the moment. In other words, we reckon these points are were Wazuh lacks the most right now.
\linej
\linej
The rest of the increments are considered optional and can be removed if there is not enough time left. The order is based on the estimation of the relevance of the increment for Wazuh and our project. This means for example having in mind the time estimation for the increment without stretching.
\begin{itemize}
	\item Increment 4: \IncrementoCuatro. This is considered a very important increment because we think it could be a selling point for some enterprises, that probably do not want the same level of security for all their computers and the time to set it up (or at least from scratch). There is a chance that something like this already exists, in which case the investigation at the start of the increment should find it.
	\item Increment 5: \IncrementoCinco. Tarlogic stated that they would like to have this increment specifically.
	\item Increment 6: \IncrementoSeis. This could have more or less the same impact as increment 1 for some clients, but Tarlogic was more interested in Windows and Wazuh seems to be more oriented towards GNU/Linux.
	\item Increment 7: \IncrementoSiete. The problem of this increment is that VirusTotal's public API key has more limited features and has a 4 requests/minute limitation\cite{virustotal_faq}. We asume the use of a public API key because it would fit the profile of a client using Wazuh, which has no charge. Also the exploration on this increment could not really be considered more than a patch to Wazuh, without really improving it, but still it would be an effective workaround for the problems we can not solve right now with only Wazuh.
\end{itemize}


\subsection{Products of the project}
At the end of the project the next elements will be delivered:
%TODO
\begin{itemize}
	\item 
	\item 
	\item 
\end{itemize}



\subsection{Exclusions} 	%necessary redundancy with introduction?
As in any project of this kind we had to leave some ideas behind.

\linej
\linej
For example an interesting way to to take advantage from IDS is to set up a honeypot (a false server just to be compromised) and learn from the intrusions suffered, improving the defenses (firewall, IPS and IDS) for the real servers. There are some honeypot implementations that automate (for example with machine learning) the generation of rules for certain IDSs, but is not yet a trend because there are problems\cite{snort_learning}\cite{honeypot_weka_learning}\cite{honeypot_ossec_trees}\cite{snort_honeypot}:
\begin{itemize}
	\item Experienced attackers have learned to avoid honeypots, because they are easy to identify due to the low security they have.
	\item Is not trivial to automate correctly the defense based on the information of the system, because its state can be very complex (for example due to more than one attack at the same time).
\end{itemize}
\linej
This automation would be a great solution to the need to update manually the rules and depending of the case it could even protect against day-zero vulnerabilities. Despite being interesting this was not even included as a possible increment because the complexity of the task. If this were included probably it would not have ended well because is something that even experts in cybersecurity have some trouble with at the moment.

\linej
\linej
There is always the risk of the intrusion disabling the security of the system. This is more or less the same problem that cybersecurity has in any scenario and there is no way to guarantee that it will not happen. In this case the attacker would have to somehow not be detected or cut the IDS before it sends the alert but in a way that is not suspicious (for example shutting it down completely would be obvious for a central manager). So our approach is to trust the IDS and work on improving the detection of known attacks instead of the worst case scenario. If there was enough time we could have considered finding a solution for this problem.

\linej
\linej
Exploring a HIDS with behaviour analysis was also considered but rejected because it is more fit for a network approach. Still is a shame there is no enough time to explore IDS based on behavior analysis, because their protection against much zero-day attacks.

\linej
\linej
We focus on a host approach, leaving aside most of the detection capabilities for the network. This means less detection, a lower detection rate, less options to improve the detection process and later and worse performance in the analysis of network traffic. Having chosen to focus on HIDS the best way to have also a good NIDS process would be the use of a NIDS along with our HIDS.
\linej
Wazuh offers this kind of integration with Bro and Suricata and probably it would be possible to extend it to other NIDSs like Snort, but yet again we had to choose and this was not a priority at the moment.



\subsection{Restrictions}
Leaving aside the time constraint of about 417 hours, the two main factors to decide what improvements to choose for this project are a student without experience in proffesional cybersecurity and that we want some kind of inmediate results from this project. This is why while we could just have a pure research project (for example machine learning with IDS) instead we opted for a more traditional and safe approach. This is the reason why most of the increments were optional (due to the high probability of initial scope being too ambitious) but the first increments are considered vital to the project.
\linej
\linej
A minor restriction is to deliver correctly all the products of the project before the presentation date.


\section{Risk management}

\subsection{Risk metrics}

\begin{table}[H]
	\centering
	\begin{tabular}{|l|l|}
		\hline
		\rowcolor{gray!30}
		Chances of the risk happening & Probability \\ \hline
		$\geq$80\% & \cellcolor{red!60}High\\ \hline
		Between 30\% and 80\% & \cellcolor{yellow!40}Medium\\ \hline
		$\leq$30\% & \cellcolor{green!60}Low\\ \hline
	\end{tabular}
	\caption{Probability classification of risks}
\end{table}


\begin{table}[H]
	\centering
	\begin{tabular}{|l|l|}
		\hline
		\rowcolor{gray!30}
		Resource in Place / Effort / Cost & Impact \\ \hline
		$\geq$20\% & \cellcolor{red!60}High\\ \hline
		Between 10\% and 20\% & \cellcolor{yellow!40}Medium\\ \hline
		$\leq$10\% & \cellcolor{green!60}Low\\ \hline
	\end{tabular}
	\caption{Impact classification of risks}
\end{table}


\begin{table}[H]
	\centering
	\begin{tabular}{|c|c|c|c|c|}
	\hline
		\multicolumn{2}{|c|}{\multirow{2}{*}{\large\textbf{Exposition}}} & \multicolumn{3}{c|}{Probability}\\
		\multicolumn{2}{|c|}{} & \cellcolor{gray!15}\textbf{High} & \cellcolor{gray!15}\textbf{Medium} & \cellcolor{gray!15}\textbf{Low}\\ \hline %\cline{3-5}
		\multirow{3}{*}{Impact} & \cellcolor{gray!15}\textbf{High} & \cellcolor{red!60}High & \cellcolor{red!60}High & \cellcolor{yellow!40}Medium\\
		& \cellcolor{gray!15}\textbf{Medium} & \cellcolor{red!60}High & \cellcolor{yellow!40}Medium & \cellcolor{green!60}Low\\
		& \cellcolor{gray!15}\textbf{Low} & \cellcolor{yellow!40}Medium & \cellcolor{green!60}Low & \cellcolor{green!60}Low\\ \hline
	\end{tabular}
	\caption{Method of calculation of Exposition based of Probability and Impact}
\end{table}

\subsection{Risk types}

\subsection{Risk identification}

\begin{table}[H]
	\caption{Project risks}
	\begin{tabularx}{\textwidth}{|l|X|}
		\hline
		\rowcolor{gray!30}
		Identifier & Name \\ \hline
		%R-000 & The scope specified is too big\\ \hline
		R-001 & Optimist planning, ``best case'' (instead of a realistic ``expected case'')\\ \hline
		R-002 & Bad requirement specification\\ \hline
		R-003 & Design errors\\ \hline

		R-004 & Lack of key information from sources\\ \hline 		%articles, documentation, manuals
		R-005 & Lack of feedback or support from the security consultants of Tarlogic\\ \hline 		%eg: holidays or sickness
		R-006 & The learning curve of some technologies is larger than expected\\ \hline
		R-007 & The unexplained parts of the project take more time than expected\\ \hline

		R-008 & Cannot access source material\\ \hline 		%is down, is too old
		R-009 & Unexpected changes to any of the APIs used in the project\\ \hline

		R-010 & Loss of work\\ \hline 	%use git, online repo, local backups for virtual machines?
		R-011 & Wrong management of the project's configuration\\ \hline 	%wrong baseline, wrong identification of the configuration elements, it takes more time than expected, wrong use of the tools, too much time between commits
		R-012 & A delay in one task leads to cascading delays in the dependent tasks\\ \hline

		R-013 & Unnecessary work\\ \hline 		%
		R-014 & The quality of the product is not enough\\ \hline 	%redo
		R-015 & Sickness or overwork\\ \hline
		R-016 & Performance issues\\ \hline 	%redo rules?, get a better machine?
	\end{tabularx}
\end{table}





\subsection{Risk analysis}


\begin{table}[H]
	\begin{tabularx}{\textwidth}{|l|X|}
		\hline
		\rowcolor{gray!30}
		Identifier & \textbf{R-000} \\ \hline
		Name & Bla \\ \hline
		Description & Bla bla bla bla bla bla bla bla bla bla bla bla bla Bla bla bla bla bla bla bla bla bla bla bla bla bla Bla bla bla bla bla bla bla bla bla bla bla bla bla \\ \hline
		Probability & Low , Medium , High \\ \hline
		Impact & Low , Medium , High \\ \hline
		Exposition & Low , Medium , High \\ \hline
		Indicator & Bla bla bla bla bla bla bla bla bla bla bla bla bla \\ \hline
	\end{tabularx}
\end{table}

\begin{table}[H]
	\begin{tabularx}{\textwidth}{|l|X|}
		\hline
		\rowcolor{gray!30}
		Identifier & \textbf{R-001} \\ \hline
		Name & Optimist planning, ``best case'' (instead of a realistic ``expected case'')\\ \hline
		Description & An optimistic planning at the start of the project does not take into account problems or delays, and so it does not allocate time for them, leading to cascading delays if they happen. \\ \hline
		Probability & Medium\\ \hline
		Impact &  High\\ \hline
		Exposition &  High\\ \hline
		Indicator & There are 3 consecutive delays, after the beginning of the project.\\ \hline
	\end{tabularx}
\end{table}



\begin{table}[H]
	\begin{tabularx}{\textwidth}{|l|X|}
		\hline
		\rowcolor{gray!30}
		Identifier & \textbf{R-002} \\ \hline
		Name & Bad requirement specification\\ \hline
		Description & The requirements specified at the beginning of the project are not specific enough, are not needed or there are new requirements after the beginning of the project. \\ \hline
		Probability & High\\ \hline
		Impact &  High\\ \hline
		Exposition &  High\\ \hline
		Indicator & There are 3 changes in the requirements specification.\\ \hline
	\end{tabularx}
\end{table}



\begin{table}[H]
	\begin{tabularx}{\textwidth}{|l|X|}
		\hline
		\rowcolor{gray!30}
		Identifier & \textbf{R-003} \\ \hline
		Name & Design errors\\ \hline
		Description & A design is not enough or is incorrect, needing a re-design and probably changes in the next steps it was used. \\ \hline
		Probability & Low\\ \hline
		Impact &  Medium\\ \hline
		Exposition & Medium\\ \hline
		Indicator & There are 3 designs that need rework.\\ \hline
	\end{tabularx}
\end{table}



\begin{table}[H]
	\begin{tabularx}{\textwidth}{|l|X|}
		\hline
		\rowcolor{gray!30}
		Identifier & \textbf{R-004} \\ \hline
		Name & Lack of key information from sources\\ \hline
		Description & Not having key information from articles, documentation or manuals can result in unexpected delays, added difficulty or the need to rework completely the functionality.\\ \hline
		Probability & Medium\\ \hline
		Impact &  High\\ \hline
		Exposition &  High\\ \hline
		Indicator & The duration of the study of the attack and the needed tools takes 50\% than expected. \\ \hline
	\end{tabularx}
\end{table}



\begin{table}[H]
	\begin{tabularx}{\textwidth}{|l|X|}
		\hline
		\rowcolor{gray!30}
		Identifier & \textbf{R-005} \\ \hline
		Name & Lack of feedback or support from the security consultants of Tarlogic\\ \hline
		Description & Because I do not know enough of some technical aspects of cibersecurity to solve all the problems in this by myself in time, Tarlogic has promised to help (in a tutoring way) if a problem arises.
		This help could be critical to solve or get around some of the most complex problems, which probably happen to be critical points, needing to be dealt with to continue working on that stage.\\ \hline
		Probability & Medium\\ \hline
		Impact &  High\\ \hline
		Exposition &  High\\ \hline
		Indicator & A simple technical question takes more than 2 working days to be answered or a complex question takes more than 7 working days.\\ \hline
	\end{tabularx}
\end{table}



\begin{table}[H]
	\begin{tabularx}{\textwidth}{|l|X|}
		\hline
		\rowcolor{gray!30}
		Identifier & \textbf{R-006} \\ \hline
		Name & The learning curve of some technologies is larger than expected\\ \hline
		Description & This is a critical need because not having enough knowledge can result in unexpected delays, added difficulty or the need to rework completely the functionality.\\ \hline
		Probability & Medium\\ \hline
		Impact &  Medium\\ \hline
		Exposition &  Medium\\ \hline
		Indicator & The duration of the study of the technologies takes 50\% than expected. \\ \hline
	\end{tabularx}
\end{table}

\begin{table}[H]
	\begin{tabularx}{\textwidth}{|l|X|}
		\hline
		\rowcolor{gray!30}
		Identifier & \textbf{R-007} \\ \hline
		Name & The unexplained parts of the project take more time than expected\\ \hline
		Description & There is not enough specification on what a tasks implies or not enough planning. This means that a part of the project is not understood as it should, and the work done is not what was expected or is not enough, needing more time to finish. \\ \hline
		Probability & Low\\ \hline
		Impact &  High\\ \hline
		Exposition &  Medium\\ \hline
		Indicator & A task takes 15\% more time than expected and when the causes are investigated it is revealed that there were ambiguous descriptions or planning.\\ \hline
	\end{tabularx}
\end{table}
\begin{table}[H]
	\begin{tabularx}{\textwidth}{|l|X|}
		\hline
		\rowcolor{gray!30}
		Identifier & \textbf{R-008} \\ \hline
		Name & Cannot access source material\\ \hline
		Description & All or part of the source material can not be accessed, probably because the only host of the resource is down. In some cases this could mean a delay in a critical task, cascading in other delays and delaying the project for a period unknown.\\ \hline
		Probability & Low\\ \hline
		Impact & High\\ \hline
		Exposition & Medium\\ \hline
		Indicator & There have been at least 10 failed attempts to download the source material, at least 5 with a computer A in a network X and at least 5 with a computer B in a network Y.\\ \hline
	\end{tabularx}
\end{table}
\begin{table}[H]
	\begin{tabularx}{\textwidth}{|l|X|}
		\hline
		\rowcolor{gray!30}
		Identifier & \textbf{R-009} \\ \hline
		Name & Unexpected changes to any of the APIs used in the project\\ \hline
		Description & Changes to an API could affect this project directly or indirectly. Programs could fail or not work as expected.
		In a project that does not work in a bleeding edge environment, like this, this should be very rare and even if it were to happen it would have to interfere with the part of the API this project uses, which (as this is not bleeding edge) normally would be backwards compatible.\\ \hline
		Probability & Low\\ \hline
		Impact & Low \\ \hline
		Exposition &  Low\\ \hline
		Indicator & There are 3 failures due to changes in APIs.\\ \hline
	\end{tabularx}
\end{table}
\begin{table}[H]
	\begin{tabularx}{\textwidth}{|l|X|}
		\hline
		\rowcolor{gray!30}
		Identifier & \textbf{R-010} \\ \hline
		Name & Loss of work\\ \hline
		Description & Due to a bad configuration management or something else, there is a loss of work related to this project.\\ \hline
		Probability & Low\\ \hline
		Impact &  High\\ \hline
		Exposition &  Medium\\ \hline
		Indicator & The need to replicate already done work is greater than 30 minutes.\\ \hline
	\end{tabularx}
\end{table}
\begin{table}[H]
	\begin{tabularx}{\textwidth}{|l|X|}
		\hline
		\rowcolor{gray!30}
		Identifier & \textbf{R-011} \\ \hline
		Name & Wrong management of the project's configuration\\ \hline
		Description & The project's configuration is inefficient or lacks work.
		Some of the problems could be:
		\vspace{-0.5em}
		\begin{itemize}
		\setlength\itemsep{0em}
			\item Wrong baselines
			\item Wrong identification of the configuration elements
			\item It takes more time than expected to manage the project
			\item Wrong use of the tools
			\item Too much time between commits
			\item Changes are unclear
		\end{itemize}
		\vspace{-0.5em}
		This means the project suffer delays because the need to redo management work and/or planned tasks. \\ \hline
		Probability & Medium\\ \hline
		Impact &  High\\ \hline
		Exposition &  High\\ \hline
		Indicator & There are 3 delays because of the configuration of the project.\\ \hline
	\end{tabularx}
\end{table}
\begin{table}[H]
	\begin{tabularx}{\textwidth}{|l|X|}
		\hline
		\rowcolor{gray!30}
		Identifier & \textbf{R-012} \\ \hline
		Name & A delay in one task leads to cascading delays in the dependent tasks\\ \hline
		Description & A task gets delayed and one or more tasks depends on its completion to start, so they get delayed too.\\ \hline
		Probability & Medium\\ \hline
		Impact &  Medium\\ \hline
		Exposition &  Medium\\ \hline
		Indicator & At least 2 tasks are delayed, due to only one of them needing more time.\\ \hline
	\end{tabularx}
\end{table}
\begin{table}[H]
	\begin{tabularx}{\textwidth}{|l|X|}
		\hline
		\rowcolor{gray!30}
		Identifier & \textbf{R-013} \\ \hline
		Name & Unnecessary work\\ \hline
		Description & Resources are wasted in work that latter is not used. This could happen because multiple reasons, like wrong assumptions or balancing of the remaining time of the project.\\ \hline
		Probability & Low\\ \hline
		Impact &  Low\\ \hline
		Exposition &  Low\\ \hline
		Indicator & At least 3 commits are reverted.\\ \hline
	\end{tabularx}
\end{table}
\begin{table}[H]
	\begin{tabularx}{\textwidth}{|l|X|}
		\hline
		\rowcolor{gray!30}
		Identifier & \textbf{R-014} \\ \hline
		Name & The quality of the product is not enough\\ \hline
		Description & The final result is does not comply the quality standard set for this project. This could mean the need to redo work in a later stage or the incorporation to the official repository being rejected.\\ \hline
		Probability & Low\\ \hline
		Impact &  High\\ \hline
		Exposition &  Medium\\ \hline
		Indicator & Getting 10 suggestions to rework functionality.\\ \hline
	\end{tabularx}
\end{table}
\begin{table}[H]
	\begin{tabularx}{\textwidth}{|l|X|}
		\hline
		\rowcolor{gray!30}
		Identifier & \textbf{R-015} \\ \hline
		Name & Sickness or overwork\\ \hline
		Description & The health of the student deteriorates to the point it affects the project, and it is caused by sickness or overwork.\\ \hline
		Probability & Medium\\ \hline
		Impact &  Medium\\ \hline
		Exposition &  Medium\\ \hline
		Indicator & There is an unexpected delay because the functionality is not done but there has not been any important issues that could explain it. \\ \hline
	\end{tabularx}
\end{table}
\begin{table}[H]
	\begin{tabularx}{\textwidth}{|l|X|}
		\hline
		\rowcolor{gray!30}
		Identifier & \textbf{R-016} \\ \hline
		Name & Performance issues\\ \hline
		Description & The program is too heavy for the environment and takes too much resources, because there are not good enough optimizations or the problems are poorly approached.\\ \hline
		Probability & Low\\ \hline
		Impact &  Low\\ \hline
		Exposition &  Low\\ \hline
		Indicator & The program takes 30\% more resources that at the beginning of the project.\\ \hline
	\end{tabularx}
\end{table}


\subsection{Risk planning}


\subsection{Risk supervision}

\section{Configuration management}
The objective of the management of the configuration is to control the changes on the configuration elements, for the duration of the project. This assures the work is always archived, resulting in multiple control advantages.

\linej
\linej
A Git repository\cite{memoria_github} hosted on Github was used to manage the documentation of the project, which revolves mostly around this document.
Another private Git repository was used to keep track of notes, uncertain elements and development in process.
\linej
Git commits are used to keep track of the changes made in the documentation.
Issues, milestones, releases and other features were not used, but they could be if any need appeared.
There is only the master branch, because this project is not about software development and there is only one contributor.

\subsection{Configuration elements}
They are the items that need to be monitored in order to guarantee the consistency of the project.
They can be grouped into:
\begin{itemize}
	\item \textbf{Code}: The scripts and ruleset elements created in the project.
	\item \textbf{Diagrams and images}: They help to explain some of the key concepts of the project.
	\item \textbf{Memory of the project}: This very document. It is mandatory because it explains the whole project.
		%Without this document there would be no way to understand it.
		This item actually makes use of the other two, to fully document the project.
\end{itemize}
\linej
This does not include the virtual machines nor its snapshots because by themselves they are not configuration material, but a result of applying certain configuration over generic images of operative systems. This configuration is already documented by other sources, cited in this document. Still local backups were made because they are still important archives to the project.

\section{Cost management}
In this section an estimation of the costs of the project is made based on the relevant data available at the end of the project.
\linej
The project is set in Spain, therefore the currency used is the Euro (\euro{}). This is used up to a precision of cents.
\linej
Even though Tarlogic is considered the client there is no payment to the student because this is the final project of a degree.
\linej
The costs were divided into \textit{direct costs} and \textit{indirect costs}.

\subsection{Direct costs}
All the software used in the project was free, including the Windows images (they were free trials).
\linej
\linej
Other materials are the CD to burn the software of the project and the version of this document in paper.
Their combined cost is estimated in 20\euro{}.
\linej
\linej
The rest of the direct costs are the human and hardware resources.

\subsubsection{Human resources}
The human resources are the student, the director from the University, the director from Tarlogic and two members of Tarlogic that helped the student with issues at several points in the project.
\linej
\linej
The next salary parameters are assumed:
\begin{itemize}
	\item There are 14 payments per year.
	\item Social Security costs 32\% of the brute salary.
	\item The job journey is 8 hours of work per day and there are 20 working days per month.
\end{itemize}
\linej
The estimation of the hours that the student worked on the project is \projecthours.
For any one else the hours are unknown, but is safe to assume a low value, therefore 11 hours is the estimation in each case.
\linej
The annual brute income for the job titles are taken from the website \textit{Indeed}\cite{indeed}, which calculates the average of hundreds of job offers in the last months.
In the case of the student the salary was an optimistic value.
\linej
The job title assumed for the student is Junior Developer.
The job title assumed for the directors is Project Manager.
The Project Manager has 22 hours because this project has 2 directors.
\linej
\begin{table}[H]
	\begin{tabularx}{\textwidth}{|X|l|l|l|l|}
		\hline
		\rowcolor{gray!30}
		Role                             & Annual brute &  Social Security &    Total annual & Cost/Hour\\ \hline
		Project Manager                  & 45448\euro{}&   14543.36\euro{}& 59991.36\euro{}& 26.78\euro{}\\ \hline
		Senior Engineer in Cybersecurity & 29754\euro{}&    9521.28\euro{}& 39275.28\euro{}& 17.53\euro{}\\ \hline
		System Administrator             & 24234\euro{}&    7754.88\euro{}& 31988.88\euro{}& 14.28\euro{}\\ \hline
		Junior Developer                 & 16000\euro{}&       5120\euro{}&    21120\euro{}& 9.42\euro{}\\ \hline
	\end{tabularx}
	\caption{Annual costs of the human resources}
\end{table}
\begin{table}[H]
	\begin{tabularx}{\textwidth}{|X|l|l|l|}
		\hline
		\rowcolor{gray!30}
		Role & Hours & Cost/Hour & Total cost\\ \hline
		Project Manager & 22 & 26.78\euro{}& 589.20\euro{}\\ \hline
		Senior engineer in cybersecurity & 11 & 17.53\euro{}& 192.86\euro{}\\ \hline
		System administrator & 11 & 14.28\euro{}& 157.08\euro{}\\ \hline
		Junior Developer & \projecthours & 9.42\euro{}& 3931.71\euro{}\\ \hline
	\end{tabularx}
	\caption{Costs of the human resources for the hours dedicated to the project}
\end{table}
\linej
The total of the last table is 4870.85\euro{}.

\subsubsection{Hardware resources}
The cost of the hardware is calculated with the amortization and the total price of the item, with the next formula:
\begin{figure}[H]
	\[ \frac{Price\ of\ the\ item}{12 \cdot Duration\ of\ the\ item\ in\ years} \cdot Duration\ of\ the\ project\ in\ months\ \]
	\caption{Formula to calculate the cost of hardware items}
\end{figure}
\linej
In this case is assumed that the duration of project is 5.5 months, due to the time the project was on hold.
The duration of the hardware items is estimated in 3-5 years, so the average is used in this case for all the items.
\linej
\linej
The hardware items are:
\begin{itemize}
	\item A computer leaning to the higher end in order to run the set of virtual machines: with 20 GB of RAM, an \textit{i5-2500k} processor and about 500GB of free disk storage (of which 100GB were of Solid State Disk). The estimation of the computer is hard to make because it is several years old and made of parts buyed years apart from each other. The value estimated is 1250\euro{}, therefore resulting in a cost of 143.23\euro{}.
	\item A monitor of 24 inches valued in 147.75\euro{} last year. Using the previous formula the cost is 16.93\euro{}.
	\item Other peripheral devices: their collective value is estimated in 20\euro{}, therefore resulting in a cost of 2.23\euro{}.
\end{itemize}
\linej
The total of the hardware cost is 162.39\euro{}.

\subsection{Indirect costs}
The indirect costs of the project mean hidden costs in the resources used.
For example the cost of the Internet connection, electrical devices, etc.
\linej
According to the General Secretary of the University of Santiago de Compostela the indirect costs in this kind of final degree project should be calculated as an extra 20\% of the direct costs\cite{indirect_costs}.
\linej
\linej
Because the estimation of the direct costs is 5033.24\euro{}, the indirect costs add 1006.64\euro{} over it.

\subsection{Total costs of the project}
The total costs is calculated simply by the addition of direct and indirect costs, resulting in 6039.88\euro{}.

\chapter{Planning}

%Planificación e presupostos: debe incluír a estimación do costo (presuposto) e dos 
%recursos necesarios para efectuar a implantación do Traballo, xunto coa planificación 
%temporal do mesmo e a división en fases e tarefas. Recoméndase diferenciar os costos relativos a persoal dos relativos a outros gastos como instalacións e equipos.








%WBS==EDT (translation!)
\section{Initial WBS}

%\node[style1] {Improvements in IDS, adding functionality to Wazuh}
%child {node[style2] (c1) {Project management}}
%child {node[style2] (c2) {Increment 1: Rules/decoders for common attacks in Windows Server}}
%child {node[style2] (c3) {Increment 2: Rules/decoders with data from Sysmon}}
%child {node[style2] (c4) {Increment 3: Detection/action against ransomware}}
%child {node[style2] (c5) {Increment 4: Adapt Wazuh configuration to typical requirements from enterprises}}
%child {node[style2] (c6) {Increment 5: Explore solutions in problems with GPDR}}
%child {node[style2] (c7) {Increment 6: Rules/decoders for GNU/Linux}}
%child {node[style2] (c8) {Increment 7: VirusTotal integration}};

%\begin{tikzpicture}[%
    %grow=right,
    %anchor=west,
    %growth parent anchor=east,
    %parent anchor=east,
    %level 1/.style={sibling distance=4cm},
    %level 2/.style={sibling distance=2.5em},
    %level distance=1cm]

%\node[root] (root) {Drawing diagrams}
    %child {node[onode] (c1) {Defining node and arrow styles}
        %child {node[tnode] (c11) {Setting shape}}
        %child {node[tnode] (c12) {Choosing color}}
        %child {node[tnode] (c13) {Adding shading}}
    %}
    %child {node[onode] (c2) {Positioning the nodes}
        %child {node[tnode] (c21) {Using a Matrix}}
        %child {node[tnode] (c22) {Relatively}}
        %child {node[tnode] (c23) {Absolutely}}
        %child {node[tnode] (c24) {Using overlays}}
    %}
    %child {node[onode] (c3) {Drawing arrows between nodes}
        %child {node[tnode] (c31) {Default arrows}}
        %child {node[tnode] (c32) {Arrow library}}
        %child {node[tnode] (c33) {Resizing tips}}
        %child {node[tnode] (c34) {Shortening}}
        %child {node[tnode] (c35) {Bending}}
%};
%\end{tikzpicture}
\newpage
{\footnotesize
\begin{forest} for tree={
    grow=east,
    growth parent anchor=east,
    parent anchor=east,
    child anchor=west,
    edge path={\noexpand\path[\forestoption{edge},->, >={latex}] 
         (!u.parent anchor) -- +(5pt,0pt) |- (.child anchor)
         \forestoption{edge label};}
}
[Improvements in IDS: adding functionality to Wazuh, root
    [Closing of the project, onode
        [Project documentation, tnode]
        [Pull request to the official ruleset repository, tnode]
    ]
    [Increment 7: VirusTotal integration, onode
        [Improved integration with antivirus and website scanners, tnode]
        %[Study of the present status, tnode]
    ]
    [Increment 6: Additional detection for GNU/Linux, onode
        [Rules and decoders, tnode]
        %[Analysis and writing of rules/decoders, tnode]
        %[Study of the present status, tnode]
    ]
    [Increment 5: Explore solutions in problems with GPDR, onode
        [Rules and decoders, tnode]
        %[Analysis and writing of rules/decoders, tnode]
        %[Study of GPDR issues, tnode]
    ]
    [Increment 4: Adapt Wazuh configuration to typical requirements from enterprises, onode
        [Configuration changes, tnode]
        %[Analysis of requirements, tnode]
    ]
    [Increment 3: Detection/action against ransomware, onode
        [Rules and decoders, tnode]
        %[Analysis and writing of rules/decoders and actions, tnode]
        %[Study of ransomware attack patterns, tnode]
    ]
    [Increment 2: Use of data from Sysmon, onode
        [Rules and decoders, tnode]
        %[Analysis and writing of rules/decoders, tnode]
        %[Study of Sysmon tools, tnode]
    ]
    [Increment 1: Common attacks in Windows Server, onode
        [Rules and decoders, tnode]
        %[Detailed study of each attack and patterns, tnode]
    ]
    [Beginning of the project, onode
        [Setup of the work environment, tnode]
        [Study of Wazuh documentation and related tools and technologies, tnode]
    ]
    [Project management, onode
        [Cost management, tnode]
        [Configuration management, tnode]
        [Time management, tnode]
        [Risk management, tnode]
        [Requirement management, tnode]
        [Scope management, tnode]
    ]
]
\end{forest}
}


\section{Initial planning}






\section{Final planning}


%para no sobrecargar la jerarquía de la memoria en 3 niveles se usa un capítulo por incremento, estilo como Manuel Simón
\cleardoublepage
\chapter{Increments 1 and 2}
\newcommand{\RNO}{\cellcolor{red!60}No}
\newcommand{\RYES}{\cellcolor{green!60}Yes}

\section{Golden Ticket}
Windows domains are a very common way to manage network accounts in companies. The servers of this kind of domain are Domain Controllers and the program that handles the domain directory is the Active Directory. The Domain Controller (DC) runs the Key Distribution Center (KDC), which handles Kerberos ticket requests, which are used to authenticate users and allow access to services (for example login).
\linej
The KRBTGT account is the equivalent of a super-administrator account for Kerberos. It is used to encrypt and sign all Kerberos tickets within a domain, so DCs use the account password to decrypt Kerberos tickets for validation.
By default this account password never changes and the account name is the same in every domain\cite{stealthbits}.
\linej
\linej
The process to access a service is as follows\cite{tarlogic_theory}\cite{tarlogic_comprehension}\cite{events_1}:
\begin{enumerate}
	\item The user request a Ticket Granting Ticket (TGT). This ticket is encrypted with the KDC key and is used for request to the KDC one or more Ticket Granting Service (TGS). This request is ciphered with the user hash.
	\item The DC returns the requested TGT is everything is in order.
	\item The user requests the TGS.
	\item The DC returns the requested TGS is everything is in order.
	\item The user sends a request to a computer running a service to make use of it. For this the TGS is sent.
	\item Optionally the Privilege Attribute Certificate (PAC) can be sent to the DC to be verified. The PAC is an structure present in almost every ticket that contains the privileges of the user and it is signed with the KDC key. Nevertheless, the PAC verification checks only its signature, without inspecting if privileges inside of PAC are correct. Furthermore, a client can avoid the inclusion of the PAC inside the ticket by specifying it in KERB-PA-PAC-REQUEST field of ticket request. Unfortunately most services do not validate the PAC.
	\item Optionally the DC returns the result to the computer that requested it, which should be running the service in question.
	\item Optionally the user receives the response to his request to use the service.
\end{enumerate}

\begin{figure}[H]
	\label{kerberos_exchange}
	\centering
	\includegraphics[width=.8\textwidth]{figuras/TGT_TGS_PAC.jpg}
	\caption{Steps for Kerberos authentication}
\end{figure}
\linej
The Golden Ticket attack depends on obtaining the password information of the KRBTGT account to generate a forged TGT. It can provide the attacker with the desired privileges in the AD (like administrator).
\linej
This means we can generate TGTs to access every account within the AD. This forged TGT is what we call Golden Ticket.
The Golden Ticket does not depend at all of the administrator password of the AD, which means that changing this password does not invalidate a Golden Ticket in anyway.
\linej
\linej
Because the attack uses a valid TGT is very hard to detect it is a forgery. Once it has been generated the TGT can be used at any time and any amount of times until the time expiration.
Any TGS obtained from a Golden Ticket is no longer a forgery, because is generated from the DC.
\linej
\linej
The password information of the KRBTGT account can be just the hash of the password, which is stored in memory and can be retrieved with enough local privileges in a DC. To generate the Golden Ticket the attacker also needs the domain name and the SID of the domain to which the KRBTGT account belongs, which are trivial to get\cite{stealthbits}.
\linej
Furthermore once the required data is obtained the Golden Ticket can be generated offline using certain programs, like Mimikatz. This means is impossible to detect the creation of the ticket itself if is not done on a computer in the network.
\linej
It follows that this attack can only be detected either during the steps needed to create the TGT (get the hash of the KRBTGT account) or by the use of the TGT.
\linej
\linej
By default Mimikatz sets the forged ticket age to 10 years, which is useful to some attackers because they would need only one attack to compromise the entire network for that time.

\subsection{Exploit methods}
To keep this simple we are only explaining the basics of the techniques used in some of the exploits that can be used for a Golden Ticket attack. Because some of the exploits are similar we number them for easier identification. The scripts in the next exploits were used to understand the different ways to perform Golden Ticket attacks and during the tests to try to detect them. Of course there are more ways to generate a Golden Ticket, and some are much more harder to detect, but there is no time to examine them all. Also it is possible to combine several of the next exploits or change some of their steps.
\linej
For example there is a sever-agent version of Mimikatz called Pypykatz\cite{pypykatz_agent}\cite{pypykatz_server} that is very new and should be a bit harder to detect that the exploits showed here. Unfortunately the student could not make it retrieve the KRBTGT hash.
\linej
\linej
These scripts try to automate as much of the process as possible, which is normally done in an interactive way. This automation helps to ensure that the results are the same each time and reduces the time for each test. All the tests in this project were executed at least twice. Tests were repeated if there were changes that could affect their results.

\subsubsection{Exploit 1: Local Mimikatz in DC}
This requires local administrator privileges in the DC and also and an already downloaded version of Mimikatz in the DC, which the attacker can easily get after gaining privileges. If we were to use this example as it is it will probably be detected by the antivirus and Windows Defender, but again they can be disabled by a local administrator and there are techniques to avoid being detected by them.
\linej
\lstinputlisting[style=PS,caption=Script to generate and inject a Golden Ticket in the local DC,captionpos=b]{scripts/genticket.ps1}
\linej
The script uses Mimikatz to get the needed data to generate the Golden Ticket, saving it to a file for convenience. Then the data is split in variables, each being a piece for generating the forged ticket. The \textit{exit} parameter is to exit the Mimikatz shell after executing the command.
\linej
After injecting the ticket (with the \textit{/ptt} option) we have administrator privileges in the AD, so we can use any service in the AD in this powershell session, any command we type should be allowed. Without the injection option Mimikatz would store the ticket in a file, which we can inject at any time with Mimikatz.
The id 500 is the normal id for the administrator account in the AD.
This script can be executed in multiple ways, for example from a powershell interactive terminal run as an administrator.
\linej
\lstinputlisting[style=txt,caption=Example of the contents of the output file,captionpos=b]{scripts/output.txt}
\linej
The password hash of the KRBTGT account is retrieved by Mimikatz interacting with the Local Security Authority (LSA) or Local Security Authority Subsystem Service (LSASS), which is run by the lsass.exe process.
\linej
This process is the Windows service responsible for providing single sign-on functionality.
With it users are not required to re-authenticate each time they access resources.
It provides access not only to the authenticated user's credentials, but every set of credentials used by every open session since the last boot.
\linej
Mimikatz exploits this cache of credentials and reports the results to the user in the various forms employed by LSASS\cite{SANS_mimikatz}\cite{wikipedia_lsass}\cite{pentestlab}\cite{lsadump_patch_inject}\cite{dump_ways}.

\subsubsection{Exploit 2: Mimikatz from memory in DC}
This is similar to the previous example but instead of having a downloaded version of Mimikatz (stored in the disk) we download the program directly into the powershell session, so Mimikatz is never stored in disk.
The clear advantage over the previous one is that it should be harder to detect.
\linej
With slight changes this could work in a computer that is not a DC, if the attacker compromised a workstation a domain admin logged onto\cite{dump_ways}.
\linej
\lstinputlisting[style=PS,caption=Script to run Mimikatz only in memory and inject a Golden Ticket in the local DC,captionpos=b]{scripts/genticket_mem.ps1}
\linej
The script downloads a version of Mimikatz from Github and creates a powershell object with its contents, that can be invoked at any time in this shell\cite{powersploit}\cite{mimikatz_details}. Then as before it dumps the needed information to generate the Golden Ticket in a file, which is read and parsed to store the interesting parameters into variables.
\linej
I was having trouble trying to automate the last \textit{Mimikatz} command to work with the parameters in the variables so as a workaround I write a new file that has the command with those parameters and then I run the file.
\linej
\linej
On the one hand it can be harder to detect with obfuscation, renaming and not using such an obvious url.
On the other hand powershell can be monitored for downloading commands, particularly those with objects.

\subsubsection{Exploit 3: Mimikatz with DCSync}
The DCSync is a Mimikatz feature which will try to impersonate a DC and request account password information from the targeted DC. This technique is less noisy as it does not require direct access to a DC (which are often heavily monitored)\cite{dump_ways}\cite{pentestlab}. To run Mimikatz we still need local administrator privileges in the computer.
\linej
\lstinputlisting[style=PS,caption=Script to run Mimikatz with DCSync from a no DC computer in the AD network,captionpos=b]{scripts/genticket_dcsync_mimikatz.ps1}
%some times the dcsync Mimikatz command does not work and needs to restart the no DC computer
\linej
This follows the same structure as the previous cases, but this time with the dcsync option.
The disadvantage in this case is that there needs to be a connection to a running DC that is not being monitored for the requests Mimikatz sends. There are open source tools available for this kind of monitoring\cite{dcsync_monitor} and it can also be detected by monitoring the network\cite{dcsync_monitor_network}.

\subsubsection{Exploit 4: DCSync with Kiwi}
In this case the access to the no-DC computer in the targeted network is done through Metasploit and its own version of Mimikatz called Kiwi\cite{pentestlab}. We use the Kali virtual machine to execute Metasploit outside the AD, but we could have used a Windows machine running in the AD to do the same.
\linej
We need to know the password of the account we want to access remotely and the targeted computer needs to have a SMB share.
In this case the variable \textit{SMBPass} stores the password, which is \textit{Passw0rd}.
\linej
\lstinputlisting[style=ruby,caption=Part 1 of the remote DCSync exploit automation,captionpos=b]{scripts/metasploit_p1_kiwi.rc}
\linej
This runs a remote process, exploiting the SMB capabilities to run commands to spawn a Meterpreter shell\cite{meterpreter} with administrator privileges. There is a chance that the \textit{run} command fails to provide a Meterpreter shell at this stage, but trying again always ends working because the session is not getting created even though the exploit is working.
\linej
\linej
Now we are in a Meterpreter shell, which we can use to get the exact privileges we need for the next part. This is because even though we have administrator privileges there are different kinds of administrator privileges on Microsoft systems.
\begin{figure}[H]
	\centering
	\includegraphics[width=\textwidth]{figuras/meterpreter.png}
	\caption{Meterpreter shell running}
\end{figure}

To do this we look for a session running administrator privileges in the AD. In this case the targeted machine had a powershell session running as administrator, to which we migrate.
\linej
\begin{figure}[H]
	\centering
	\includegraphics[width=\textwidth]{figuras/migrate.png}
	\caption{Migration to a powershell session as AD administrator}
\end{figure}

Now we are ready to run the real exploit. This loads the Metasploit version of Mimikatz (Kiwi) in the Meterpreter shell, allowing the attacker to use Kiwi commands.
The command in this case retrieves the information of the KRBTGT account needed to generate the Golden Ticket.
\linej
In this case the generation of the ticket is not using the data in an automated way, because there was no real need since is the same every time and the time needed to do this with Ruby felt like a waste. In this case the ticket is saved to the \textit{/tmp/golden.tck} file in the Kali machine.
\linej
\lstinputlisting[style=ruby,caption=Part 2 of the remote DCSync exploit automation,captionpos=b]{scripts/metasploit_p2_kiwi.rc}
\begin{figure}[H]
	\centering
	\includegraphics[width=\textwidth]{figuras/kiwi_p2.png}
	\caption{Retrieval of KRBTGT data and generation of the Golden Ticket with DCSync}
\end{figure}
The obvious downside of this method for the attacker is that Metasploit is very widely used and known, therefore there could be security monitoring for it\cite{detect_metasploit_traffic}. But again we are using a technique that does not need to control a DC and does not need to store anything in the disk of the targeted system, making it much harder to detect in that regard.
\linej
\linej
Of course there is no real need to use Metasploit to get a remote shell to run Mimikatz. The attacker could use ssh, run remote commands with \textit{psexec} or use the Windows Remote Shell. But some of these need to be enabled and they would not be much different of the previous examples.

\subsubsection{Exploit 5: Hashdump with Meterpreter}
Using a reverse TCP exploit the attacker access the targeted DC with a Meterpreter shell. This is similar to the previous case but using the Meterpreter command \textit{hashdump} instead of the DCSync retrieval of Kiwi\cite{pentestlab}. This stills uses Kiwi to generate the Golden Ticket.
\linej
\lstinputlisting[style=ruby,caption=Part 1 of the remote Hashdump exploit automation,captionpos=b]{scripts/metasploit_p1_hashdump.rc}
\linej
Again there is a migration to an administrator account of the AD. In this case another command to get the SID of the network would be needed if we did not know it already, for example a simple \textit{whoami /user} would suffice.
\linej
\lstinputlisting[style=ruby,caption=Part 2 of the remote Hashdump exploit automation,captionpos=b]{scripts/metasploit_p2_hashdump.rc}
\begin{figure}[H]
	\centering
	\includegraphics[width=\textwidth]{figuras/hashdump.png}
	\caption{Retrieval of KRBTGT data with hashdump}
\end{figure}
As before it is expected of the DCs to be more monitored. This means that the DCSync version is more interesting to an attacker because it has the same difficulty and beneficts at a lower risk.

\subsection{Detection purely with signatures}
Searching for suspicious strings can be used to trigger alerts with Wazuh.
Signature matching of data coming from Windows events and default logs from Windows systems is not much different from what any antivirus do. It follows that it would be as easy to evade as them. For example with substitution of suspicious strings from the source code and recompilation\cite{understanding_powersploit_mimikatz}.
\linej
\linej
This does not mean that signature matching is totally worthless, but that it is not something to focus on. In this project we avoid relying in signatures as much as possible.
Useful signatures to match not only amount to third-party tools. It is easy to monitor extensions, directories, certain files, command line options, usernames, etc.
\linej
Obviously matching some signatures is more valuable than matching others. For example finding \textit{Mimikatz.exe} is easier to implement and less valuable than \textit{krbtgt}, \textit{lsadump}, \textit{kerberos::} or \textit{privilege::}.
\linej
\linej
As we know the techniques to detect or avoid detection evolve with each other over time.
Any way we manage to detect programs like Mimikatz may be overcome in the future.
\linej
That does not mean that detection techniques are bound to fail or that there is no point to them.
Many attackers know how hard it can be to overcome detection techniques.
\linej
It is also important to remember these scenarios are continously changing, with new techniques to avoid detection and the creation of new tools.

\subsection{Detection purely with Windows events}
In theory we can identify certain attacks with the security events of Windows. There are multiple websites in which this attack has been analyzed and its events identified\cite{events_1}\cite{detection_events}.
\linej
Unfortunately the events recorded did not always probe to be the same as the cited sources (probably because we tested on the new Windows Server version, 2019).
In most cases their frequency is not enough to be distinguished of the regular activity, even when using a lab environment without work load.
This could probably be improved if these events had more information (particularly those as critical as Kerberos'), but they are very short and generic.
\linej
\linej
Wazuh provides access to Windows events by default, due to the rules and decoders of its ruleset\cite{wazuh_ossec_ruleset}, the user only needs to define rules to specify what and how he wants to monitor.
\linej
We can enable additional logging with the Advanced Audit Policy Configuration. For example for auditing kernel objects, more Kerberos logging, changes in settings or account events.
\linej
\linej
The student tried to analyze the security events to find patterns in the previous exploits, by recording all the data received by Wazuh in during their execution. This was done just by looking the current line of the log, executing the exploit and copying the log from there to a new file.
\linej
The obvious problem of this method is that it results in logs with tens to hundreds of lines filled with a not very easy to read format. The workaround used was to parse the logs with custom AWK scripts\cite{memoria_github}, removing fields to make the logs more readable and counting each of the events in them.
\linej
\linej
The idea of finding a relationship between certain windows events and this attack was abandoned because:
\begin{itemize}
	\item It was not providing any new results.
	\item It was too time consuming.
	\item It can be accomplished using Sysmon\cite{sysmon}\cite{sysmon_event_7_mimikatz}.
	\item Also it was concerning the amount of noise this method has, even though in a laboratory without real system load, to tell apart an intrusion from a totally healthy system.
\end{itemize}
\linej
The real useful addition to the Windows builtin events is having Sysmon\cite{sysmon} in each of the monitored Windows computers. They can be combined to identify attackers better\cite{detection_events}.
\linej
With Sysmon we can have reports of events [1-21] and 255, which in some cases provide very precise and useful information of the system. For example we can configure Sysmon to log data about process with a certain processes, like: \textit{powershell}, \textit{Mimikatz}, any \textit{.ps1} or any \textit{.exe}.
Sysmon can monitor each of the events either by whitelisting or blacklisting by default or both. We can also combine it with rules from Wazuh, using Sysmon to increase the report capabilities and Wazuh to filter them.
\linej
\linej
Also is important to note that Sysmon can be a bit tricky to balance the configuration to get as much suspicious events as possible, while not reporting so much it affects the performance of the network. This is responsibility of the administrators of the network, who also have to tune the configuration to their custom needs.
There are public configs for Sysmon that attempt to provide a good insight of the system while not logging too much data\cite{sysmon_config}.

\subsection{Detection of Mimikatz}
Mimikatz is the tool of choice for this kind of attack for most attackers because it is very effective, easy to use and has multiple ways to be used in different attacks\cite{mimikatz_github}\cite{mimikatz_details}. This is a double edge sword for Mimikatz because it has become one of the programs to look for in antimalware detection programs. In this case we assume these programs have not detected Mimikatz and is up to Wazuh to do it. It is interesting to note that the author of Mimikatz provides ways to detect it, like the YARA rules he maintains\cite{mimikatz_github} or BusyLights\cite{understanding_powersploit_mimikatz}.
\linej
\linej
Detecting Mimikatz is not a sign of a Golden Ticket attack (unless is clear in the way it is used), but still it is a big and dangerous threat to the system and worth checking out.
\linej
Unfortunately as seen in the exploits before there are multiple ways to execute Mimikatz, in an attempt not to be discovered by known techniques.
\linej
\linej
Each time a Mimikatz shell spawns certain DLLs are loaded. The technique to identify a sucession of events in a short time as another event is called grouping.
Grouping is a very effective technique, but it may require a lot of work to identify its components. In some cases the attack may not produce enough noise or it may not be possible to tell it apart from the normal events of the system\cite{sysmon}\cite{sysmon_event_7_mimikatz}.
\linej
Fortunately Mimikatz needs a fair amount of DLLs to work and some of them are not very usual. This makes the execution of Mimikatz noticeable.
\linej
\linej
 The load of a DLL can be detected by the event 7 of Sysmon and the grouping can be identified with Wazuh rules. It also can be detected by the event 10 of Sysmon, for inter-process access, but a greater cost of bandwith.
For this task is better to configure Sysmon to monitor these 5 images, to avoid logging too much:
\linej
\lstinputlisting[style=xml,caption=Sysmon monitoring with event 7 for certain DLLs,captionpos=b]{scripts/sysmon_event_7.xml}
\linej
On the manager side the next rules are needed:
\linej
\lstinputlisting[style=xml,caption=Rules for suspecting a Mimikatz execution as a group of events,captionpos=b]{scripts/rules_event_7_mimikatz_alternative.xml}
The \textit{sysmon\_event7} means that another rule has marked the log as a Sysmon event of type 7.
\linej
The \textit{same\_field} option means that every one of the matches must have the same value in the designed field, which in this case means that these events come from the same computer.
\linej
The \textit{frequency} option means that the rule has to be matched that number of times to trigger. Is set to 2 because is the minimum value possible.
\linej
\linej
Normally each of the suspicious DLLs would have its own rule, but it would not always identify Mimikatz because the frequency has to be at least 2. Therefore rules \textit{300301} and \textit{300302} identify 2 and 3 DLLs each (using a logical OR), making it possible to trigger the grouping rule.
The last rule identifies the use of the DLLs in a 10 seconds gap as the execution of Mimikatz. This detection could be evaded by adding time between the load of the DLLs in the source code.
\linej
\linej
The problem of these less precise rules is that it is possible to have false positives. None were seen during this project for this case.
\linej
Unfortunately due to the way OSSEC matches rules there is no way to have an hierarchy of rules to trigger a precise grouping rule or the other one.
\linej
\linej
Detecting the use of every variant of Mimikatz is virtually impossible, not only because their sheer number due to its popularity but because anyone can compile their own. Therefore the logical way to detect Mimikatz would be to detect the basic step for every version: the interaction with the LSASS and process injection. More can be read on page \pageref{detect_lsass}.

\begin{table}[H]
	\centering
	\begin{tabular}{|l|l|l|}
		\hline
		\rowcolor{gray!30}
		Exploit & Detected \\ \hline
		1: Local Mimikatz in DC& \RYES\\ \hline
		2: Mimikatz from memory in DC& \RYES\\ \hline
		3: Mimikatz with DCSync& \RYES\\ \hline
		4: DCSync with Kiwi& \RYES\\ \hline
		5: Hashdump with Meterpreter& \RYES\\ \hline
	\end{tabular}
	\caption{Exploit detection by grouping events}
\end{table}
This method detects the use of Mimikatz in all the ways implemented in this project.
The hashdump exploit is detected because Kiwi is used in the session in that machine to generate the Golden Ticket. A real attacker probably would generate the ticket outside of the network, avoiding being detected by this technique.

\subsection{Detection of the use of the TGT with klist}
We can not always detect when a forged TGT is generated, but the attacker still needs to use it to gain access to the active directory domain with the privileges set in the ticket. The first choice for this task would be to monitor the Kerberos log searching for unusual patterns, but it proved to be more hard than it should, so instead we use scan the cache of Kerberos tickets every few minutes.
\linej
The program to examine the contents of the cache is \textbf{klist}.
\linej
\linej
In order to do this we need to enable the execution of Wazuh's remote commands in the Windows agent and set the properties of the command in the manager in \textit{/var/ossec/etc/shared/default/agent.conf}\cite{wazuh_remote_command}:
\linej
\begin{lstlisting}[style=xml]
<agent_config os="Windows">
	<wodle name="command">
		<disabled>no</disabled>
		<tag>remoteklist</tag>
		<command>powershell C:\\Users\\Public\\Documents\\klist.ps1</command>
		<interval>5m</interval>
		<ignore_output>no</ignore_output>
		<run_on_start>yes</run_on_start>
		<timeout>0</timeout>
		<skip_verification>yes</skip_verification>
	</wodle>
</agent_config>
\end{lstlisting}
\linej
In this case the command is a script to get all the tickets of all the sessions with klist, compare the ticket value for the field \textit{TicketExpireHours} with the value of \textit{MaxTicketAge} of the Group Policy (putting the difference in a new field) and parse the output to JSON. Having the output in JSON makes it a bit easier to read from the logs (which is useful to fix any mistake in the script) and removes the need of a decoder in the manager.
The idea came from a very different klist script that only works interactively and reports in plain text\cite{klist_script_idea}.
\linej
This script needs to be run in every member of the network to ensure detection for every user.
\linej
Doing this with only powershell ensures it will work in any Windows system without external programs. The downside of this parsing and my limited knowledge of powershell is that the script is a bit bulky and the dependency on the format of the output of klist.
\linej
\lstinputlisting[style=PS,caption=Script to scan and parse to JSON the tickets in the cache,captionpos=b]{scripts/klist.ps1}
\lstinputlisting[style=PS,caption=Way to get the MaxTicketAge from the Group Policy,captionpos=b]{scripts/report.ps1}
\linej
Unfortunately that way to get the \textit{MaxTicketAge} from the Group Policy in the last script does not work by default with remote commands because Windows remote commands only allow certain types of commands.
In any case the \textit{MaxTicketAge} value is not normally changed and it requires AD administrator privileges to do it, so due to the time constrains of the project this automation was abandoned. There are other ways to get the \textit{MaxTicketAge} value, but as mentioned is not something that we should spend time on.
\linej
\linej
Next there is an example of the difference between the normal output of klist and the string stored in an alert in the manager.

\begin{figure}[H]
	\centering
	\includegraphics[width=.9\textwidth]{figuras/klist_normal.png}
	\caption{Klist listing tickets for a certain session}
\end{figure}
\begin{figure}[H]
	\centering
	\makebox[\textwidth][c]{\includegraphics[width=1.2\textwidth]{figuras/klist_alert.png}}
	\caption{Latest alert of the klist monitoring in the manager}
\end{figure}
\linej
The time difference mentioned before is a very easy way to detect a forged ticket. With a simple subtraction in the powershell script only a rule that makes a number comparison in the manager is needed to launch the alert.
\linej
\lstinputlisting[style=xml,caption=Rules to detect a suspicious expiration age from the report of the klist script,captionpos=b]{scripts/rules_klist.xml}
\linej
The purpose of the first rule is to identify any log in JSON with \textit{MaxTicketAge} and \textit{TicketExpireHours} fields. The second rule is used to examine the contents of the \textit{TicketExpireHoursGap} field of the logs that the first rule has identified. If the value of the \textit{TicketExpireHoursGap} field starts with a digit different than 0 then it means that \textit{MaxTicketAge} $>$ \textit{TicketExpireHours}, therefore the expiration age is greater that it should, triggering an alert.
Additionally it can only trigger once each 600 seconds, to avoid flooding of alerts.
\linej
\linej
This attack is often used because it may grant the highest privileges in the domain, is hard to detect and is very persistent because it does not care for the password changes in the active directory. That is why is very attractive for domains in which the attacker may decide to come back later, maybe even years later. That means is very unlikely for a forged ticket to not have a very big expiration age, because is one of its most appealing beneficts; but again it would be possible to an attacker to keep generating forged tickets with a valid expiration age forever, at a greater risk of being detected in other ways.
\linej
\linej
The testing of the script was satisfactory, the scripts that inject a TGT were detected and there were no false positives:
\begin{table}[H]
	\centering
	\begin{tabular}{|l|l|l|}
		\hline
		\rowcolor{gray!30}
		Exploit & Detected & As expected \\ \hline
		1: Local Mimikatz in DC& \RYES& \RYES\\ \hline
		2: Mimikatz from memory in DC& \RYES& \RYES\\ \hline
		3: Mimikatz with DCSync& \RYES& \RYES\\ \hline
		4: DCSync with Kiwi& \RNO& \RYES\\ \hline
		5: Hashdump with Meterpreter& \RNO& \RYES\\ \hline
	\end{tabular}
	\caption{Exploit detection by the klist script}
\end{table}
Of course if we chose to store the ticket in a file we could inject it in other moment or computer, but then it could be detected by this method.
\linej
\linej
Additionally we could monitor looking for unusual usernames, because is possible to get a TGT with administrator privileges with non existent username to avoid the monitoring that administrator accounts are often under.
\linej
\linej
It can be worth checking if the attacker is using klist to clean the cache of injected tickets, to cover any tracks. This can be easily accomplished monitoring the execution of klist with the event 1 of Sysmon:
\begin{lstlisting}[style=xml]
<ProcessCreate onmatch="include">
	<Image condition="contains">klist</Image>
</ProcessCreate>
\end{lstlisting}
\linej
And checking with Wazuh if the option \textit{purge} was used:
\lstinputlisting[style=xml]{scripts/rules_klist_purge.xml}

\subsection{Silver Ticket}
A Silver Ticket is very similar to a Golden Ticket, is a forged TGS instead of TGT. Therefore a Silver Ticket only grants access to a service in a computer. Is important to note that some services need the privileges of more services, therefore more than a Silver Ticket may be needed.
\linej
Steps 1 and 2 of a normal Kerberos authentication exchange are not needed (figure \ref{kerberos_exchange}) because they are only to get a TGT. Without a TGT a TGS can not be requested from the DC, so steps 2 and 3 are also not a part of the Silver Ticket attack.
\linej
There is no need to connect to a DC, only a connection to the computer hosting the service is needed (steps [5-8]). Unless PAC validation is required, the service accepts all data in the TGS ticket.
\linej
The TGS is cyphered with the password hash of the account running the service, making changes of the password an effective mitigation against Silver Tickets.
To extract this data from memory the attacker has to have local administrator privileges\cite{events_1}\cite{silver_ticket}.
\linej
\linej
To extract the data the attacker would need to run Mimikatz with:
\begin{lstlisting}[style=PS,numbers=none]
"privilege::debug" "sekurlsa::logonpasswords" exit
\end{lstlisting}
\linej
For example in the next scenario:
\begin{itemize}
	\item The user to impersonate is the AD Administrator.
	\item The computer is is \textit{WIN-GQR2EQ8M0TF}.
	\item The domain is \textit{wazuh.local}.
	\item The domain is identified as \textit{S-1-5-21-3307301586-4221688441-1196996515}.
	\item The attacker wants access to the \textit{HOST} service.
	\item The password hash of the account is \textit{68fbd238f574f7685beed96a2db15004}.
\end{itemize}
The Mimikatz command would be:
\begin{lstlisting}[style=PS,numbers=none]
"kerberos::golden /admin:Administrator /id:500 /sid:S-1-5-21-3307301586-4221688441-1196996515 /domain:wazuh.local /target:WIN-GQR2EQ8M0TF.wazuh.local /rc4:68fbd238f574f7685beed96a2db15004 /service:HOST /ptt" exit
\end{lstlisting}
Allowing the attacker to access the \textit{HOST} service on that computer with AD Administrator privileges.
\linej
\linej
Silver Tickets get registered in the Kerberos' cache in the same way as the Golden Tickets, so they can be detected with the klist script. The execution of Mimikatz can be detected with grouping just as before.

\subsection{Mitigation}
These exploits take advantage of the inherent weaknesses of Kerberos, so there is no way to prevent them. Nevertheless, Microsoft provides a public guide explaining how to mitigate this kind of attacks\cite{microsoft_mitigation}.
The easiest way to mitigate this attack is to change the password of the KRBTGT account to invalidate any existing Golden Ticket, which has to be done twice (make sure the domain converges before doing the second password change\cite{hood}), but it also invalidates existing proper TGTs.
\linej
The recommendation from Microsoft is to regularly reset the password\cite{tarlogic_comprehension}\cite{adsecurity_483}, which can be done with their official script\cite{reset_script}. This could be also triggered by alerts that we are confident detect Golden Tickets, but as mentioned this could affect other functionality and so the decision is for the network administrators to make. Any TGT that is not valid produces an error in a TGS request, which can be used for exposing an attacker\cite{scom_GT}.
\linej
\linej
Also we always can take measures like:
\begin{itemize}
	\item Have administrative passwords longer than 25 characters to avoid brute force cracking and make them unique for each system.
	\item Enforce a least privilege model.
	\item Minimize the quantity of administrative accounts.
	\item Isolate DCs: Use DCs only as servers, never work stations of any kind.
	\item Isolate administrator accounts: Use administrator accounts only for administrator duties.
	\item Isolate AD accounts: Create tiered groups with very granular permissions on the domain and create Access Control List permissions on the Organization Units of the AD\cite{AD_tier}.
	\item Use Read Only Domain Controllers (RODCs): keep Read Write DCs segregated using network segregation and AD sites to force users to logon to RODCs, making breach detection easier. RODCs don't have any real user hashes (nor the hash of the KRBTGT account)\cite{hood}\cite{reset_RODC}.
	\item Use honeypots: With populated the LSASS cache with false credentials\cite{SANS_mimikatz}\cite{honeyhashes} or with decoy AD objects\cite{decoy_AD}. Then we monitor the logs for attempts to use them. This can lead to detect attackers or to find vulnerabilities in the network.
	\item Disable storage of clear text passwords in LSASS memory to limit the information provided by Mimikatz\cite{SANS_mimikatz}.
	\item Run LSASS in protected mode (from Windows 8.1): calls to LSASS are only allowed by other protected-mode processes\cite{SANS_mimikatz}\cite{understanding_powersploit_mimikatz}.
	\item Use choke points: Create a choke point for access to your DCs, adding another layer of protection. Create a Terminal Server that can only talk to the DCs. Configure the DCs to only accept administrative connections from that Terminal Server\cite{choke}.
\end{itemize}
\linej
We could go on with more detail and increasing the mitigation\cite{AD_defense}, but is not the objective of this project.

\subsection{Conclusion}
We have seen how the data to generate Golden Tickets can be obtained in different ways and the difficulties for both the attacker and the defender roles.
\linej
\linej
Relying on the klist detection means there is no real need to detect each of the different ways to generate the Golden Ticket because it may be impossible depending on the circumstances. More importantly the attacker still needs to present it to a DC to get the TGSs, to get any benefit from the Golden Ticket.
\linej
Detecting certain Sysmon events in a close time gap can guarantee the detection of Mimikatz, therefore detecting one of the most used ways to gather this data. Detecting certain signatures for running commands, reads and accesses are a worthy way to detect the creation of a Golden Ticket, without spending much resources.
\linej
These are good examples of how detecting common steps to multiple exploits is one of the strong points of an HIDS.
\linej
\linej
Another way of detection is to use YARA to look for certain patterns in memory, just like we can search for strings in the events. In the case of events the data comes from the program, which is easy to modify with multiple techniques like substitution or obfuscation. The patterns in memory are much more harder to change because it involves changing the logic of the program. That means most attackers would just take the risk to be detected by this kind of technique.
\linej
YARA is very interesting for this kind of project, but it belongs to the Virustotal pack of malware detection tools and so it could be used with Wazuh with a Virustotal API key. The free version only allows a few queries and we didn't consider the option of getting a premium key because there has been for months an open issue in the Github page of Wazuh for integrating it with YARA (as other IDSs have done before), which has recently evolved to an issue to integrate YARA into Wazuh as a module\cite{yara_module}.
%Unfortunately this was not done before this part of the project as completed, so there was no time left to investigate more on it.
%TODO revisar estado issue en el futuro


\section{More about the extraction of credentials}
In the previous section the extraction of credentials was explained to understand the details surrounding a Golden Ticket attack. This includes extracting password hashes from the memory of the LSASS process with Mimikatz or Hashdump. But there are more ways of extraction that are used against AD network now a days.
\linej
Once credentials have been retrieved an attacker has more options, like generating a Golden Ticket.
The key points of access are the \textit{NTDS.DIT} file that is stored in disk and the running process \textit{lsass.exe}.

\subsection{Exploit methods}
Because the database file of the AD accounts is locked from copying and reading, only Windows tools are allowed to. These tools are \cite{dump_ways}:
\begin{itemize}
	\item Reg: Allows to change or save registry hives, including those that contain credentials.
	\item Ntdsutil: Provides management of this database, including creation of backups.
	\item WMIC: Commands for the Windows Management Instrumentation. They allow all kinds of remote management, including copy of files using Shadow Copy.
\end{itemize}
Another way to extract these credentials is to dump them from memory using third party tools and scripts. This is saving part of the data of a process running in the system\cite{dump_ways}.
There are multiple tools available for this, but in this project only these were used: Mimikatz, Hashdump, ProcDump, pd, Minidump and NinjaCopy.
\linej
Some of these tools have the option to retrieve the password or hashes history, meaning that the attacker could gain valuable insight on the password policy of the target.
\linej
\linej
There was no effort to automate these exploits because they are too simple.
All the extraction programs were executed with local administrator privileges in a DC.

\subsubsection{Exploit 6: Retrieval of NTDS.DIT with ntdsutil}
Another way to get the desired information is to copy the database of the AD Domain Services (the NTDS.DIT file) and conduct an offline password audit of the domain. This means once we have this data we can use a wide selection of tools to crack it\cite{ntdsdit_tools}\cite{extracting_ntds}\cite{ntds_powershell}.
\linej
\linej
The attacker has to open a shell as administrator in a DC to create the backup:
\begin{lstlisting}[style=PS,frame=none]
ntdsutil "activate instance ntds" ifm "create full C:\temp\ntdsutil" quit quit
\end{lstlisting}
\linej
This command creates a ntdsutil shell and activates the instance to later create a backup in a temporary directory (inside the ifm subshell).
\begin{figure}[H]
	\centering
	\includegraphics[width=\textwidth]{figuras/ntdsutil.png}
	\caption{Backing up the database of the AD using the ntdsutil shell}
\end{figure}
There are other ways to use ntdsutil in ways harder to detect\cite{more_dumps}, but this is enough for gathering events for analysis.
\linej
\linej
The creation of processes related to ntds can be reported by Sysmon:
\begin{lstlisting}[style=xml]
<ProcessCreate onmatch="include">
	<Image condition="contains">ntdsutil.exe</Image>
</ProcessCreate>
\end{lstlisting}
\linej
And alerts set with Wazuh:
\lstinputlisting[style=xml]{scripts/rules_ntds.xml}
\linej
The first rule is the parent, it filters windows events with the \textit{ntds} string.
\linej
The second rule detects the ntdsutil executable. This signature is more useful than normal because it is a builtin tool. The third matches \textit{ntds} for the command line, which is not really reliable.
\linej
The rest are for detecting suspicious database events from Windows events. These database events ensure the detection of ntdsutil even if it were to be executed without using known signatures.
\linej
\linej
There is also the remote version of Reg: WinReg. But there was no time to investigate on it. Is likely that it can be detected with network monitoring, as other remote tools.

\subsubsection{Exploit 7: Storing registry hives with Reg}
These commands produce the different save files, each of a different group of credentials, that can be later extracted offline with certain tools\cite{more_dumps}:
\begin{lstlisting}[style=PS]
reg.exe save hklm\sam c:\temp\sam.save
reg.exe save hklm\security c:\temp\security.save
reg.exe save hklm\system c:\temp\system.save
\end{lstlisting}
\linej
Reg is not detected as malware because it is a builtin tool in Windows. But we can detect it with Sysmon and Wazuh. With Sysmon we report the execution of Reg with the event 1, reporting the creation of a process:
\begin{lstlisting}[style=xml]
<ProcessCreate onmatch="include">
	<Image condition="contains">reg.exe</Image>
</ProcessCreate>
\end{lstlisting}
\linej
And Wazuh to trigger an alert:
\lstinputlisting[style=xml]{scripts/rules_reg.xml}
\linej
The parent rule matches the creation of Reg from the report of Sysmon.
\linej
The second rule detects the \textit{save} string in the reg command.
\linej
The rest of the rules detect the registry strings for credentials.
\linej
\linej
This also could be detected using the event 7, but it would interfere with grouping detection. Of course it is possible that these rules do not cover all the extraction uses of Reg. Is worth noting that this is a signature base detection, therefore it could be overcome with certain third-party programs.
\linej
\linej
Additionally a grouping detection for the DLLs Reg uses was tested. It detected Reg events every time without relying on the \textit{reg.exe} signature, but there were false positives, particularly during boot.

\subsubsection{Exploit 8: Dump of LSASS with ProcDump}
ProcDump\cite{procdump} is a command-line utility whose primary purpose is monitoring an application for CPU spikes and generating crash dumps during a spike. This program can be used to create a dump file of the running \textit{lsass.exe} process:
\begin{lstlisting}[style=PS,numbers=none]
C:\Users\Administrator\Downloads\procdump.exe -accepteula -64 -ma lsass.exe c:\temp\lsass.dmp
\end{lstlisting}
\linej
The dumped file can be used to extract the credentials by other programs, like Mimikatz\cite{more_dumps}. This is also true for the next exploits.

\subsubsection{Exploit 9: Dump of LSASS with pd}
ProcessDumper, also known as pd\cite{pd}, is another program to dump the \textit{lsass.exe} contents. For example if the id of the process is 552:
\begin{lstlisting}[style=PS,numbers=none]
C:\Users\Administrator\Downloads\pd.exe -p 552 > c:\temp\lsass.dump
\end{lstlisting}

\subsubsection{Exploit 10: Dump of LSASS with Minidump}
Minidump is a script from the PowerSploit Post-Explotation Framework\cite{powersploit}. It can be combined with the \textit{Get-Process} builtin to dump the process into a file:
\begin{lstlisting}[style=PS]
Import-Module c:\users\administrator\downloads\PowerSploit-master\Exfiltration\Out-Minidump.ps1
Get-Process lsass | Out-Minidump -DumpFilePath c:\temp
\end{lstlisting}

\subsubsection{Exploit 11: Dump of LSASS with NinjaCopy}
%%fixes needed
%%	%https://github.com/PowerShellMafia/PowerSploit/commit/bd6fe64316afe293d6b4cdf095ed3cfb64b6ab25
%%	%https://github.com/PowerShellMafia/PowerSploit/issues/293
Another PowerSploit module that can be used to dump LSASS into a file is \textit{NinjaCopy}:
\begin{lstlisting}[style=PS]
Import-Module C:\Users\Administrator\Downloads\PowerSploit-master\Exfiltration\Invoke-NinjaCopy.ps1
Invoke-NinjaCopy -Path "c:\windows\ntds\ntds.dit"  -LocalDestination "c:\temp\ntds.dit"
\end{lstlisting}
\linej
Or to copy the NTDS.DIT of the DC in this laboratory to a no-DC computer:
\begin{lstlisting}[style=PS]
Import-Module C:\Users\Administrator\Downloads\PowerSploit-master\Exfiltration\Invoke-NinjaCopy.ps1
Invoke-NinjaCopy -Path "c:\windows\ntds\ntds.dit"  -LocalDestination "c:\temp\ntds.dit" -ComputerName "WIN-25U0PFAB511"
\end{lstlisting}
\linej
This module allows any file, even if it is locked, to be copied without starting suspicious services or injecting in to processes. This is because it can copy \textbf{any file} from a NTFS volume, by opening a read handle to the entire volume, therefore bypassing the following protections\cite{dump_ways}:
\begin{itemize}
	\item Files which are opened by a process and cannot be opened by other processes, such as the NTDS.dit file or SYSTEM registry hives. This is known as locking the file.
	\item Flags set on a file to alert when the file is opened. Windows can not set a flag because NinjaCopy does not use a Win32 API to open the file.
\end{itemize}
\linej
The code to parse NTFS is loaded with a reflective DLL, making it harder to detect because it does not use the Windows loader nor a DLL file.
\linej
The direct read of the device can be reported with the event 9 of Sysmon, but event 1 can be useful to make sure in the remote case. This needs to be in the configuration file of Sysmon in the DC:
\begin{lstlisting}[style=xml]
<ProcessCreate onmatch="include">
	<Image condition="contains">wsmprovhost.exe</Image>
</ProcessCreate>
<RawAccessRead onmatch="include">
	<Image condition="contains">powershell.exe</Image>
	<Image condition="contains">wsmprovhost.exe</Image>
</RawAccessRead>
\end{lstlisting}
\linej
\lstinputlisting[style=xml,caption=Rules for detecting NinjaCopy,captionpos=b]{scripts/rules_ninjacopy.xml}
\linej
The local case only generates an event of type 9 and its only signatures are \textit{powershell} and \textit{Device{\textbackslash}HarddiskVolume}. It does not even record the name of the file because the handle is for the volume. This can not really ensure this event comes from the execution of NinjaCopy, but until now has always worked and never reported a false positive. This rule could be much more useful if the NTDS.DIT file were in a different volume than normal.
\linej
\linej
The remote case is much more easier to detect. In the tests it spawned at least 4 events of each type in about 12 seconds, all from the DC.
The bigger the NTDS.DIT file and the slower the connection the more events are produced.
\linej
The remote command is executed by the \textit{wsmprovhost} program, which stands for Windows Remote Powershell Session.
The contents of the logs of the event type 1 are easy to distinguish from normal, due to the constant value of the \textit{commandLine} and \textit{parentCommandLine} fields.
\linej
The second and third rules detect the events of type 1 and 9 for the remote execution, and the last rule is a grouping rule of these two. This ensures there are no false positives, but the second rule matches the exploit so well that it could be left out and probably would never cause false positives.
\linej
If the remote command is executed from DC and set to copy from the same computer it still triggers the remote alert.
\linej
\linej
Grouping the DLLs loaded by the \textit{Import-Module} command probed to be effective. As with Mimikatz, this loads the DLLs needed for NinjaCopy to work, which can be registered by the event 7 of Sysmon:
\lstinputlisting[style=xml]{scripts/sysmon_event_7_ninjacopy_import.xml}
\linej
These 8 DLLs are detected by 4 rules to reach the minimum frequency:
\lstinputlisting[style=xml]{scripts/rules_ninjacopy_import.xml}
\linej
No false positives were found, but they may appear in a real environment. This technique depends too much on the time frame and can be avoided changing the code. It works with both local and remote cases the same.
%\linej
%\linej
%The next detection attempt was to use a grouping technique for the DLLs the program uses during the copy process. All of them were monitored with the event 7 of Sysmon. The most frecuent were: advapi32, kernel32, kernelbase, msvcrt and ntdll.
%\linej
%These DLLs are used very often in any healthy system. The script did not increase their use too much, therefore false positives were produced during normal execution, discarding this method of detection.

\subsection{Detection of process accessing LSASS} \label{detect_lsass}
The event 10 of Sysmon reports when a process access another process, possibly detecting hacking tools that read the memory contents of processes\cite{sysmon}.
This event can be used to detect LSASS dumps, at least in some cases\cite{sysmon_event_10_lsass}.
\linej
The downside is it can generate significant amounts of logging, therefore it was configured to log only the LSASS process and exclude the instances from the OSSEC agent and the Virtual Box service.
%This exclude seems to affect other events of type 7 too, even though it should not.
\linej
\begin{lstlisting}[style=xml,caption=Sysmon monitoring with event 10 for lsass reads,captionpos=b]
<ProcessAccess onmatch="include">
	<TargetImage condition="contains">C:\Windows\System32\lsass.exe</TargetImage>
</ProcessAccess>
<ProcessAccess onmatch="exclude">
	<SourceImage condition="contains">C:\Windows\System32\vboxservice.exe</SourceImage>
	<SourceImage condition="contains">C:\Program Files (x86)\ossec-agent\ossec-agent.exe</SourceImage>
</ProcessAccess>
\end{lstlisting}
\linej
After some analysis of these events it was clear that normal accesses could be identified by the \textit{grantedAccess} field. They had a value of \textit{0x3000} (even though these do not happen often) or \textit{0x1000} if the process is \textit{svchost.exe}. The detected malicius programs produced at least one event with a different value on this field.
\linej
\lstinputlisting[style=xml,caption=Rules to detect unusual values of grantedAccess,captionpos=b]{scripts/rules_lsass.xml}
\linej
The first identifies events of type 10 for the LSASS process.
\linej
The second rule triggers if the string \textit{UNKNOWN} is in the field \textit{callTrace}. This occurrence was found during testing of the exploits 4 and 5, that use a reverse TCP shell to connect to their target. This rule causes false positives with the Minidump command and is possible that it would cause false positives on a real network.
\linej
The third and fourth match the normal cases, excluding them from the detection of the last rule.
\linej
The last rule uses a regular expression to match any hexadecimal value of \textit{grantedAccess}, therefore detecting any unusual value, because all normal logs have being identified as normal at this point.
\linej
\linej
The results may change with the size of the database, the status of the system and the version of the system and the programs.
All the exploits used until this point were tested, resulting in half producing unusual values, therefore being detected.

\begin{table}[H]
	\begin{tabularx}{\textwidth}{|l|X|}
		\hline
		\rowcolor{gray!30}
		Exploit & unusual grantedAccess\\ \hline
		1: Local Mimikatz in DC& \cellcolor{green!60}0x143a\\ \hline
		2: Mimikatz from memory in DC& \cellcolor{green!60} 0x143a\\ \hline
		3: Mimikatz with DCSync& \cellcolor{red!60}\\ \hline
		4: DCSync with Kiwi& \cellcolor{red!60}\\ \hline
		5: Hashdump with Meterpreter& \cellcolor{green!60}0x1f3fff\\ \hline
		6: Retrieval of NTDS.DIT with ntdsutil& \cellcolor{red!60}\\ \hline
		7: Storing registry hives with Reg& \cellcolor{red!60}\\ \hline
		8: Dump of LSASS with ProcDump& \cellcolor{green!60}3 events with 0x1fffff\\ \hline
		9: Dump of LSASS with pd& \cellcolor{green!60}
			0x1f3fff \linej
			0x1452 followed by 0x1410, this pair repeated 28 times \linej
			0x1452 \\ \hline
		10: Dump of LSASS with Minidump& \cellcolor{green!60}
			0x1f3fff \linej
			0x1fffff \\ \hline
		11: Dump of LSASS with NinjaCopy& \cellcolor{red!60}\\ \hline
	\end{tabularx}
	\caption{Exploit detection of unusual grantedAccess values}
\end{table}
\linej
Another way to detect unusual access to LSASS is with the event 8 of Sysmon\cite{detection_events}, that reports when a process creates a thread in another process:
\begin{lstlisting}[style=xml]
<CreateRemoteThread onmatch="include">
	<TargetImage condition="image">lsass.exe</TargetImage>
</CreateRemoteThread>
\end{lstlisting}
\linej
As for Wazuh just a rule to detect the event of type 8 and the LSASS executable are enough:
\lstinputlisting[style=xml]{scripts/rules_lsass_event_8.xml}
\linej
Only exploits 1, 2 and 5 were detected with this rule.

\subsection{Mitigation}
Some of the measures for the Golden Ticket attack can be used for this, particularly those about protecting LSASS.
\linej
There are multiple ways to protect the NTDS.DIT file\cite{protect_NTDS}\cite{hood}:
\begin{itemize}
	\item Install the system in multiple volumes with multiple file formats.
	\item Monitor or restrict the ntdsutil command.
	\item Backup and disk encryption.
	\item Restrict access to DCs and AD administrators.
	\item Remove the ability to start/stop the Volume Shadow Copy service from ALL users on the system.
	\item Remove the ability to modify the security settings of the Volume Shadow Copy service from all users except for SYSTEM.
\end{itemize}

\subsection{Conclusion}
%TODO


\cleardoublepage
\chapter{Increment 3}
In this increment we assume the malware has already breached into the system and our role is to detect and mitigate it, as soon as it begins to hold hostage the system or data.
It does not make sense for this project to study intrusion strategies related to ransomware, because there are too many and there is no guarantee they can be detected by an IDS.

\section{Ransomware}
Ransomware is a term used to describe a class of malware that is used to digitally extort victims into payment of a specific fee.
Is not limited to any particular geography or operating system, and can take action on any number of devices\cite{ransomware_oReilly}.
\linej
Once the ransom is paid the attacker provides the instructions to restore the affected resources.
The best guarantee the ransom works is because attackers are interested in keeping the pay rate as high as possible, but as an illegal activity there is no way to ensure is going to free the system and that it will not be affected again in the future (the attacker could install a backdoor).
\linej
\linej
Any fully functional ransomware needs\cite{ransomware_digital_extortion}:
\begin{itemize}
\item Some mean to hold a resource hostage.
\item An anonymous system for exchanging data with the affected system.
\item A ransom payment method that can not be traced back to the digital extortionist.
\end{itemize}
\linej
There are two basic forms of ransomware, that are not mutually exclusive\cite{ransomware_digital_extortion}\cite{ransomware_oReilly}:
\begin{itemize}
\item Cryto: They encrypt, obfuscate, or deny access to files.
Depending on the target it may search for specific directories or file extensions.
In most scenarios the ransomware does not affect the critical system files or functionalities and does not deny access to the system.
Normally is more sophisticated and targets systems with more robust security than locker ransomware.
\item Locker: They restrict access or lock users out of the system.
Usually the affected system is not able to perform basic tasks, even for payment, which results in a preference of payment voucher systems.
Is easy to recover from but also to implement.
\end{itemize}
\linej
Both can take extra steps, like exfiltrating data or take down any antimalware detected software.
Infected systems are often used by attackers to spread the malware, for example across the network.
The cybercriminal wants the victim to notice as soon as the attack is done, to get paid as fast as possible, and the most common method is sending direct messages or changing the desktop background.
\linej
\linej
The next image shows a simplification of the steps of a ransomware attack.
We only care for the detection of the ``Destruction'' segment, but that does not mean we can ignore the whole picture.
\begin{figure}[H]
	\centering
	\includegraphics[width=\textwidth]{figuras/anatomy_of_a_ransomware_attack.png}
	\caption{Anatomy of a ransomware attack\cite{ransomware_oReilly}}
\end{figure}

\subsection{State of ransomware}
Ransomware is important for this project because it has gained much relevance in cybersecurity in the last years.
We think understanding it better is the first step to develop detection for it.
\linej
\linej
The main problems for studying ransomware is that is illegal and sometimes is hard to gather information.
For example most of the attacks are not reported and ransomware attacks are often analyzed offline and using auditing tools like decompilers because most of the time there is no source code\cite{ransomware_digital_extortion}.

\subsubsection{The growth of ransomware}
Next there are some estimations about the ransomware economy\cite{ransomware_economy}:
\begin{itemize}
	\item There are more than 6000 dark web marketplaces selling ransomware, with 45000 products listed.
	\item The ransomware marketplace on the dark web has grown in 2502\% from 2016 to 2017.
A major antimalware company states a 90\% increase in detection for business and a 93\% for individuals, in the same year\cite{ransomware_malwarebytes}.
	\item Some sellers of ransomware are making more than 100 thousand USD per year, just by retailing ransomware, when a legitimate software developer makes 30\% less.
\end{itemize}
\linej
Often ransomware includes remote control software to allow the remote execution of commands.
They often check if the server (certain ips or domains) can be reach before starting the process, waiting for a possible connection in the future if not.
Because these domain addresses are always resolving to new host ips, the criminal enterprises can regularly move around the Internet in relative safety, as they will always know their malcode can speak to them.
Keeping this Command and control server functional and anonymous may be hard and expensive depending on the approach used by the attacker.
They have evolved to use algorithms to generate a list of thousands of domains dynamically.
%Particularly DNS scans for new suspicious entries is advised.
\linej
The flaw of this approach is that can trigger behavioural alerts and reduces the scope of the attacks.
There is no need to initially contact the Command and control server, even for crypto ransomware because the public RSA key can be downloaded with the malware.
In some cases the attacker can command the malware to delete itself to avoid leaving evidence for security proffesionals\cite{ransomware_oReilly}\cite{ransomware_digital_extortion}.
\linej
\linej
The fundamental reason why this market exists is because the victims are willing to pay.
Is hard to know specifics because is estimated that most of the times these attacks are not reported (fewer are prosecuted and fewer are sentenced), but the mean crypto ransom is 300 USD per computer\cite{ransomware_digital_extortion}.
%\begin{figure}[H]
	%\centering
	%\includegraphics[width=\textwidth]{figuras/ransomware_pay_chance.png}
	%\caption{Statistics on the chance to pay ransomware\cite{ransomware_economy}}
%\end{figure}
\linej
\linej
Unlike many other forms of cyberattacks, ransomware can be quickly and brainlessly deployed with a high probability of profit nowadays.
It has been a revelant type of malware since its beginning 23 years ago, but in the last years its economy has grown hugely, mainly because\cite{ransomware_digital_extortion}\cite{ransomware_economy}:
\begin{itemize}
	\item Bitcoin and Tor: for pseudo-anonymous activities.
	\item Proliferation of service providers: anyone can get into the ransomware business because technical knowledge is not needed for every step of the process.
	\item Lack of fundamental security controls: such as backups, penetration testing, patching, etc.
\end{itemize}
\linej
Bitcoin allows money to be transferred in a way that makes it nearly impossible for law enforcement to follow the money trail.
Anyone can set up a free and unrestricted Bitcoin wallet address without needing any approval from financial institution, regulations, or dealing with providing evidence and proofs of identity, taxation, evidence of residence, and so on.
Anybody can see the Bitcoin transactions or the flow of the cryptocurrency from address to address in the blockchain, making it possible to backtrack it to a real identity.
Unfortunately cybercriminals use to mix transactions to make them very hard to follow, this is what Bitcoin mixing services are for.
\linej
Tor is an anonymity network, that can be used to mask illicit activities and can be used just by running a program.
\linej
Neither provides perfect anonymity, but both are very easy to use, which has lowered the risk and barrier to entry for ransomware perpetrators.
The requirement for ransoms to be paid over the Tor network has ensured there's no centralized endpoint to investigate with traditional geo-based law enforcement approaches\cite{ransomware_digital_extortion}.
\linej
\linej
Due to these innovations, the underground ransomware economy is now an industry that resembles commercial software. It can be divided into sections like: development, support, distribution and even help desks. This market can be divided into tiers\cite{ransomware_economy}:
\begin{itemize}
	\item Authors: They can be responsible for the creation of the malware (including frameworks) and training and support on them. In the current state of the market they can just remain as authors, without ever running the malware in other computers. The cost is based on how customized the code is for a particular target.
	\item RaaS: Ransomware-as-a-Service borrows from the Software-as-a-Service model. RaaS is designed to make ransomware available to even novice criminals. It provides technical and step-by-step information on how to launch the ransomware attack with the purchased software. The most sofisticated have a platform for checking the current status of the attack. In some cases the ransom is split among the members of the supply chain.
	\item Distributors: A high profit/risk tier. They can distribute it themselves by spam campaigns, social engineering, targeted hacks or exploit kits. They can also leverage RaaS.
\end{itemize}
\linej
Splitting the work into different modules makes the ransomware market more complex and competitive.
It encourages role specialization (increasing quality and reducing risk for authors), it makes it easier to enter (expanding the market) and the modularity increases the quantity of different products (making the malware change faster).
\linej
Of course making it easier to enter the market is a double edge sword, because it makes it easier for police to investigate.
Defenders have the inherent advantage of interrupting the entire attack if they can break or interrupt any link of the chain.
\linej
This market can be simplified into the next image:
\begin{figure}[H]
	\centering
	\includegraphics[width=\textwidth]{figuras/ransomware_chain.png}
	\caption{Ransomware supply chain and economy tiers\cite{ransomware_economy}}
\end{figure}
\linej
As long as the victims keep paying the ransom this trend will continue and the specialization will increase, resulting in more and bigger ransomware incidents.
%More creative forms of ransomware are sure to emerge over the years, like encrypting the master boot record\cite{ransomware_digital_extortion}.
The more ransomware is used the more secure the systems are against it, therefore in the future is also safe to assume that it would be more economical for criminals to use cryptominers or credential stealing instead\cite{ransomware_malwarebytes}.
\linej
\linej
Is also interesting to consider how ransomware and other attacks may take place if the main attack fails.
For example even if a ransomware attack fails it should be possible to use a distributed denial of service attack at greater cost and with lesser benefit.
A increase of malware as a distraction is expected in the future as the use and security of backups improves\cite{ransomware_economy}.

\subsubsection{Ransomware targets}
The more important the data the more money the victim is asked for.
Generally the bigger the business the better security it has and the more money cybercriminals can extort.
\linej
But the value of data may change from person to person, therefore there is no absolute best target for ransomware.
In the last years there has been an increase on the number of personal data sold in the dark web, which includes all kinds of data.
Cybercriminals can put pressure on the victim not only with direct ransom but also with the threat of selling the data.
\linej
In practice certain type of targets seems to rate higher on cybercriminals' priority lists\cite{ransomware_digital_extortion}:
\begin{enumerate}
	\item \textbf{Healthcare}: Apparently nothing sells as good on the black market as private healthcare records.
		Medical records do not lose value over time and they contain not only the person's medical history, but offers a full set of sensitive data that can be exploited in more ways than one (credit card numbers, social security numbers, banking credentials, e-mail IDs and employment history).
Cybercriminals use this currency to spread infections by phishing attacks, data fraud, and theft of medical histories.
	\item \textbf{Manufacturing}: Including businesses like automotive, electronics, textile and pharmacy.
The nature of the chemical and automotive businesses makes certain aspects horrifying, however, cybercriminals' motivation is predominantly financial, as they attack corporations not with the intention of mass murder, but to obtain valuable data and lucrative sensitive information.
	\item \textbf{Financial services}: The accessibility of payment methods and globally spread banking services whose main purpose is customer convenience will keep the financial industry high on the list. Attackers only need to impersonate or trick the victim in order to gain access to an account.
	\item \textbf{Goverment agencies}: They hold all types of personal and confidential data. If a goverment agency stops working it can affect many people.
	\item \textbf{Transportation}: Usually ransomware attacks set a time limit to pay the ransom, but in transportation real life sets the time limit. A higher effortless pressure results in less advance malware needs, meaning that transportation businesses could be effectively extorted with locker ransomware or denial of service attacks.
As in previous cases there is also interest in personal data and financial accounts.
	\item \textbf{Home users}: Ransomware is one of the best effective malware against personal computing users, who are considerably not experienced in cybersecurity.
This has increased due to the rise of smartphones and IoT devices.
Most users either don't use backups, they are not done often enough or they are stored in the same computer.
\end{enumerate}
\linej
In the end the target is always the people and the cybercriminal can either create a situation where is cheaper to pay or to gamble on the feelings (fear, shame, guilt, etc) of the victim.
\begin{figure}[H]
	\centering
	\includegraphics[width=.7\textwidth]{figuras/credit_cards_for_sale.png}
	\caption{Credit cards for sale\cite{ransomware_digital_extortion}}
\end{figure}

\subsection{Common patterns in crypto ransomware}
64\% of ransomware detected in 2016 was crypto ransomware\cite{ransomware_digital_extortion}, but it was not until 2013 that cybercriminals came back to it as their primary source of ransomware income\cite{ransomware_oReilly}.
The are many different crypto ransomware products, but we take interest in those who encrypt the files because, they are the ones that make the most impact on them.
If the encryption is done with strong algorithms it should not be possible to decrypt them in a lifetime (without knowing the decryption key).
\linej
\linej
This means suspicious activity regarding backups and files is common among the different crypto ransomware variants.
We assume the attacker prefers to execute the attack as fast as possible because:
\begin{itemize}
\item Once the encryption process starts is probably going to be detected very fast, either by an user or by an antimalware detector.
\item The only advantage of waiting is keeping the noise down.
\item If the encryption is only done on key files it should be very fast.
\item The system can be scanned quite silently before starting the encryption.
\end{itemize}
\linej
The most eficient way to encrypt the files is to use a symmetric algorithm (like AES-256), ensuring very strong and fast encryption even for big files.
The symmetric key is the same for encryption and decryption, and in this type of attack it holds the entire value of the ransom: once the ransom is paid the symmetric key is send to the victim to decrypt the files.
\linej
Obviously this key has to be kept secret by the attacker, therefore is ciphered with a public-key asymmetric algorithm (like RSA), allowing the attacker to transfer it without any risk.
Only the people with the private key should be able to decipher the symmetric key, therefore only the attacker can decipher it.
The public key is either downloaded with the malware or after the installation.
\linej
A major disadvantage of symmetric key encryption is that, in some cases, the key can be retrievered from RAM using special software.
Fortunately the DRAM in most systems is still live for anywhere from a few seconds to a few minutes after power loss, allowing a shutdown to stop the attack and still being able to recover from it.
However these techniques are unlikely to work with functions that avoid leaving trails in memory\cite{ransomware_oReilly}.

%\linej
%\linej
%Most ransomware families use the built-in Windows Crypto API to handle the encryption.
%While this is not a process that can necessarily be blocked, it is a process on which a security team can be alerted.
%The problem is not detecting its use, but rather not spawning false positives\cite{ransomware_oReilly}.

\subsubsection{File encryption and backup deletion}
This example shows how easily removal of local backups and encryption of files can be done with the right tools.
\linej
In this scenario the attacker uses AES for the files and ciphers the key passphrase with a RSA public key.
Both algorithms are supported by many tools: GnuGG, OpenSSL, custom DLLs and programming languages like PowerShell or Python.
OpenSSL was chosen because it works the same on Windows and GNU/Linux and is easy to use and install.
\linej
\linej
The first step would be the generation of the pair of RSA keys.
The private key stays in the computer of the attacker and the public key is stored into the victim's.
In this case the files are named ``keys.pem'' and ``public.pem'' respectively and the length of the key is 1024 bits.
The first file includes both private and public keys.
The second file is generated by the second command, which extracts the public key out of the combined file ``keys.pem''.
\begin{lstlisting}[style=PS,keywordstyle=\color{black}]
openssl genrsa -out keys.pem 1024
openssl rsa -in keys.pem -pubout -out public.pem
\end{lstlisting}
\linej
The next PowerShell script creates a random passphrase of letters, numbers and some special characters, which is used later for the symmetric key by the AES implementation of OpenSSL.
This passphrase is ciphered with the public RSA key and saved to the file ``passphrase.txt''.
Then script loops over all the files in \textit{C:/temp/} with the \textit{.ps1} extension, encrypting them to AES-256 (in block cipher mode) and deleting the original file when done.
The \textit{openssl} variable is just for convenience:
\lstinputlisting[style=PS]{src/ransomware_basic.ps1}
\linej
The passphrase file can be deciphered with the private key, getting the random passphrase used in the script:
\begin{lstlisting}[style=PS,keywordstyle=\color{black}]
openssl rsautl -decrypt -in passphrase.txt -inkey keys.pem
\end{lstlisting}
\linej
Decrypting the files can be done using the same passphrase, because AES is a symmetric algorithm.
This example assumes the same value for the variables as in the encryption process:
\lstinputlisting[style=PS]{src/ransomware_decrypt.ps1}
\linej
%TODO detection

\linej
\linej
Backups offer effective mitigation against crypto ransomware attacks in most cases.
There are many scenarios between distributed backups in servers accross the world and just having local backups.
Their ability to overcome ransomware attacks depends on multiple factors like: the exact malware, the distribution of the backup system, the security of the backup system, etc.
\linej
In this case we assume the target only has local backups that are managed with the Shadow Copy service, which allows files to be copied even when in use.
Some known ransomware like Locky and Cerber try to delete existing shadow copies before encrypting the files\cite{ransomware_oReilly}:
\begin{lstlisting}[style=PS]
C:\Windows\system32\vssadmin.exe delete shadows /all /quiet
C:\Windows\system32\wbem\wmic.exe shadowcopy delete
\end{lstlisting}
\linej
%TODO detection


%TODO Mitigation
%%MITIGATION
%%	Prevent the ransomware from accessing the Windows Registry
%%		Ransomware families use the registry to maintain persistence through reboots and to disable system restore on the victim machine.
%%		Administrators can disable writing to the registry, or at least to certain registry keys, using Windows Resource Protection.
%%	Cyber insurance
%%		in some cases
%%			costs like
%%				notifications costs to data breach victims
%%				legal defense costs
%%				forensics and investigation costs
%%	Decoy resources
%%	Security awareness and education
%%		particularly on e-mails and web browsing
%%		everyday documents: pdf, office macros
%%	Fundamental security controls
%%		backups, penetration testing, patching and software updates, firewall
%%	Restrictions to unnecessary services and software
%%		eg powershell or Tor
%%		for certain directories
%%		only allow digitally signed software
%%		deny access to vssadmin and wmic to avoid the deletion of shadowcopies
%%	Removing unused devices
%%		This action prevents possible further infection spreading and it should be applied to varied physical
%%		devices such as mapped drives, USB storage devices or memory sticks, smartphones, and cameras.
%%		All writeable devices should be removed from a station when not in use.
%%	File exchange management
%%		Businesses are based on sharing and it is impossible to do a job unless certain files are shared and
%%		worked on collaboratively. Once the process of file sharing becomes a routine, security gets on
%%		wobbly feet. To keep the filesystem safe, organizations should establish best practices for sharing
%%		data and files in a safe and secure manner. An effective way to minimize risks is application of digital
%%		signatures.
%%	Response plan development
%%		Prompt action is almost impossible if actors are unprepared. At a moment of crisis, decision-making
%%		can be weak and if the organization does not have an incident response plan at hand, the consequences
%%		of the infection can exacerbate. Developing a solid plan for fighting malware infections is the first
%%		mitigation task that should be completed by the responsible business leaders in the organization.
%%		A well-developed plan will prevent the company falling victim to payment at the moment of panic
%%		happening in the coal-and-ice a few hours after the attack.
%%		Several behaviors are key-avoiding payment and rather immediately referring to the incident plan,
%%		disconnecting infected units from the network, employing company digital security teams in line with
%%		the incident plan, keeping records of the information, and notifying law enforcement authorities.
%%
%%		Quick five-step guide for businesses under attack
%%			Disabling sync features:
%%				Enabled syncing features makes it easier for offenders to instigate
%%				attacks that will overwrite files, especially when they use crypto ransomware. By disabling sync
%%				features you can prevent targeting data in the cloud.
%%			Removing malware from the affected devices:
%%				Running a full scan is vital to remove the
%%				malware from the infected devices, including synced or mapped drives. Many operating systems
%%				come with a built-in basic anti-malware tool, which is not 100% effective. An advanced anti-
%%				malware software tool is the ultimate protective solution.
%%			File recovery:
%%				File recovery depends on the system version in use. For example, Windows
%%				users can recover files through the File Restore or System Protection functions.
%%			Blocking the payment transaction:
%%				Under certain circumstances, the payment transaction can
%%				be blocked, even if you have already started the payment process. This is throughble when the
%%				files have been successfully recovered without using the help provided by the attackers.
%%			Contacting law enforcement and reporting the crime:
%%				Getting in touch with the relevant
%%				cybercrime authority in the country is important not only for taking action in the concrete case,
%%				but also for predicting future criminal behavior and for undertaking protective measures to
%%				prevent similar attacks to other victims. Sending a report to the relevant software authorities is
%%				also recommended. By clicking the Send Report button you protect yourself and your
%%				organization, show solidarity, and contribute to a great business practice in building effective
%%				advanced anti-threat solutions.


%TODO Conclusion



\cleardoublepage
\chapter{Conclusions and additions}
\section{Conclusion}
This project has shown how Wazuh can be used to improve the cybersecurity of a system, in some cases without the need of expert knowledge in scripting or pentesting.
\linej
Wazuh has been under development for some years and there are still parts that lack basic funcionality.
This project has tried to show and fix some of them, without changing the source code in order to have more time for other tasks.
\linej
\linej
The main use of a HIDS should be to gather and process data in order to detect security threats.
Wazuh provides a set of features that make it a very good tool for monitoring system behaviour, but not as good in other aspects.
\linej
Stopping the threats can be too much for this kind of system, there are obvious limitations that make it impossible to be as effective as a local antimalware software.
Trusting user scripts for defensive security is very risky and bound to fail sooner or later.
\linej
\linej
Even though in this project only the server runs GNU/Linux is clear from its management and the documentation that Wazuh is less suited for Windows systems.
Many workarounds and third-party tools are needed because of this.
Therefore it is assumed that the project could have advanced faster if it were on GNU/Linux security instead of Windows.
\linej
\linej
Using an incremental methodology proved to be a right choice because there were many desired features that were left behind because lack of time.
It also fits very well with the work flow in this kind project, with very simple development.
The essential requirements and increments were satisfied and a bit more depth than initially planned was applied in many cases, resulting in what we consider a higher quality for the product.

\section{Additions}
Multiple ways to detect and act against malware have been analyzed in this project, but this is just the tip of the iceberg and much work remains to be done.
There are many features and research that could improve the current state of Wazuh as a reliable cybersecurity tool:
\begin{itemize}
	\item All the ideas left behind in this project due to lack of time. They are explained in the exclusions part of the scope section \ref{exclusions}.
	\item Ability to set real variables in the rules, instead of the current ones that are actually constans because they need a restart of the service to be updated.
Unfortunately this can also result in a loss of performance for processing the rules.
	\item A functional database system for all kinds of data, instead of the current CDB that only is useful for IP addresses. This would solve the problem of just monitoring a value, like the case of the free space of the backup volume \ref{free_space_volume}.
	\item Having more decoders for common monitoring commands in Windows, instead of having to write your own or parse the output of the command with an script.
	\item Executing remote commands with direct feedback to the manager, instead of having to use workarounds like writing to a monitored log.
	\item Active response directly with PowerShell, instead of having to use CMD or SSH. This is in development.
	\item Active response with dynamic arguments from the alert, instead of processing the alert manually with a local script and using another program (in this case SSH) for the remote execution.
	\item Option to have only a frequency of 1 in composite rules, which is not such a rare need, instead of needing workarounds for it.
	\item Integration with local antimalware software. This has been done using APIs and custom scripts with some software (like VirusTotal) but not with others as basic as Windows Defender.
	\item Integration with other IDSs. This can provide detection features based on behaviour or network data that Wazuh can not gather on its own.
	\item Further research and development in any of the parts treated in this project.
\end{itemize}




\appendix
\cleardoublepage
\chapter{Glossary}
%TODO añadir terminos importantes o abstractos (no solamente siglas)
	%payload
	%signature
%dar por hecho cosas como AES, TLS, PMBOK, PMI ?
%quitar del texto normal las explicaciones ?
	%done="$(awk -F '{' '/\\textbf\{/ {print $2}' glossary.tex  |cut -d '}' -f1 |tr '\n' '|')"; \grep -E '\b[[:upper:]][[:upper:]]+\b'  *.tex |cut -d ':' -f2  | \grep -v '^[[:space:]]*%' |\grep -Eo '\b[[:upper:]][[:upper:]]+\b' | sort -u |grep -vE "UNIVERSITY|SANTIAGO|COMPOSTELA|ESCOLA|TÉCNICA|SUPERIOR|DE|ENXEÑARÍA|TODO|LARGE|STATE|JSON|XML|FAQ|UDP|TCP|IP|ARD|CA|CRD|GNU|GPL|OF|PA|RD|SL|TF|SEC|WBS|KERB|REQUEST|CPU|DIT|HOST|NTDS|OR|PS|RAM|RNF|RNO|RYES|SYSTEM|UNKNOWN|WIN|WMIC|${done:0:-1}"


\textbf{AD}: Active Directory. The directory domain of Windows systems, thought it can be also used by GNU/Linux with Samba.
\linej
\linej
\textbf{API}: Application Program Interface. Is a set of subroutines, functions and procedures from a library to be used by other software.
\linej
\linej
\textbf{AWK}: Programming language created by Alfred \textbf{A}ho, Peter \textbf{W}einberger, and Brian \textbf{K}ernighan, used mostly for string parsing.
\linej
\linej
\textbf{AES}: Advanced Encryption Standard. Popular symmetric encryption algorithm with different key lengths.
\linej
\linej
\textbf{CDB}: Short for constant database. File format and library for item creation and reading in a database at fast speeds.
\linej
\linej
\textbf{DC}: Domain Controller. In this case a server that runs part of an Active Directory domain. The main DC is named Primary Domain Controller.
\linej
\linej
\textbf{DLL}: A Dynamic Link Library file. It is a grouping of code or data for programs of the system. It is in a separate file for easier management or better performance.
\linej
\linej
\textbf{ELK}: Elasticsearch, Logstash and Kibana. Stack used for Wazuh to gather and transform data.
\linej
\linej
\textbf{Golden Ticket}: Forged TGT that normally provides access as administrator of the AD for 10 years.
\linej
\linej
\textbf{GPDR}: General Data Protection Regulation. Regulation in EU law on data protection and privacy for all individuals within the European Union and the European Economic Area. In practice in this project means law protected files against changes.
\linej
\linej
\textbf{HIDS}: Host-based Intrusion Detection System.
\linej
\linej
\textbf{ICS}: Industrial Control System. They are control systems for critical tasks. Normally they are used for industrial control, but in this project we consider any purpose, like data analysis.
\linej
\linej
\textbf{IDS}: Intrusion Detection System. Mitigates the damage of intrusions, providing passive protection by alerts.
\linej
\linej
\textbf{IPS}: Intrusion Prevention System. Minimizes the chance of intrusions, providing active protection by actions.
\linej
\linej
\textbf{KDC}: Key Distribution Center. Service that handles the Kerberos requests. It runs in a DC.
\linej
\linej
\textbf{Kerberos}: Computer network authentication protocol that uses tickets to allow computers over a network to authenticate in a secure manner. Windows 2000 and later uses Kerberos as its default authentication method.
\linej
\linej
\textbf{KRBTGT}: Kerberos super-administrator account, used for encrypting all the authentication tokens for the DC. Is hidden, local, can not be deleted, neither the name changed.
\linej
\linej
%\textbf{PMBOK}: Project Management Body of Knowledge. Is a set of standard terminology and guidelines for project management.
%\linej
%\linej
%\textbf{PMI}: Project Management Institute. Nonprofit proffesional organization for project management.
%\linej
%\linej
\textbf{Metasploit}: Penetration testing framework. There are Windows and GNU/Linux versions.
\linej
\linej
\textbf{Meterpreter}: Meterpreter is a Metasploit attack payload that provides an interactive shell from which an attacker can explore the target machine and execute code. Meterpreter is deployed using in-memory DLL injection.
\linej
\linej
\textbf{Mimikatz}: Program to extract authentication data or generate forged authentication tickets. In this project we use it for extracting the KRBTGT hash and generating Golden Tickets.
\linej
\linej
\textbf{LSA}: Short for \textbf{L}ocal \textbf{S}ecurity \textbf{A}uthority Subsystem Service. Process in Microsoft Windows operating systems that is responsible for enforcing the security policy on the system. It verifies users logging on to a Windows computer or server, handles password changes, and creates access tokens.
\linej
\linej
\textbf{LSASS}: Local Security Authority Subsystem Service. Process in Microsoft Windows operating systems that is responsible for enforcing the security policy on the system. It verifies users logging on to a Windows computer or server, handles password changes, and creates access tokens.
\linej
\linej
\textbf{NIDS}: Network-based Intrusion Detection System.
\linej
\linej
\textbf{OSSEC}: \textbf{O}pen \textbf{S}ource HIDS \textbf{SEC}urity. Is an HIDS solution with detection based on rules and decoders.
\linej
\linej
\textbf{OU}: Organizational Unit. They are a type of group structure for AD.
\linej
\linej
\textbf{Samba}: Is a set of programs for interoperability between Linux and Windows.
\linej
\linej
\textbf{SMB}: Server Message Block protocol for sharing files, printers, communications, etc in Microsoft systems. Some of its services can be accessed with Samba.
\linej
\linej
\textbf{Shadow Copy}: Windows technology for copying files or volumes when they are in use.
\linej
\linej
\textbf{SID}: Security Identifier. Unique value used to differentiate security elements or groups in Windows Systems.
\linej
\linej
\textbf{TGT}: Ticket Granting Ticket. This ticket is encrypted with the KDC key and is used for request to the KDC one or more TGS.
\linej
\linej
\textbf{TGS}: Ticket Granting Service. This ticket is encrypted with the service key and is used to authenticate against a service.
\linej
\linej
\textbf{TLS}: Transport Layer Security. Cryptography protocol designed to provide communications security over a computer network with hybrid (symmetric and asymmetric) cryptography.
\linej
\linej
\textbf{YARA}: Tool that does pattern/string/signature matching, with great in performance, results and easiness to write rules.
\linej
\linej


%\cleardoublepage
%\include{capitulos/technical-manual}

\cleardoublepage
%Manuais de usuario: incluirán toda a información precisa para aquelas persoas que utilicen o Sistema: instalación, utilización, configuración, mensaxes de erro, etc. A documentación do usuario debe ser autocontida, é dicir, para o seu entendemento o usuario final non debe precisar da lectura de outro manual técnico.
\chapter{User Manual}
%TODO


\cleardoublepage
%done="$(\grep -oh 'ref{lst:\w*}' *.tex | cut -d '{' -f2 | tr '\n' '|')"; \grep -oh 'label={lst:\w*}' code-configuration.tex | \grep -vE "${done:0:-1}"

\chapter{Programming and configuration code}
%\lstinputlisting[style=xml,caption=,captionpos=b]{src/}
%\lstinputlisting[style=PS,caption=,captionpos=b]{src/}

\section*{Workaround to reach the minimum frequency with active response}
\lstinputlisting[style=xml,caption=Active response configuration in \textit{/var/ossec/etc/shared/default/agent.conf} for the basic DLL alerts,captionpos=b,label={lst:freq_active_response}]{src/freq_workaround/ossec_mimikatz_duplicate_log.conf}
\linej
\lstinputlisting[style=sh,caption=Bash script in \textit{/var/ossec/active-response/bin/} to write to a log twice per DLL alert received,captionpos=b]{src/freq_workaround/dll_duplicate_log.sh}
\linej
\lstinputlisting[style=xml,caption=Wazuh rules for detection of DLLs with Sysmon and detection of the special duplicate entries in the custom log,captionpos=b]{src/freq_workaround/rules_event_7_mimikatz_duplicate_log.xml}
\linej

\section*{Detection of the use of the TGT with klist}
\lstinputlisting[style=xml,caption=Remote command configuration in \textit{/var/ossec/etc/shared/default/agent.conf} for the klist script,captionpos=b,label={lst:klist_wodle}]{src/ticket_detection/klist_wodle.xml}
\linej
\lstinputlisting[style=PS,caption=Script to scan and parse to JSON the tickets in the cache,captionpos=b,label={lst:klist}]{src/ticket_detection/klist.ps1}
\linej
\lstinputlisting[style=PS,caption=Way to get the MaxTicketAge from the Group Policy,captionpos=b,label={lst:klist_getMaxTicketAge}]{src/ticket_detection/report.ps1}
\linej

\section*{Detection of suspicious logins}
\lstinputlisting[style=PS,caption=Distributed brute force logins changing the ip address for the internal network,captionpos=b,label={lst:distributed_logins_winrs}]{src/logins/distributed_logins_winrs.ps1}
\linej
\lstinputlisting[style=xml,caption=Wazuh rules for checking OU logins outside of usual hours,captionpos=b,label={lst:OU_rules}]{src/logins/rules_OU.xml}
\linej
\lstinputlisting[style=xml,caption=Remote command configuration in \textit{/var/ossec/etc/shared/default/agent.conf} to find logins on unusual hours for the OU users,captionpos=b,label={lst:OU_wodle}]{src/logins/OU_wodle.xml}
\linej
\lstinputlisting[style=PS,caption=Remote script to find logins on unusual hours for the OU users,captionpos=b,label={lst:OU}]{src/logins/OU.ps1}
\linej

\section*{File monitoring}
\lstinputlisting[style=xml,caption=Windows Defender's Controlled Folder Access rules,captionpos=b,label={lst:windows_defender}]{src/file_monitoring/rules_windows_defender.xml}
\linej
\lstinputlisting[style=xml,caption=Syscheck local configuration in the \textit{ossec.conf} file on the agent,captionpos=b,label={lst:syscheck_agent}]{src/file_monitoring/syscheck_agent.xml}
\linej
\lstinputlisting[style=xml,caption=Syscheck rules for crypto ransomware detection,captionpos=b,label={lst:syscheck_rules}]{src/file_monitoring/rules_syscheck.xml}
\linej
\lstinputlisting[style=xml,caption=Syscheck rules for crypto ransomware detection with more deletion events required,captionpos=b,label={lst:syscheck_rules_2}]{src/file_monitoring/rules_syscheck_2.xml}
\linej
\lstinputlisting[style=xml,caption=Windows File Auditing rules for crypto ransomware detection,captionpos=b,label={lst:file_audit}]{src/file_monitoring/rules_file_audit.xml}
\linej

\lstinputlisting[style=xml,caption=Wazuh rules for detecting suspicious file extensions,captionpos=b,label={lst:extensions_rules}]{src/file_monitoring/detect_encrypted/rules_extensions_only_modified.xml}
\linej
\lstinputlisting[style=xml,caption=Wazuh rules for detecting encrypted files,captionpos=b,label={lst:trid_rules}]{src/file_monitoring/detect_encrypted/rules_trid.xml}
\linej
\lstinputlisting[style=xml,caption=Active response configuration in \textit{/var/ossec/etc/shared/default/agent.conf} for the Trid CMD,captionpos=b,label={lst:trid_ossec}]{src/file_monitoring/detect_encrypted/ossec.conf}
\linej
\lstinputlisting[style=PS,caption=CMD script only for executing the Trid PowerShell script,captionpos=b,label={lst:trid_cmd}]{src/file_monitoring/detect_encrypted/trid.cmd}
\linej
\lstinputlisting[style=PS,caption=PowerShell script that writes a Trid log entry for each modified file in the last 2 minutes in the folder,captionpos=b,label={lst:trid_ps1}]{src/file_monitoring/detect_encrypted/trid.ps1}
\linej

\section*{Registry monitoring and active response for Windows Defender}
\lstinputlisting[style=xml,caption=Sysmon rules for monitoring the Windows registry,captionpos=b,label={lst:sysmon_events_registry}]{src/registry/sysmon_events_registry.xml}
\linej
\lstinputlisting[style=xml,caption=Wazuh rules for registry monitoring,captionpos=b,label={lst:rules_registry}]{src/registry/rules_registry.xml}
\linej
\lstinputlisting[style=xml,caption=Configuration for 3 active response commands for registry rules in \textit{/var/ossec/etc/ossec.conf},captionpos=b,label={lst:active_response_registry}]{src/registry/active_response_registry.xml}
\linej
\lstinputlisting[style=PS,caption=CMD script only for executing another PowerShell script,captionpos=b,label={lst:change-registry-value_0}]{src/registry/change-registry-value_0.cmd}
\linej
\lstinputlisting[style=PS,caption=PowerShell script for setting registry entries to 0,captionpos=b]{src/registry/change-registry-value_0.ps1}
\linej
\lstinputlisting[style=PS,caption=CMD script only for executing another PowerShell script,captionpos=b]{src/registry/change-registry-value_1.cmd}
\linej
\lstinputlisting[style=PS,caption=PowerShell script for setting registry entries to 1,captionpos=b]{src/registry/change-registry-value_1.ps1}
\linej
\lstinputlisting[style=PS,caption=CMD script only for executing another PowerShell script,captionpos=b]{src/registry/create-registry-value_1.cmd}
\linej
\lstinputlisting[style=PS,caption=PowerShell script for creating registry entries with value 1,captionpos=b,label={lst:create-registry-value_1}]{src/registry/create-registry-value_1.ps1}
\linej
\lstinputlisting[style=PS,caption=CMD script only for executing another PowerShell script,captionpos=b]{src/registry/delete-registry.cmd}
\linej
\lstinputlisting[style=PS,caption=PowerShell script for deleting registry entries,captionpos=b,label={lst:delete-registry}]{src/registry/delete-registry.ps1}
\linej

\section*{Backup deletion}
\lstinputlisting[style=xml,caption=Sysmon configuration for monitoring vssadmin,captionpos=b,label={lst:vssadmin_sysmon}]{src/backup_deletion/sysmon_vssadmin.xml}
\linej
\lstinputlisting[style=xml,caption=Wazuh rules for the detection of backup deletion with vssadmin,captionpos=b,label={lst:vssadmin_rules}]{src/backup_deletion/rules_vssadmin.xml}
\linej

\lstinputlisting[style=xml,caption=Wazuh rules for the detection of suspicious wbadmin and bcdedit commands,captionpos=b,label={lst:other_backup_rules}]{src/backup_deletion/rules_other_backup.xml}
\linej
\lstinputlisting[style=xml,caption=Sysmon configuration for monitoring vssadmin,captionpos=b,label={lst:other_backup_sysmon}]{src/backup_deletion/sysmon_other_backup.xml}
\linej

\lstinputlisting[style=xml,caption=Remote command configuration in \textit{/var/ossec/etc/shared/default/agent.conf} for the free space script,captionpos=b,label={lst:free_space_wodle}]{src/backup_deletion/space_volume/wodle_free_space.xml}
\linej

\lstinputlisting[style=PS,caption=PowerShell script to manage the free space in the backup volume and report in JSON,captionpos=b,label={lst:free_space_ps1_simple}]{src/backup_deletion/space_volume/free_space_simple.ps1}
\linej
\lstinputlisting[style=xml,caption=Wazuh rules for processing the output of the free space remote command,captionpos=b,label={lst:free_space_rules_simple}]{src/backup_deletion/space_volume/rules_free_space_simple.xml}
\linej

\lstinputlisting[style=PS,caption=PowerShell script to get the free space in the backup volume and parse its output to JSON,captionpos=b,label={lst:free_space_ps1}]{src/backup_deletion/space_volume/free_space.ps1}
\linej
\lstinputlisting[style=xml,caption=Configuration in \textit{/var/ossec/etc/ossec.conf} for the free space detection,captionpos=b,label={lst:free_space_ossec_configuration}]{src/backup_deletion/space_volume/ossec.conf}
\linej
\lstinputlisting[style=sh,caption=Bash script in \textit{/var/ossec/active-response/bin/} to compare the free space value with the previous one and update the storage and log files,captionpos=b,label={lst:free_space_sh}, stringstyle=\color{black}]{src/backup_deletion/space_volume/free_space.sh}
\linej
\lstinputlisting[style=xml,caption=Wazuh rules for processing processes for the detection of the increase of the free space,captionpos=b,label={lst:free_space_rules}]{src/backup_deletion/space_volume/rules_free_space.xml}
\linej

\section*{Active response against crypto ransomware}
\lstinputlisting[style=PS,caption=PowerShell script to stop and mitigate the execution of crypto ransomware,captionpos=b,label={lst:active_response_crypto_ransomware_ps1}]{src/active_response_crypto_ransomware/stop.ps1}
\linej
\lstinputlisting[style=xml,caption=Configuration in \textit{/var/ossec/etc/ossec.conf} for the integrator custom script,captionpos=b,label={lst:active_response_crypto_ransomware_ossec}]{src/active_response_crypto_ransomware/ossec.conf}
\linej
\lstinputlisting[style=python,caption=Python script in \textit{/var/ossec/integrations/} to parse the triggering alert and execute a remote script with ssh,captionpos=b,label={lst:active_response_crypto_ransomware_py}]{src/active_response_crypto_ransomware/custom-ransomware.py}
\linej

\section*{Testing with real crypto ransomware}
The configuration is exactly the same as with the previous cases in \ref{lst:trid_cmd}, \ref{lst:trid_ps1}, \ref{lst:change-registry-value_0} to \ref{lst:delete-registry}, \ref{lst:free_space_ps1} and \ref{lst:free_space_sh}.
It changes slightly in the rest, included here:
\linej
\lstinputlisting[style=xml,caption=Configuration in \textit{/var/ossec/etc/ossec.conf} for the active response and integrator modules,captionpos=b,label={lst:real_crypto_ransomware_testing}]{src/crypto_ransomware_testing_configuration/manager/ossec.conf}
\linej
\lstinputlisting[style=xml,caption=Configuration in \textit{/var/ossec/etc/shared/default/agent.conf} for the configuration of the Windows agent from the server,captionpos=b]{src/crypto_ransomware_testing_configuration/manager/agent.conf}
\linej
\lstinputlisting[style=xml,caption=Wazuh rules for detection in \textit{/var/ossec/etc/rules/local\_rules.xml},captionpos=b]{src/crypto_ransomware_testing_configuration/manager/rules.xml}
\linej
\lstinputlisting[style=python,caption=Python script in \textit{/var/ossec/integrations/} to parse the triggering alert and execute a remote script with ssh,captionpos=b,label={lst:custom_ransomware_py_1}]{src/crypto_ransomware_testing_configuration/manager/custom-ransomware.py}
\linej
\lstinputlisting[style=python,caption=Alternative Python script in \textit{/var/ossec/integrations/} to parse the triggering alert and kill the executable with ssh,captionpos=b,label={lst:custom_ransomware_py_2}]{src/crypto_ransomware_testing_configuration/manager/custom-ransomware_2.py}
\linej

\lstinputlisting[style=PS,caption=PowerShell script to stop and mitigate the execution of crypto ransomware,captionpos=b,label={lst:crypto_ransomware_testing_configuration_stop}]{src/crypto_ransomware_testing_configuration/agent/stop.ps1}
\linej
\lstinputlisting[style=xml,caption=Syscheck local configuration in the \textit{ossec.conf} file on the agent,captionpos=b]{src/crypto_ransomware_testing_configuration/agent/ossec.conf}
\linej
\lstinputlisting[style=xml,caption=Sysmon configuration,captionpos=b]{src/crypto_ransomware_testing_configuration/agent/sysmon.xml}
\linej

\section*{Others}
\lstinputlisting[style=xml,caption=Enable Sysmon log forwarding to Wazuh in \textit{/var/ossec/etc/shared/default/agent.conf},captionpos=b,label={lst:forward_sysmon}]{src/forwarding/forward_sysmon.xml}
\linej
\lstinputlisting[style=xml,caption=Enable PowerShell log forwarding to Wazuh in \textit{/var/ossec/etc/shared/default/agent.conf},captionpos=b,label={lst:forward_powershell}]{src/forwarding/forward_powershell.xml}
\linej
\lstinputlisting[style=xml,caption=Enable Windows Defender log forwarding to Wazuh in \textit{/var/ossec/etc/shared/default/agent.conf},captionpos=b,label={lst:forward_Windows_Defender}]{src/forwarding/forward_Windows_Defender.xml}
\linej


%\cleardoublepage
%\include{capitulos/licenza}

\cleardoublepage
\include{capitulos/bibliografia}

\end{document}


