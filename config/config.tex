\newcommand\realnumberstyle[1]{}

\makeatletter
\newcommand{\zebra}[3]{%
    {\realnumberstyle{#3}}%
    \begingroup
    \lst@basicstyle
    \ifodd\value{lstnumber}%
        \color{#1}%
    \else
        \color{#2}%
    \fi
        \rlap{\hspace*{\lst@numbersep}%
        \color@block{\linewidth}{\ht\strutbox}{\dp\strutbox}%
        }%
    \endgroup
}
\makeatother

%%%%%%%%%%%%%%%%%%%%%%%%%		MORE	ALIASES		%%%%%%%%%%%%%%%%%%%%%%%%%

\newcommand{\lineh}{\rule{\textwidth}{1pt}\hfill\break}
\newcommand{\linej}{\hfill\break}

\newcommand{\projecthours}{417}

\newcommand{\IncrementoSiete}{VirusTotal integration}
\newcommand{\IncrementoSeis}{Additional detection for GNU/Linux}
\newcommand{\IncrementoCinco}{Explore solutions in problems with GPDR}
\newcommand{\IncrementoCuatro}{Adapt Wazuh configuration to typical requirements from enterprises}
\newcommand{\IncrementoTres}{Detection/action against ransomware}
\newcommand{\IncrementoDos}{Use of more data sources}
\newcommand{\IncrementoUno}{Common attacks in Windows Server}


\newcommand{\RNO}{\cellcolor{red!60}No}
\newcommand{\RYES}{\cellcolor{green!60}Yes}

%%%%%%%%%%%%%%%%%%%%%%%%%		END BASIC CONF		%%%%%%%%%%%%%%%%%%%%%%%%%



%\definecolor{mygreen}{rgb}{0,0.6,0}
%\definecolor{mygray}{rgb}{0.5,0.5,0.5}


%\renewcommand{\thesubsection}{\thesection.\alph{subsection}}		%change subsection style to letters

%https://en.wikibooks.org/wiki/LaTeX/Source_Code_Listings


\input{config/latex-listings-powershell}
\lstdefinestyle{PS}{
	language=Powershell,		%for syntax highlighting
	basicstyle=\footnotesize\ttfamily,
	%numbers=left,
	numbers=none,
	numberstyle=\tiny\zebra{gray!10}{white},
	frame=tb,
	postbreak=\mbox{\textcolor{red}{$\hookrightarrow$}\space},
	tabsize=2,
	columns=fixed,
	showstringspaces=false,
	showtabs=false,
	keepspaces,
	commentstyle=\color{red},
	stringstyle=\color{gray},		%needs xcolor
	keywordstyle=\color{blue}
}

\lstdefinestyle{sh} {
	language=sh,		%for syntax highlighting
	basicstyle=\footnotesize\ttfamily,
	numbers=none,
	frame=tb,
	postbreak=\mbox{\textcolor{red}{$\hookrightarrow$}\space},
	tabsize=2,
	columns=fixed,
	showstringspaces=false,
	showtabs=false,
	keepspaces,
	commentstyle=\color{red},
	stringstyle=\color{gray},		%needs xcolor
	keywordstyle=\color{blue},
	%directivestyle=\color{gray},	%shebang color
	emph={local, export },
	emphstyle=\color{green},
}

\lstdefinestyle{ruby} {
	language=Ruby,		%for syntax highlighting
	basicstyle=\footnotesize\ttfamily,
	%numbers=left,
	numbers=none,
	numberstyle=\tiny\zebra{gray!10}{white},
	frame=tb,
	postbreak=\mbox{\textcolor{red}{$\hookrightarrow$}\space},
	tabsize=4,
	columns=fixed,
	showstringspaces=false,
	showtabs=false,
	keepspaces,
	commentstyle=\color{red},
	stringstyle=\color{gray},		%needs xcolor
	keywordstyle=\color{blue}
}

\lstdefinestyle{xml} {
	language=XML,		%for syntax highlighting
	basicstyle=\footnotesize\ttfamily,
	%numbers=left,
	numbers=none,
	numberstyle=\tiny\zebra{gray!10}{white},
	frame=tb,
	postbreak=\mbox{\textcolor{red}{$\hookrightarrow$}\space},
	tabsize=4,
	columns=fixed,
	showstringspaces=false,
	showtabs=false,
	keepspaces,
	commentstyle=\color{red},
	stringstyle=\color{gray},		%needs xcolor
	keywordstyle=\color{blue}
}

\lstdefinestyle{txt} {
	language=Ruby,
	basicstyle=\scriptsize\ttfamily,
	%numbers=left,
	numbers=none,
	numberstyle=\tiny\zebra{gray!10}{white},
	frame=tb,
	postbreak=\mbox{\textcolor{red}{$\hookrightarrow$}\space},
	tabsize=4,
	columns=fixed,
	showstringspaces=false,
	showtabs=false,
	keepspaces,
	commentstyle=\color{black},
	stringstyle=\color{black},
	keywordstyle=\color{black}
}

%\lstdefinestyle{C} {
%	language=C,		%for syntax highlighting
%	basicstyle=\footnotesize\ttfamily,
%	frame=tb,
%	tabsize=4,
%	columns=fixed,
%	showstringspaces=false,
%	showtabs=false,
%	keepspaces,
	%commentstyle=\color{gray},
	%stringstyle=\color{brown},
	%keywordstyle=\color{blue},
	%directivestyle=\color{gray},			%preprocessor color
	%emph={int,char,double,float,unsigned},
	%emphstyle=\color{green},
%}
%\lstset{ language=C }	%		\begin{lstlisting}[style=C]		\lstinputlisting[style=C]{main.c}



%\lstdefinestyle{sql} {
%	language=sql,		%for syntax highlighting
%	basicstyle=\footnotesize\ttfamily,
%	frame=tb,
%	tabsize=4,
%	columns=fixed,
%	showstringspaces=false,
%	showtabs=false,
%	keepspaces,
%	commentstyle=\color{red},
%	stringstyle=\color{orange},		%needs xcolor
%	keywordstyle=\color{blue}
%}
%\lstset{ language=sql }		\begin{lstlisting}		\lstinputlisting[style=sql]{file.sql}




%\lstdefinestyle{python} {
	%language=python,		%for syntax highlighting
	%basicstyle=\footnotesize\ttfamily,
	%frame=tb,
	%tabsize=4,
	%columns=fixed,
	%showstringspaces=false,
	%showtabs=false,
	%keepspaces,
	%commentstyle=\color{gray},
	%stringstyle=\color{brown},
	%keywordstyle=\color{blue},
	%directivestyle=\color{gray},			%shebang color
	%emphstyle=\color{green},
%}
%\lstset{ language=python }			%\begin{lstlisting}		\lstinputlisting[style=python]{script.py}



%\lstdefinestyle{file} {
	%basicstyle=\footnotesize\ttfamily,
	%frame=tblr,
	%tabsize=4,
	%columns=fixed,
	%showstringspaces=false,
	%showtabs=false,
	%keepspaces,
	%title=\lstname
%}


%\lstset{		%basicstyle=\small\sffamily,
	%basicstyle=\footnotesize\ttfamily,
	%numbers=left,
	%numberstyle=\tiny\zebra{green!10}{white},
	%frame=tb,
	%tabsize=4,
	%columns=fixed,
	%showstringspaces=false,
	%showtabs=false,
	%keepspaces,
	%commentstyle=\color{red},
	%stringstyle=\color{orange},
	%extendedchars=true,
	%literate={á}{{\'a}}1
	%{é}{{\'e}}1
	%%{í}{{\'{\i}}}1
	%{í}{{\'i}}1
	%{ó}{{\'o}}1
	%{ú}{{\'u}}1
	%{Á}{{\'A}}1
	%{É}{{\'E}}1
	%{Í}{{\'I}}1
	%{Ó}{{\'O}}1
	%{Ú}{{\'U}}1
	%%{ü}{{\"u}}1
	%%{Ü}{{\"U}}1
	%{ñ}{{\~n}}1
	%{Ñ}{{\~N}}1
	%{¿}{{?``}}1
	%{¡}{{!``}}1,
	%keywordstyle=\color{blue}
%}




%atm general file listing with
%\lstinputlisting[style=file]{}


%%%%%%%%%%%%%%%%%%%%%%%%%		END CURRENT DOCUMENT CONF		%%%%%%%%%%%%%%%%%%%%%%%%%

