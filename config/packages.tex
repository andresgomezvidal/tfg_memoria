% pódense engadir todos os packages necesarios
\usepackage[utf8]{inputenc}
%\usepackage[T1]{fontenc}	%Correct accented characters and copy paste them, also not unexpected results with some characters like pipe


%				%es-noquotes]{babel}			si caracteres dan problemas
%				%es-noshorthands]{babel}		si caracteres dan todavia problemas
%				%es-noindentfirst]{babel}		para eliminar sangría
%\usepackage[latin1]{inputenc}
%\usepackage[galician]{babel}
%\usepackage[spanish]{babel}
%\usepackage[spanish, es-tabla,es-noindentfirst]{babel}		%http://www.tex-tipografia.com/spanishopt.html
%\usepackage[spanish, es-tabla]{babel}		%http://www.tex-tipografia.com/spanishopt.html
\usepackage[english]{babel}		%http://www.tex-tipografia.com/spanishopt.html
\addto\captionsenglish{
  \renewcommand{\contentsname}
    {Index} % ToC will show "Index" instead of "Content"
}

%\usepackage[document]{ragged2e}	%centering

\usepackage{enumitem}

\usepackage{graphicx}
\usepackage[dvips]{epsfig}
\usepackage{amssymb}
\usepackage{eurosym}
\usepackage{float}
\usepackage{latexsym}
\usepackage{a4}
\usepackage{hyperref}

%tables
\usepackage{multirow}
\usepackage{multicol}
%\usepackage{changepage}		%margin, eg for tables before tabular \begin{adjustwidth}{-1.4cm}{}
\usepackage{tabularx}

%\usepackage{color}
\usepackage[table,xcdraw,usenames,dvipsnames]{xcolor}


%WBS
\usepackage{forest}
\usetikzlibrary{arrows.meta,shapes,positioning,shadows,trees}
\tikzset{
    basic/.style  = {draw, text width=2cm, drop shadow, font=\sffamily, rectangle},
    root/.style   = {basic, rounded corners=2pt, thin, align=center,
                     fill=green!30, text width=2cm,},
    onode/.style = {basic, thin, rounded corners=2pt, align=left, fill=green!60,text width=6cm,},
    tnode/.style = {basic, thin, align=left, fill=gray!30, text width=5cm},
    edge from parent/.style={draw=black, edge from parent fork right}
}


\usepackage{pgfgantt}


\usepackage{amsmath}		%https://www.sharelatex.com/learn/Aligning_equations_with_amsmath
\usepackage{amsfonts}		%Extended set of fonts for use in mathematics

%%\usepackage{afterpage}		%Commands specified in its argument are expanded after the current page is output. EG: \afterpage{\clearpage} and the current page will be filled up with text as usual, but then the \clearpage command will flush out all the floats before the next text page begins

\usepackage{listings}
	\lstset{breaklines}			%line wrap for listings
	\newcommand{\lstlistinginput}{\lstinputlisting}		%alias


\usepackage{courier}		%listings better font
\usepackage[sorting=none]{biblatex} %use with \cite{} and \printbibliography[type=online,heading=subbibliography,title={Referencias web}]
	\addbibresource{config/bib.bib}



%%\usepackage{microtype}		%Improves spacing (words and letters) and more stuff. Load after fonts, because is dependent on this font
%%\usepackage{siunitx}		%SI units
%%\usepackage{xspace}			%Decides whether to insert a space to replace one "eaten" by the command decoder
%%\usepackage{todonotes}		%Mark things to do later
%\usepackage{hyperref}		%\hyperref, \url and \href
%%\usepackage[a4paper]{geometry}	%Margins without needing to remember the particular page dimensions commands. Eg [a4paper], [top=1in, bottom=1.25in, left=1.25in, right=1.25in]
%%\usepackage{cleveref}		%LAST \usepackage in the preamble. If anything else modifies the referencing system (like amsmath) it all goes wrong
%
%%\DisableLigatures{encoding = *, family = *}		%https://en.wikibooks.org/wiki/LaTeX/Text_Formatting#Ligatures
%%\def\spanishoperators{}		%stuff like sin traduced to sen, max to máx, etc


